\newpage

\chapter{Introduction}
\label{chap:introduction}

Decision Trees and Random Forests are among the most popular 
methods for classification tasks. Decision Trees, specially small ones, are easy to interpret,
while Random Forests usually yield more accurate classifications. One of the key issues in these methods
is how to select an attribute to associate with a node of the tree/forest. An important
related issue is how to split the samples once the attribute is selected.

There is a number of papers  discussing aspects related with 
attribute selection, such as:
how to design criteria to evaluate the quality of different types of attributes;
whether binary or multi-way splits shall be used and
how to remove bias from  splitting criteria.
For recent surveys on this topic we refer to \cite{books/sp/datamining2005/RokachM05,
Loh2014,series/sbcs/BarrosCF15}.

Many criteria, with different properties,  have been proposed to evaluate 
the quality of different types of attributes, including
continuous and categorical ones.  Among the most popular criteria,
we have the Gini Gain and the Information Gain \cite{}.

Despite the large body of work we believe  there are still questions to be answered.
One of them is to how to  properly handle nominal  attributes that may assume a large number of values.
Before explaining the reason behind our  statement we would
like to remark that this kind of attribute
appears naturally in some applications  (e.g.: states of a country or letters from some alphabet).
In addition, they may arise as the result of aggregating
attributes that have few distinct values
with the goal of capturing possible correlation between them, as pointed out by \cite{Chou:91}.
As an example, consider 5 binary attributes (e.g. medical tests) and a
target binary variable  that has large probability of being positive if at least $3$ out
of the $5$ binary tests are positives. By aggregating
the $5$ binary variables we obtain a new attribute with $2^5=32$
values that  captures  this relation. 
If we used the 5 attributes separately we would 
need 5 levels in the tree to be able to capture the relation between
them and the target class, thus 
incurring a large fragmentation of the set of samples.

To properly face  multi-valued nominal attributes we have to deal with the computational time required to compute good splits.
Our contribution, explained in the next section, is related with this issue. 
 
A brute force search to compute the best binary split 
requires $\Omega(2^n)$ time, where $n$ is the number of distinct values the attribute may assume. The computational efficiency can be improved if a $n$-ary split is used rather than a binary one. However,  this may lead to a severe fragmentation of the sample space, which is not desirable: the number of samples available for each of the children of the split node 
may be small and, as a consequence, the underlying classification tasks may become significantly more difficult.
When the target variable is binary, the Gini Gain, proposed
in the influential monography by Breiman et al \cite{Breiman84}, can be computed efficiently.
However, when the number of classes $k$ is larger than 2,
most, if not all, of the available exact solutions take exponential time
in $(n,k)$.
The Twoing method \cite{Breiman84}, which is equivalent to
Gini Gain when $k=2$, 
is an  interesting case since its running time is $O(2^{\min\{n,k\}})$ rather than $O( 2^ n)$.

When both $n$ and $k$ are large, in the sense that an exhaustive search does not run in a reasonable time, one can rely on heuristics to compute the best binary split.
As an example, the GUIDE algorithm  \cite{Loh2009}, the  last
of a series of algorithms/developments designed by Loh and its contributors, 
deals with a nominal variable $X$
as follows: if $k=2$ or $n \le 11$ the Gini Index is computed;
if $k \le 11$ and $n > 20$ a new variable $X'$ with at most $k$ distinct values is created according to a certain rule and an exhaustive search is performed over it;  finally, if $k > 11$ or $n \le 20$,  $X$ is binarized and a Linear Discriminant Analysis (LDA) is employed.
These rules reflect the difficulty in dealing with multi-valued nominal attributes.
In general, the main drawback of using heuristics is the lack of a theoretical guarantee about their behavior. 

\section{Our Contribution}
\label{sec:contribution}

Given this scenario,  in chapter \ref{chap:framework} we propose
a framework for designing criteria, with nice theoretical properties, for evaluating the quality of 
 multi-valued nominal attributes.
Criteria generated according to this framework
run in polynomial time in both the number of values and classes and
have a theoretical guarantee that they are close to optimal.
The key idea consists of formulating the problem
of finding the best binary partition for a given attribute $A$ as the  problem of finding a 
cut with maximum weight  in a complete graph whose nodes are associated with the values that $A$ may assume and the edges' weights capture the benefit of putting
values in different partitions. The  motivation behind the use of the max-cut problem is 
the existence of efficient algorithms with 
approximation guarantee, in particular 
the one proposed  by \cite{GoeWil95}, with $0.878$ approximation,
and  local search  algorithms with 0.5-approximation \cite{journals/corr/AngelBPW16}.


We discuss two criteria that are derived from this framework:
the first one  can be seen as a natural variation of the
Gini Gain, while the second criterion uses the $\chi^2$-test  to set the edges' weights. For that, each
edge $e_{ij}$, between nodes  $v_i$ and $v_j$,
is thought as a binary attribute $A(i,j)$, with values $v_i$ and $v_j$.
After discussing these criteria, we show how to extend them to handle
numeric attributes.

We also  present a number of experiments 
that suggest that one of our  criteria 
is  competitive with the Twoing method, which 
is -- as far as we know -- the only well-established
criterion with binary splits that can be optimally computed for large $n$ when $k > 2$.
However, in contrast with our methods, Twoing cannot handle
datasets that also have a large number of classes.
In addition, the experiments also  provide 
evidence of the potential of aggregating  attributes for improving
the accuracy of decision trees.


\section{Organization}
\label{sec:organization}
In chapter \ref{chap:background} we explain how decision trees are used for classification problems and how they are constructed. We also present the main impurity measures and splitting criteria used in the literature, together with their execution-time complexity.

Chapter \ref{chap:framework} contains the framework for generating splitting criteria that run in polynomial time. Its relation with the Max-Cut problem and its approximation algorithms are explained and some criteria obtained from this framework are presented.

Later, in chapters \ref{chap:experiments-splits} and \ref{chap:experiments-datasets}, we compare all the existing criteria and see how they perform in practice. In chapter \ref{chap:experiments-splits} we explore how the many heuristics used to find splits with optimal impurity perform compare with the exact methods. This suggests a couple of criteria that perform better and can be used when the number of values and classes are high. In chapter \ref{chap:experiments-datasets} we analyse these heuristics together with exact methods on real datasets that contain attributes with large number of values and classes. Lastly, in chapter \ref{chap:con	clusions} we present our study conclusions.