\newpage

\chapter{Conclusions}
\label{chap:conclusions}


%TODO: PODEMOS FIXAR UM NUMERO MAXIMO DE ITERACOES E QUANDO n e pequeno podemos rodar a %versao exata.

In this paper we proposed a framework for
designing splitting criteria for handling 
multi-valued nominal attributes.
Criteria derived from our framework 
can be implemented to run in polynomial time in
$n$ and $k$, with 
theoretical guarantee of producing a split that is close to the optimal
one.

Experiments over 11 datasets
suggest that the GL$\chi^2$ criterion, obtained from our framework, is competitive with the well-established Twoing criterion in terms of both accuracy and speed for datasets with a small number of classes ($k \leq 7$). It is also much faster than Twoing when the number of classes is greater than 10, while keeping a comparable accuracy.
 
Therefore, our methods are an interesting alternative to deal with
datasets with a large number of classes that contain nominal attributes with a large
number of different values, since those cannot be properly handled by Twoing due to its exponential running time dependence on the number of classes.

Furthermore, our experiments also reinforce
the potential of  aggregating attributes as a tool 
for improving the accuracy of decision trees.
An interesting topic for  future research is evaluating the behavior of our criteria in boosted tree methods.
Another direction for future work is developing new methods for automatic aggregating attributes, or improving the available ones.

