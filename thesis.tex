\documentclass[msc,british,bibkey]{ThesisPUC_uk}

%---------- Math ----------%
\usepackage{amsmath}
\usepackage{amssymb}
%\usepackage{amsthm}
\usepackage{mathtools}
\usepackage{bbm}
%---------- Floting ----------%
\usepackage{float}
%---------- References ----------%
\usepackage{natbib}
%---------- Algorithm ----------%
\usepackage{algorithm}
\usepackage{algorithmic}
%---------- Tables ----------%
\usepackage{rotating}
\usepackage{tabularx}
\usepackage{multicol}
\usepackage{multirow}
\usepackage{booktabs}
\usepackage{subcaption}
\usepackage{diagbox}
%---------- Others ----------%
\usepackage{hyperref}
\usepackage{bold-extra}
\usepackage{graphicx}

%\colorlet{dark}{red!85!blue!60!black}

%---------- Cover ----------%
\author{\mbox{Felipe} \mbox{de} \mbox{Albuquerque} \mbox{Mello} \mbox{Pereira}}
\authorR{\mbox{Pereira}, \mbox{Felipe} \mbox{de} \mbox{Albuquerque} \mbox{Mello}}
\advisor{\mbox{Eduardo} \mbox{Sany} \mbox{Laber}}
\advisorR{\mbox{Laber}, \mbox{Eduardo} \mbox{Sany}}
\title{Binary Splitting Criteria for Large Categorical Attributes in Decision Trees}
\day{28$^{th}$} \month{February} \myyear{2018}

\city{Rio de Janeiro}
\CDD{004}
\department{Inform\'atica}
\departmentbr{Inform\'atica}
\program{Inform\'atica}
\programbr{Inform\'atica}
\school{Centro T\'{e}cnico Cient\'{\i}fico}
\university{Pontif\'{\i}cia Universidade Cat\'{o}lica do Rio de Janeiro}
\uni{PUC-Rio}

%---------- Jury ----------%

\jury{
  \jurymember{H\'elio C\^ortes Vieira Lopes}{Departamento de Inform\'atica --- PUC-Rio}
  \jurymember{Marco Serpa Molinaro}{Departamento de Inform\'atica --- PUC-Rio}
  \jurymember{M\'arcio da Silveira Carvalho}{Vice Dean of Graduate Studies\\ Centro T\'ecnico Cient\'ifico da PUC-Rio}
}

%---------- Front letters ----------%
\resume
{
Bachelor's in Electrical Engineering and Pure Mathematics at the Pontif\'icia Universidade Cat\'olica do Rio de Janeiro (2010 and 2011). Masters' in Mathematics at the Pontif\'icia Universidade Cat\'olica do Rio de Janeiro (2013).
}

\acknowledgment
{
TODO: acknowledgment.

Thanks to CNPq for the conceded scholarship during my Masters.
}

\keywords
{
  \key{Decision Trees; Max-cut Problem; Approximated Algorithms}
}

\abstract{
In this dissertation we proposed a framework for designing splitting criteria for handling multi-valued nominal attributes for decision trees. Criteria derived from our framework can be implemented to run in polynomial time in the number of classes and values, with theoretical guarantee of producing a split that is close to the optimal one. This is the only known criteria that have all these characteristics simultaneously. We also run multiple experiments to evaluate its running times and accuracy in real datasets.
}

\titulobr{Crit\'erios de Splits Bin\'arios para Atributos Categóricos Grandes em \'Arvores de Decis\~ao}
\departamentobr{Inform\'atica}

\chavesbr
{
  \chave{\'Arvores de Decis\~ao; Problema de Corte M\'aximo; Algoritmos Aproximativos}
}

\resumobr
{
Nesta disserta\c{c}\~ao \'e apresentado um framework para desenvolver crit\'erios de split para lidar com atributos nominais multi-valorados em \'arvores de decis\~ao. Crit\'erios gerados por este framework podem ser implementados para rodar em tempo polinomial no n\'umero de classes e valores, com garantia te\'orica de produzir um split pr\'oximo do \'otimo. Este \'e o \'unico crit\'erio conhecido que possui ambas caracter\'isticas simultaneamente. Tamb\'em s\~ao realizados m\'ultiplos experimentos para avaliar seu tempo de execu\c{c}\~ao e acur\'acia em datasets reais.
}

\tablesmode{figtab}


%%%%%%%%%%%%%%%%%%%%%%%%%%%%%%%%%%%%%%%%%%%%%%%%%%%%%%

\begin{document}

\newcommand{\remove}[1]{}
\newcommand{\dist}[3]{d(#1, #2)}
\newcommand{\distC}[2]{d(#1, #2)}
\newcommand{\OPT}[1]{\textrm{OPT}(#1)}
\newcommand{\OPTf}[2]{\textrm{OPT}(#1, #2)}
\newcommand{\Cf}[2]{\textrm{cost}(#1, #2)}
\newcommand{\C}[1]{\textrm{cost}(#1)}
\newcommand{\ans}[1]{\textbf{#1}}
\newcommand{\bl}{\textrm{blocked}}
\newcommand{\un}{\textrm{unassigned}}
\newcommand{\comments}[1]{}
\newcommand{\commento}[1]{\marginpar{\tiny \flushleft{#1}}}
\newtheorem{lemma}{Lemma}
\newtheorem{definition}{Definition}[section]

\newpage

\chapter{Introduction}
\label{chap:introduction}

Decision Trees and Random Forests are among the most popular 
methods for classification tasks. Decision Trees, specially small ones, are easy to interpret,
while Random Forests usually yield more accurate classifications. One of the key issues in these methods
is how to select an attribute to associate with a node of the tree/forest. An important
related issue is how to split the samples once the attribute is selected.

There is a number of papers  discussing aspects related with 
attribute selection, such as:
how to design criteria to evaluate the quality of different types of attributes;
whether binary or multi-way splits shall be used and
how to remove bias from  splitting criteria.
For recent surveys on this topic we refer to \cite{books/sp/datamining2005/RokachM05},
\cite{Loh2014} and \cite{series/sbcs/BarrosCF15}.

Many criteria, with different properties,  have been proposed to evaluate 
the quality of different types of attributes, including
continuous and categorical ones.  Among the most popular criteria,
we have the Gini Gain and the Information Gain (\cite{Breiman84}, \cite{quinlan2014c4}).

Despite the large body of work we believe  there are still questions to be answered.
One of them is to how to  properly handle nominal  attributes that may assume a large number of values.
Before explaining the reason behind our  statement we would
like to remark that this kind of attribute
appears naturally in some applications  (e.g.: states of a country or letters from some alphabet).
In addition, they may arise as the result of aggregating
attributes that have few distinct values
with the goal of capturing possible correlation between them, as pointed out by \cite{Chou:91}.
As an example, consider 5 binary attributes (e.g. medical tests) and a
target binary variable  that has large probability of being positive if at least $3$ out
of the $5$ binary tests are positives. By aggregating
the $5$ binary variables we obtain a new attribute with $2^5=32$
values that  captures  this relation. 
If we used the 5 attributes separately we would 
need 5 levels in the tree to be able to capture the relation between
them and the target class, thus 
incurring a large fragmentation of the set of samples.

To properly face  multi-valued nominal attributes we have to deal with the computational time required to compute good splits.
Our contribution, explained in the next section, is related with this issue. 
 
A brute force search to compute the best binary split 
requires $\Omega(2^n)$ time, where $n$ is the number of distinct values the attribute may assume. The computational efficiency can be improved if a $n$-ary split is used rather than a binary one. However,  this may lead to a severe fragmentation of the sample space, which is not desirable: the number of samples available for each of the children of the split node 
may be small and, as a consequence, the underlying classification tasks may become significantly more difficult.
When the target variable is binary, a family of impurity measures that include both the Gini Gain and the Information Gain can be computed efficiently, as shown
in the influential monograph by Breiman et al \cite{Breiman84}.
However, when the number of classes $k$ is larger than 2,
most, if not all, of the available exact solutions take exponential time
in $(n,k)$.
The Twoing method, also from \cite{Breiman84}, 
is an  interesting case since its running time is $O(2^{\min\{n,k\}})$ rather than $O( 2^ n)$ while being equivalent to
Gini Gain when $k=2$.

When both $n$ and $k$ are large, in the sense that an exhaustive search does not run in a reasonable time, one can rely on heuristics to compute the best binary split.
As an example, the GUIDE algorithm  \cite{Loh2009}, the  last
of a series of algorithms/developments designed by Loh and its contributors, 
deals with a nominal variable $X$
as follows: if $k=2$ or $n \le 11$ the Gini Index is computed;
if $k \le 11$ and $n > 20$ a new variable $X'$ with at most $k$ distinct values is created according to a certain rule and an exhaustive search is performed over it;  finally, if $k > 11$ or $n \le 20$,  $X$ is binarized and a Linear Discriminant Analysis (LDA) is employed.
These rules reflect the difficulty in dealing with multi-valued nominal attributes. Other interesting heuristics are the PC and PC-ext criteria, which calculate the principal component of the class probability vectors and uses the order given by the vector projections in this direction to look for splits.
In general, the main drawback of using heuristics is the lack of a theoretical guarantee about their behavior. 


\section{Our Contribution}
\label{sec:contribution}

Given this scenario,  in chapter \ref{chap:framework} we propose
a framework for designing criteria, with nice theoretical properties, for evaluating the quality of 
 multi-valued nominal attributes.
Criteria generated according to this framework
run in polynomial time in both the number of values and classes and
have a theoretical guarantee that they are close to optimal.
The key idea consists of formulating the problem
of finding the best binary partition for a given attribute $A$ as the  problem of finding a 
cut with maximum weight  in a complete graph whose nodes are associated with the values that $A$ may assume and the edges' weights capture the benefit of putting
values in different partitions. The  motivation behind the use of the max-cut problem is 
the existence of efficient algorithms with 
approximation guarantee, in particular 
the one proposed  by \cite{GoeWil95}, with $0.878$ approximation,
and  local search  algorithms with 0.5-approximation as shown in \cite{journals/corr/AngelBPW16}.


We discuss two criteria that are derived from this framework:
the first one  can be seen as a natural variation of the
Gini Gain, while the second criterion uses the $\chi^2$-test  to set the edges' weights. For that, each
edge $e_{ij}$ between nodes  $v_i$ and $v_j$
is thought as a binary attribute $A(i,j)$ with values $v_i$ and $v_j$.
After discussing these criteria, we show how to extend them to handle
numeric attributes.

We also  present a number of experiments that suggest that one of our criteria is competitive with the Twoing method, which is -- as far as we know -- the only well-established criterion with binary splits that can be optimally computed for large $n$ when $k > 2$. However, in contrast with our methods, Twoing cannot handle datasets that also have a large number of classes. Some criteria based on heuristics, such as the PC-ext and Hypercube Cover, are also part of the comparison. In addition, the experiments also  provide evidence of the potential of aggregating  attributes for improving the accuracy of decision trees.

\section{Related Work}
\label{chap:relatedwork}

Many splitting criteria have been proposed to 
deal with continuous and nominal attributes.
Arguably, the Gini Gain---used by CART---and entropy-based measures---such as 
the Information Gain, adopted by C4.5---are among
the most popular (\cite{books/sp/datamining2005/RokachM05,
Loh2014,series/sbcs/BarrosCF15}).

There has been  some investigation on 
methods to compute the best split efficiently 
(\cite{Breiman84,Chou:91,BPKN:92,journals/datamine/CoppersmithHH99}).
For the 2-class problem,  \cite{Breiman84} proved a theorem which states that an optimal
binary partition, for a certain class of splitting criteria,
can be determined in linear time on $n$ (TODO: qual o erro?), the number of distinct values of the attributes, after ordering.
The Gini  Gain belongs to this class.
The  other  three papers generalize
this theorem  in different directions 
and show necessary conditions that are satisfied by optimal partitions for a certain class of splitting criteria. 
These conditions, though useful to restrict the set of partitions
that need to be considered, do not yield  a method that
is efficient (polynomial time) for large values of $n$ and $k$. These papers also  present  heuristics, without approximation guarantee, to obtain good splits.
Another related result is a theorem from \cite{journals/datamine/CoppersmithHH99} that guarantees that the optimum split can be found by separating the class probability vectors by a hyperplane. This motivated the creation of the PC criterion, as will be shown in the next chapter. TODO: falar do Sliq e FlipFlop.

Other proposals to  speed up the attribute selection phase
 include  \cite{MolaSiciliano1997,Shih2001}. 
The first presents a simple  heuristic
to reduce the number of binary splits considered to
choose the best nominal variable among the $m$ available ones.
 The second   extends the method for another class
of impurity measures.

In order to properly
handle nominal attributes with a large number of values,
apart from efficiently computing good splits, it is
important to prevent bias in the attribute selection.
Indeed, it is widely  known that many splitting criteria have bias toward
attributes with a large number of values. There are some  proposals available
to cope with this issue 
(\cite{conf/icml/DobraG01,Shih2004,Hothorn:2006:URP}). 
This topic, though relevant, is not the focus of this dissertation.

\section{Organization}
\label{sec:organization}
In chapter \ref{chap:background} we explain how decision trees are used for classification problems and how they are constructed. We also present the main impurity measures and splitting criteria used in the literature, together with their execution-time complexity.

Chapter \ref{chap:framework} contains the framework for generating splitting criteria that run in polynomial time. Its relation with the Max-Cut problem and its approximation algorithms are explained and some criteria obtained from this framework are presented.

Later, in chapters \ref{chap:experiments-splits} and \ref{chap:experiments-datasets}, we compare the criteria and see how they perform in practice. In chapter \ref{chap:experiments-splits} we explore how the many heuristics used to find splits with optimal impurity compare among themselves. This suggests a couple of criteria that perform better and can be used when the number of values and classes are large. In chapter \ref{chap:experiments-datasets} we analyze these methods on real datasets that contain attributes with large number of values and classes. Lastly, in chapter \ref{chap:conclusions} we present our study conclusions.
\newpage

\chapter{Background}
\label{chap:background}

\section{Decision Trees}

TODO: explicar como sao usadas pra classificacao com um exemplo/imagem.

\section{Notation}
\label{sec:notation}
We adopt the following notation throughout the dissertation.
Let $S$ be a set of $N$ samples and 
 $C=\{c_1,\ldots,c_k\}$ be the domain of the class label. 
In addition, for an attribute  $A$, we use $A(s)$ to denote the value taken by attribute
$A$ on sample $s$; we use 
  $V=\{ v_1,\ldots,v_n \}$ to denote the set of values
taken by $A$;
$A_{ij}$ to refer to the  number of samples
from class $c_j$ for which  $A$ takes value $v_i$; 
 $N_i$ for the number of samples with value $v_i$ for attribute $A$
and $S_j$ for the number of samples from class $c_j$.
Furthermore, we let $p_j = S_j /N$ and $p_{ij}= Pr[C=c_j | A = v_i]$.
We observe that the estimator of maximum likelihood for $p_{ij} $ is
$A_{ij} / N_i$.  

\section{Impurity Measures}
Many of the splitting criteria follow the structure of algorithm \ref{alg:create-tree}.

\begin{algorithm}[tb]
   \caption{CreateTree($S$: set of samples, $List\_A$: list of attributes information, $I$: split impurity measure)}
   \label{alg:create-tree}
\begin{algorithmic}
\IF{$S$ does not meet the stopping criterion}
\FOR{attribute $A$ in $List\_A$}
\STATE $s_A = $ split of $A$ with smallest impurity as measured by $I$
\ENDFOR
\STATE $(L, R) = s_{A^*}$ such that $I(s_{A^*}) \leq I(s_A) $ for any $A$
\STATE CreateTree($L$)
\STATE CreateTree($R$)
\ENDIF
\end{algorithmic}
\end{algorithm}

It is important to note that most impurity calculations on a nominal attribute are done based on what's called a contingency table. It consists of an $n\times k$ matrix where the $ij$-th entry corresponds to the number of samples that have value $i$ and belong to class $k$. We'll assume that every decision tree node has the contingency table pre-calculated for all nominal attributes, which takes $O(N)$ time for each attribute and cannot be avoided\footnote{The only exception is the attribute used to split the parent node, which can be calculated in $O(n\times k)$.}. Therefore, whenever we mention the time complexity for a decision tree constructing algorithm/criterion, this cost will not be mentioned since it does not affect their comparison.

A good place to start is by presenting the two most common impurity measures found in the literature. Both can be used to generate binary splits
and, as a consequence, binary decision trees.

\subsection{Gini}
\label{subsec:Gini}
The Gini Index for a set of samples $S$ is given by 
\begin{equation}
 Gini(S) =  1- \sum_{i=1}^k (p_i)^2 .
\label{eq:gini}
\end{equation}

The Gini Gain, $\Delta_G$, induced by a binary partition $(L,R)$ 
of the set of values $V$ is
given by 
\begin{equation}
 \Delta_G (L,R) = Gini(S) -
p_L Gini(S_L) - p_R Gini(S_R),
\label{eq:Ginigain}
\end{equation}
where $S_L= \{ s \in S | A(s) \in L \}$, $S_R= \{ s \in S | A(s) \in R \}$,
 $p_L=|S_L| /N $
and $p_R=|S_R| /N$. Therefore, the largest the Gini Gain is, the better the partition.

\subsection{Entropy}
The Entropy for a set of samples $S$ is given by 
\begin{equation}
 Entropy(S) =  - \sum_{i=1}^k p_i \log p_i
\label{eq:entropy}
\end{equation}

The Information Gain, $IG$, induced by a binary partition $(L,R)$ 
of the set of values $V$ is given by 
\begin{equation}
 IG(L,R) = Entropy(S) -
p_L Entropy(S_L) - p_R Entropy(S_R),
\label{eq:InformationGain}
\end{equation}
where $S_L= \{ s \in S | A(s) \in L \}$, $S_R= \{ s \in S | A(s) \in R \}$,
 $p_L=|S_L| /N $ and $p_R=|S_R| /N$. Therefore, the largest the Information Gain is, the better the partition.


\section{Splitting Criteria}
In this section we recall some well-known splitting criteria.

First note that, for numeric attributes, all criteria follow the same algorithm to choose the best split: the values are ordered and the valid splits have the form 
$$L = \{v_i : v_i \leq v\}$$
$$R = \{v_i : v_i > v\}$$
for some chosen value $v$. The criteria evaluate the impurity of these splits and chooses the one with the smallest impurity. Since we only have to evaluate at a single value $v$ between each pair of consecutive values $v_i$ and $v_{i+1}$, it takes $O(n \log n)$ time. Since this is polynomial and quite fast, its complexity is largely ignored throughout this dissertation.

\subsection{Gini Gain}
This criterion generates all $2^n$ binary values' split and the partition with maximum Gini Gain shall be selected.
As shown in \cite{Breiman84}, for the two-class problem this optimal partition  
can be computed  in $O(n \log n)$ time. First fix one of the two classes and calculate each value's frequency of samples on this class.
If we denote by $v_1,\ldots,v_n$ the values in this order, then the best binary split of $S$ will have the form 
$$L = \{v_i : v_i \leq v\}$$
$$R = \{v_i : v_i > v\}$$
for some chosen value $v$. Since we only need to test a single value of $v$ that is between each pair of consecutive values $v_i$, $v_{i+1}$, the time complexity follows.

For problems with more than two classes, however, there is no efficient procedure with theoretical approximation guarantee to compute the Gini Gain in subexponential time in $n$.

\subsection{Twoing}
The Twoing criterion
for a  binary partition $(L,R)$ 
of the set of values $V$ is given by
$$ 0.25 \cdot p_L \cdot p_R  \cdot \left ( \sum_{i=1}^k | p_L^i - p_R^i | \right )^2$$
where

$$ p_L^i= \frac{|\{s \in S_L: s \mbox{ belongs to class } c_i \}|}{ |S_L|} $$
 and 
$$ p_R^i= \frac{|\{s \in S_R: s \mbox{ belongs to class } c_i\} |}{ |S_R|} $$

When the Twoing criterion is used to generate binary decision trees, the binary partition with maximum twoing shall be selected at each node. 

As shown in \cite{Breiman84}, such partition can be calculated in $O(\min \{ n (k + \log n) 2^k, 2^n \} )$
time by considering all possibilities of partitioning the classes into two superclasses
and applying the Gini Gain criterion on each of them. We shall remark that, for the two-class problem, the Twoing criterion and the Gini Gain compute the same binary partitions.


\subsection{Hypercube Cover}

The Hypercube cover criterion works exactly the same as the Twoing criterion, except when it comes to the split impurity calculation. Instead of calculating the impurity based on the two superclasses, we calculate it using the original $k$ classes. This guarantees, for instance, that the split impurity, when measured with respect to the original classes, is never worse than that of Twoing. This method was first suggested in \cite{icml2018}, together with its approximation guarantee of 2 for both the Gini and Entropy impurities.


\subsection{Information Gain}
This criterion works exactly the same as the Gini Gain, but replacing the Gini impurity by the Entropy. First it generates all $2^n$ binary values' split and the partition with maximum Information Gain shall be selected.
For the two-class problem, the same result valid for the Gini Gain works here, and the optimal partition  
can be computed in $O(n \log n)$ time. Once again, when the number of classes is larger than 2 there is no efficient procedure with theoretical approximation guarantee to compute the Information Gain in subexponential time in $n$.

A related criterion is the Gain Ratio, where the Information Gain of an attribute is normalized by that attribute's potential information. This is used as a way of decreasing the bias of the k-ary Information Gain criterion towards attributes with larger number of values. Since we are only interested in binary splits in this dissertation, we will not go into its details.


\subsection{$\chi^2$-criterion}
The $\chi^2$ is a popular criterion that was  used in \cite{Mingers.87}. It is also the first one shown here not based on impurity measures, and it only generates k-ary (instead of binary) splits. Is is mentioned here because of its relation to the framework that will be presented in chapter \ref{chap:framework}.

The $\chi^2$-criterion chooses the attribute $A$ that maximizes
\begin{equation}
\label{eq:chitest}
\sum_{i=1}^n \sum_{j=1}^k \frac{(A_{ij}-E[A_{ij}] )^2}{E[A_{ij}]},
\end{equation}
where $E[A_{ij}]=N_i p_j$.

\subsection{Conditional Inference Trees}
Conditional Inference Trees are actually a framework for creating criteria that are bias-free when it comes to the number of values in an attribute. It was published by \cite{Hothorn:2006:URP} and still is the only known method of obtaining criteria that do not have any bias towards attributes with larger number of values.

It works by first choosing the best attribute to split at the current node and then evaluating all possible binary splits using any given impurity measure, picking the best one found.

To choose the attribute in which to split, first one has to calculate the permutation test's conditional expectation $\mu \in \mathbb{R}^{nk}$ and covariance $\Sigma \in \mathbb{R}^{nk\times nk}$ of every attribute\footnote{Since the formulae are very complicated and won't be used throughout this dissertation, they are ommited. The interested reader can obtain them in the section 3 of  \cite{Hothorn:2006:URP}.}. Then, in order to compare the attributes, we need to calculate the p-value of a univariate test statistics $c$ calculated on $\mu$ and $\Sigma$ of every attribute. The only exact form of doing this comparison is by using the quadratic form $c_{quad}$ (see equation \ref{eq:c_quad}), which follows an asymptotic $\chi^2$ distribution with degrees of freedom given by the rank of $\Sigma$. Since this involves the calculation of a pseudoinverse, which has cubic complexity on the dimension of $\Sigma$, this criterion can be very time consuming.

\begin{equation}
\label{eq:c_quad}
c_{quad}(t, \mu, \Sigma) = (t-\mu)\Sigma^+(t-\mu)\top
\end{equation}

This method, although very complicated and quite slow, is employed by the data mining community when accuracy counts for more than time spent training\footnote{Since this method does not have any bias when choosing which attribute to split, the accuracy of these trees tend to be higher than the trees obtained by biased criteria.}. Thus this criterion will be used in our experiments in chapter \ref{chap:experiments-datasets} to compare the different accuracies obtained when changing the splitting criterion used to choose the attribute's values split.

\section{Heuristics for Splitting Decision Tree Nodes}
As seen in the previous section, calculating the optimal binary split takes exponential time in the number of values or classes. Therefore many heuristics were created to construct decision trees in this situation. The most used ones are listed below. All of them work with any impurity measure (e.g.: Gini or Entropy), but some of them work best with just one of them. When this is the case, it will be mentioned.

%\subsection{SLIQ and SLIQ-ext}
%SLIQ was presented in \cite{mehta1996sliq} and it's a very simple greedy heuristic. Given an attribute, one starts with all the values going to the left split, and none on the right split. We then choose a value to go from the left to the right split. This value is the one that, when changing from the left to the right sides, decreases the impurity (increases the impurity gain) the most. This is repeated until there is no way of moving a value from the left to the right and decreasing the impurity.
%
%SLIQ-ext is a simple extension, where we keep changing values from the left to the right until the left side is empty (that is, we move from the left to the right even if that increases the impurity). Once again the value to move is chosen in a greedy fashion. SLIQ-ext returns the values' split seen that had the lowest impurity.
%
\subsection{PC and PC-ext}
These heuristics are based on the Principal Component of the contingency table and were presented in \cite{journals/datamine/CoppersmithHH99}.
They are based on a theorem that states that the optimal partition of values can be found by separating the class probability vectors by a hyperplane.

In other words, the theorem above states that, in order to find the optimal partition, we just need to choose the right hyperplane. This motivates the PC criterion, which look at the hyperplanes in $\mathbb{R}^k$ whose normal vector is the principal direction of the attribute's contingency table.

In more details, one first calculates the class probability distribution of every value, which is done by normalizing the contingency table rows to measure 1 in the sum norm. Then the values are grouped into ``supervalues'' where all values in the same supervalue have the same class probability distribution. Now the first principal component of these sypervalues' contingency table is calculated. One then calculates the inner product of each class probability vector of the supervalues  with the principal component and sort the supervalues by it. Denote by $v_1,\ldots,v_{n^*}$ the $n^*$ supervalues sorted in this manner and denote by $p$ the first principal component. We then calculate the impurity gain of the supervalues' splits of the form

$$L = \{v_i | v_i \cdot p \leq t\}$$
$$R = \{v_i | v_i \cdot p > t\}$$

where $t$ is a chosen threshold. Once we find the supervalues split with the largest impurity gain, we translate the supervalues into original values to obtain a valid partition.

PC-ext is a simple extension of this algorithm, where instead of only testing the supervalues splits given by the equations above, we also test the splits given by  exchanging the last supervalue on the left with the first supervalue on the right (where first and last are given by the order after calculating the inner product).

These procedures require $O(k^3)$ operations to find the principal component and $O(n)$ impurity calculations and inner products in the class probability space.

\subsection{Largest Class Alone}

This is a heuristic that, when using the Gini impurity, has an approximation guarantee of 2 compared with the optimal partition found by Gini Gain.

First one calculates the most frequent class and group the other classes in a single superclass. One then applies the Gini Gain criterion on this two-class problem. Since calculating the class frequencies can be done using the contingency table, this heuristic takes $O(N + n k + n \log n)$ time in total.

It is also proved that there is no other way of grouping classes into superclasses that has a smaller approximation guarantee for the Gini impurity. It can also be used with the Information Gain impurity, instead of Gini Gain, but its approximation guarantee increases to 3. These bounds are all proved in \cite{icml2018}.

\subsection{List Scheduling}

TODO: tirar do paper, ja que saiu da icml?

Similarly to the Largest Alone heuristic, List Scheduling has an approximation guarantee of 2 for the Informatio Gain criterion when using the Entropy impurity (TODO: conferir).

First one calculates the frequency of every class. Then, one uses a List Scheduling algorithm to group the classes into 2 superclasses as balanced as possible (in terms of number of samples). Lastly the Information Gain criterion is applied on this two-class problem. Again, since calculating the class frequencies can be done using the contingency table and the List Scheduling algorithm is linear in the number of classes, this heuristic also takes $O(N + n k + n \log n)$ time in total.

It is also proved that, for the entropy impurity, the best form of grouping classes into superclasses is by balancing them the best way possible.

\newpage

\chapter{Framework for Generating Splitting Criteria for Multi-valued Attributes}
\label{chap:framework}

First we recall some definitions and results for the Max-Cut problem. These definitions will be used in the following section, when we define our framework.

\section{The Maximum Weighted Cut Problem}
\label{sec:maxcutbackground}

We  recall some definitions from graph theory. A cut  $X$ in  a weighted graph $G=(V,E)$ is
a subset of vertexes of $V$. The weight of a cut $X$, denoted here by $w(X)$, is the sum of the weights
of the edges that have one endpoint in $X$ and the other one in $V-X$.

The problem of computing the  cut $X^*$ with maximum weight in a graph with non-negative weights is NP-Hard.
However, there are good  approximation algorithms
available. A remarkable one is the randomized algorithm
 proposed in  \cite{GoeWil95} that relies on a formulation of
the max-cut problem via semidefinite programming (SDP). This algorithm,  denoted throughout this dissertation by GW, 
returns a cut $X$ that satisfies  $E[w(X)] \ge 0.878 w(X^*)$. It involves solving an SDP on the graph weights matrix, calculating the Cholesky decomposition of it and then generating a random partition of the values based on the inner product of the decomposition column vectors with a randomly generated vector on the sphere of dimension $n$. As solving such an SDP takes $O(n^4)$ arithmetic operations (see \cite{navascues2009power}) and calculating the Cholesky decomposition takes $O(n^3)$ operations, the time complexity is high but polynomial.

Another possibility to solve the Max-Cut problem is by using the {\tt GreedyCut} algorithm, presented in Algorithm \ref{alg:greedy}. It
obtains a cut $X$ such that $w(X)  \ge 0.5 w(X^*)$, as proved in \cite{SahGon:76}.
The algorithm starts with two empty sets $X$ and $X'$. Then, it
scans the nodes 
and assigns each of them to the set that provides
the maximum improvement on  the weight of the current cut (ties are broken arbitrarily). Is it easy to see that the time complexity of this greedy algorithm is $O(n^2)$.


\begin{algorithm}[tb]
   \caption{ GreedyCut($V$: set of nodes)}
   \label{alg:greedy}
\begin{algorithmic}
\STATE{$X \leftarrow \emptyset$;$X' \leftarrow \emptyset$}
\FOR{$j=1,..,n$ }
\STATE{{\bf If} $$\sum_{v \in X} w(v_i,v) > \sum_{v \in X'-V} w(v_i,v) $$ add $v_i$ to $X'$ {\bf Else} add $v_i$ to $ X$  }
\ENDFOR
\STATE{{\bf Return} $X$ and $X'$ }

\end{algorithmic}
\end{algorithm}


The solutions obtained by both GW and {\tt GreedyCut} 
can be improved via a local search.
In its simplest version, it
moves a node from one group to
the other while some improvement on the cut weight is possible.
Although this algorithm is not polynomial  in the worst case,
it has polynomial behavior in the smoothed analysis framework (see \cite{journals/corr/AngelBPW16}). In addition, it is always possible
to set a limit on the number of moves.
A more refined version  allows exchanging a pair of nodes
as long as the weight of the cut is improved.
In our experiments this is the version we use, as presented in Algorithm \ref{alg:localsearch}.


\begin{algorithm}[tb]
   \caption{ LocalSearch($X$, $X'$): set of nodes}
   \label{alg:localsearch}
\begin{algorithmic}
\STATE{$label:~loop$\_$start$}
\FOR{$i = 1, ..., n$ }
\IF{switching $v_i$'s side improves cut weight}
\STATE switch $v_i$ and update cut weight, $X$, $X'$
\STATE $goto~ loop$\_$start$
\ENDIF
\ENDFOR
\FOR{pair $(v_i, v_j) \in X \times X'$ }
\IF{switching $v_i$ and $v_j$ improves cut weight}
\STATE{switch $v_i$ and $v_j$, update cut weight, $X$, $X'$}
\STATE{$goto~ loop$\_$start$}
\ENDIF
\ENDFOR
\STATE{{\bf Return} $X$, $X'$}

\end{algorithmic}
\end{algorithm}


\section{A Framework for Generating Splitting Criteria}
\label{sec:maxcut}
In this section we explain 
our approach to building binary splitting criteria
for  multi-valued nominal attributes.

Let $A$ be a nominal attribute  that takes
values in the domain $V=\{v_1,\ldots,v_n\}$.
Our framework to produce a splitting criterion $I$ 
consists of three steps:

\begin{enumerate}
\item  Create a complete graph $G=(V,E)$ with $n$ vertexes.

\item  Assign a non-negative weight $w_{ij}$ to the edge 
that connects $v_i$ to $v_j$. This value shall reflect the benefit of putting  $v_i$ and $v_j$ in different partitions.
Different definitions of $w_{ij}$ yield to different criteria,
as we explain further.

\item  Ideally, the value of the criterion $I$ for attribute
$A$ is the weight of the cut with maximum weight  in $G$.
However, this is not a reasonable possibility for large $n$ since, as mentioned before, the problem of computing the  cut $X^*$ with maximum weight in a graph with non-negative weights is NP-Hard.
Thus, the value of criterion $I$ is given by the weight
of the cut obtained by some  algorithm, with approximation guarantee, for the maximum cut
problem in $G$.   
\end{enumerate}

What distinguishes the  criteria
generated  by our framework
is how the weights of the edges are set and what 
method is employed to compute the cut on graph $G$.
Here, we discuss two ways to set the weights:
the first one yields to criteria that
are related with the Gini Gain, while the second 
is built upon some given splitting  criterion that works well for binary attributes.


\subsection{The Squared Gini Criterion}
Here, we discuss how to set the weights so that we obtain 
a criterion that can be seen as a variation of the Gini Gain discussed in Section \ref{def:Gini}.

In fact, Lemma \ref{lem:GiniSq} below  show that  it is possible to define the
weights of the edges so that 
\begin{equation}
 \label{lem:squaredgini}
w(S_L)= Gini(S) - p^2_L \cdot Gini(S_L) - p^2_R \cdot Gini(S_R) 
\end{equation}
for every partition $(L,R)$ of $V$. 

Note that the weight of the cut  $S_L$ in the above identity 
is similar to the expression for the Gini Gain given by equation (\ref{eq:Ginigain}).
The difference is that $p_L$ and $p_R$ are replaced with
$p_L^2$ and $p_R^2$, respectively. Because of the squares, this new criterion tends to favor more balanced partitions. Another practical observation is that one can define the edges without the $2/N^2$ term in \ref{lem:squaredgini} and find the same maximum cut, since this constant appears in all edges weights.

For the proof of Lemma \ref{lem:GiniSq}, recall that $A_{ix}$ is the number
of samples of  class $x$ that have value $v_i$, and that $C$ is used to denote the set of classes.


\begin{lemma}
For every $i,j$, with $i \ne j$ and $i,j \in \{1,\ldots,n\}$,  let
\begin{equation}
 \label{eq:squaredgini}
w_{ij} = \frac{ 2 \sum_{ x,y \in C \atop x \ne y }  A_{ix} A_{jy} }{ N^2} 
\end{equation}
Then, for every partition $(L,R)$ of  $V$ we have
$$w(S_L)=Gini(S) - p^2_L \cdot Gini(S_L) - p^2_R \cdot Gini(S_R)$$
\label{lem:GiniSq}
\end{lemma}

\begin{proof}
Let $S_{x,L}$ and $S_{y,R}$  be the number of samples of classes $x$ and $y$ in groups $L$
and $R$, respectively. Moreover, let  $N_L$ and $N_R$ be 
the number of samples in $L$ and $R$, respectively.
It follows from equation (\ref{eq:gini}) that
$$N^2 Gini(S)=  N^2 -  \sum_{x=1}^k (S_{x,L} + S_{x,R})^2 $$
$$N_L^2 Gini(S_L)=  N_L^2 - \sum_{x=1}^k S_{x,L}^2 $$
and
$$N_R^2 Gini(S_R)= N_R^2 - \sum_{x=1}^k S_{x,L}^2. $$ 
Since $N=(N_L+N_R)$ it follows that 
$$N^2 Gini(S) - N_L^2 Gini(S_L)  - N_R^2 Gini(S_R) =$$
$$ 2 N_L N_R - 2 \sum_{x \in C} S_{x,L} S_{x,R} = $$
$$ 2  \sum_{x \in C } S_{x,L} \sum_{x \in C} S_{x,R}  - 2 \sum_{x \in C} S_{x,L} S_{x,R} =$$  
$$ 2  \sum_{ x \neq y \atop x,y \in C } S_{x,L} S_{y,R} =  2 \sum_{ x \neq y \atop x,y \in C } \left ( \sum_{ i \in L   }\sum_{ j \in R   }  A_{ix}A_{jy} \right ) =$$
$$ N^2 \sum_{i \in L  } \sum_{j \in R  } w_{ij} =  N^2 w(S_L) $$
Dividing the first term and the last term by $N^2$ in the  above expression and, using
 $N_L=p_L \cdot N$ and $N_R=p_R \cdot N$,
we  establish 
the lemma.
\end{proof}


It is worth mentioning that symmetric 
misclassification costs can be easily introduced in this case.
In fact, let $mix(x,y)$  be the cost 	
of  mixing  samples from classes $x$ and $y$.
We can define 
$$ w_{ij} =   \sum_{ x,y \in C \atop x \ne y } mix(x,y)  p_{ix} p_{jy} .$$
This measure favors the separation of the classes
that incur  a large cost in the case they are mixed.

Another natural question is whether the maximum weighted cut problem, which is NP-complete in general, is not easier for the case where the weights are set by equation \ref{eq:squaredgini}. The answer is no, as proved in the theorem below.

\begin{theorem}
Finding the maximum weighted cut in a complete graph whose weights are set by equation \ref{eq:squaredgini} is NP-complete.
\end{theorem}

\begin{proof}
 The idea is to show a reduction from the NP-complete problem PARTITION using the fact that, for some specific instances of our problem, the optimal partition is the most balanced one.
 
 Recall that, in the PARTITION problem, we are given a multiset of integers and want to decide whether it can be partitioned into two multisets with the same sum of elements. Consider an instance given by a multiset $U = \{u_1, \ldots, u_k\}$ of integers. Create an instance of our decision tree problem as follows: for each $u_i \in U$ add a value $v_i$ whose row in the contingency table is given by $u_i e_i$. In other words, in this instance every value has a single class and every class appears for a single value.
 
 W.l.o.g. we can ignore the $2/N^2$ factors in \ref{eq:squaredgini}. Hence every edge $w_{ij}$ in the associated Squared Gini graph will have weight equal to $ij$. Therefore any cut partitioning $V$ into $(L, R)$ will have value equal to
 $$cut(L, R) = \Big(\sum_{i|v_i \in L} u_i\Big) \cdot \Big(\sum_{i|v_i \in R} u_i\Big)$$
 
Note that this formula is maximized when the two terms are as close as possible, i.e. the two partition sides are as balanced as possible. Thus, if we can find a polynomial time algorithm that finds the best partition of this instance's values, we can solve the associated PARTITION problem in polynomial time. This concludes the reduction.
\end{proof}

\subsection{Setting weights according to other splitting criteria}

Our second way of defining the weights makes use of some 
given splitting criterion 
for  binary nominal attributes. 
Such  criterion is used to measure the quality of separating samples with value $v_i$ from those with value $v_j$,
for each $i$ and $j$, and, thus, defining the edges' weights.
Here, we investigate the criterion obtained by 
defining $w_{ij}$ as the value of the $\chi^2$-test
for the attribute $A$ when  evaluated over the restricted dataset 
that contains only the samples of $S$ with values $v_i$ and $v_j$:
$$w_{ij}=  \sum_{\ell=1}^k \frac{(A_{i \ell}-E[A_{i \ell}] )^2}{E[A_{ i \ell}]}
+ \sum_{\ell=1}^k \frac{(A_{j \ell}-E[A_{j \ell}] )^2}{E[A_{ j \ell}]}
$$
where $E[A_{i \ell }]=N_i p_{\ell}$
and $E[A_{j \ell }]=N_j p_{\ell} $.

In order to reduce the bias towards attributes with many values, we divide 
$w_{ij}$ by  $n-1$, for every pair $(i,j)$. We make this adjustment  because each value contributes
to the weight of $n-1$ edges.

We shall remark that, although not explored in this work,
other criteria such as Information Gain or Gini Index
could be used, instead of the $\chi^2$-test, to
set the weights.


\subsection{Example}

Let's assume the are using the Squared Gini criterion on a dataset with a single attribute (the marital status) and we try to predict whether it's a male or female. The contingency table and associated Squared Gini graph are given in figure \ref{fig:cut-example}.

Since this example is small (there are only $7$ non-empty distinct cuts), we can run through all the possible cuts and see that the one shown in red is actually the maximum one. In larger examples we need to use approximated algorithms such as the one mentioned before, since the number of possible cuts grows exponentially with the number of nodes/values.

\begin{figure}[h]
\centering
\includegraphics[width=0.75\textwidth]{cut-example2}
\caption{Contingency table, associated Squared Gini graph and maximum cut.}
\label{fig:cut-example}
\end{figure}

\subsection{Handling Numeric Attributes}
We observe  that criteria from our framework can handle
 a numeric attribute $A$ with $t$ distinct values
$v_1,\ldots,v_t$
by considering it as collection of 
$t-1$ binary attributes, where the
$j$-th attribute, $A^j$,  splits the samples into the
groups $\{s | A(s) \le v_j \}$ and $\{s | A(s) > v_j \}$. 
The split  obtained by criterion $I$ on a numeric attribute
$A$ matches the split of  the best attribute 
among $A^1,\ldots,A^ {t-1}$, according to $I$.

\newpage

\chapter{Experiments on Splits}
\label{chap:experimentssplits}

In this chapter we compare the ability of different heuristics in finding the values' split with lowest impurity.

We report a number of experiments with the heuristics proposed/analyzed
in the previous sections.
Our experiments are very similar to those proposed in \cite{journals/datamine/CoppersmithHH99}
except for a few details.
All experiments are Monte Carlo simulations with 10,000 runs, each using a randomly-generated contingency table for the given number of values $n$ and classes $k$. 
By a contingency table we mean a matrix where each row  corresponds to a distinct vector of
the input $V$. Each table  was created by uniformly picking a number in $\{0, \ldots, 7\}$ for each entry. This guarantees a substantial probability of a row/column having some zero frequencies, which is common in practice. Differing from  \cite{journals/datamine/CoppersmithHH99}, if all the entries corresponding to a value or a class are zero, we re-generate the contingency table, otherwise the number of actual values and classes would not match $n$ and $k$.

We evaluated  Twoing, \Alg  and PCext. The latter is a method
proposed in  \cite{journals/datamine/CoppersmithHH99} that defines the partition of $V$ by using  a hyperplane
in $\R^k$ whose normal  direction is the principal direction of the 
contingency table associated with the instance.
According to the experiments reported in \cite{journals/datamine/CoppersmithHH99}  PCext
consistently  outperformed SliqExt \cite{} and the Flip Flop method \cite{}
in terms of speed and the impurity of the partitions found.

Table \ref{tab:Wins-Gini} and \ref{tab:wins-entropy} show, for different values of $n$ and $k$, the percentage of times that
Twoing outperformed/ was outperformed by PCext for Gini and Entropy, respectively.
Note that the percentages do not necessarily sum up $100\%$ because
there were ties.
We only show results for $k \leq 9$ because for larger values
of $k$ Twoing becomes non practical due to its running time. In addition,
we do not present results for small values of $n$ because in this
case the optimal  partition can be found reasonably quick by testing all possible partitions
so that there is no motivation for heuristics.

\begin{table}[ht]
\caption{Percentage of Wins for PCExt and Twoing for Gini impurity. At each entry the top and the bottom values
corresponds to the number of wins for Twoing and PCExt, respectively}
\begin{center}
\begin{tabular}{c|c|c|c|c}
    
		n/k & 3 & 5 & 7 & 9 \\ \hline
    \multirow{2}{*}{12 }& 8.7 & 12.0  &  13.4 & 15.0  \\ &2.4 & 0.7 & 0.1 & 0 \\ \hline
		\multirow{2}{*}{25 }&  24 & 33.6 & 38.4 & 45.1 \\ & 21.4 & 27.9 & 21.1  & 10.9 \\ \hline
		\multirow{2}{*}{50 } & 38.3  & 41.6  &  41.8 & 46.8  \\ & 37.8 & 52.4 & 54.6  & 49.9 \\
		
				%\multirow{2}{*}{12 }&X & 6\\ &Y & 2\\ & Z & 4 \\
		
		%\multirow{2}{*}{70 }&X & 3\\ &Y & 2\\ 
		
    
\end{tabular}
\end{center}
\label{tab:Wins-Gini}
\end{table}

\begin{table}[ht]
\caption{Percentage of Wins for PCExt and Twoing for Gini Entropy. At each entry the top and the bottom values
corresponds to the number of wins for Twoing and PCExt, respectively}
\begin{center}
\begin{tabular}{c|c|c|c|c}
    
		 n/k & 3 & 5 & 7 & 9 \\ \hline
    \multirow{2}{*}{12 }& 19.7 & 25.8  &  26.8 & 27.7  \\ &1.3 & 0.4 & 0.0 & 0.0 \\ \hline
		\multirow{2}{*}{25 }&  43.4 & 54.4 & 58.4 & 60.9 \\ & 12.4 & 15.8 & 12.1  & 7.0 \\ \hline
		\multirow{2}{*}{50 } & 64.6  & 70.2  &  70.0 & 69.8  \\ & 18.9 & 25.7 & 27.5  & 28.0 \\
		
				
    
\end{tabular}
\end{center}
\label{tab:wins-entropy}
\end{table}


In general, we observe an advantage of Twoing  
for both criteria, being more clear for Entropy impurity.
The maximum relative excess  between the impurities 
of the partitions found by  Twoing and PCext was $0.9\%$ for
Gini and $1.4 \%$ for Entropy.
On the other hand, the 
 maximum relative excess  between the impurities 
of the partitions found by  PCext and Twoing was $3.48 \%$ for
Gini and $6.23 \%$ for Entropy. 
In terms of speed, as expected, Twoing was faster up to $k=7$ and then 
PCext becomes faster.

% Although Twoing  only outperforms
% PCExt for limited number of classes we believe that this is relevant
%because one can find a number of classification tasks (e.g. UCI) 
%where the number of classes is small but larger than 2.


We do not report the results for \Alg because it was not competitive with the other
two heuristics. However, due to its running time it might be used when both $n$ and $k$ are very large and
speed is an  issue. When $n=200$ and $k=100$ $\Alg$, using one care, is almost 50 times
faster than PCExt, with the latter using  8 cores. In addition,
\Alg could be  used together with  PCExt, incurring a negligible overhead, to guarantee that the ratio 
between the impurity of the partition found and the optimal one is bounded.

 
Taking into account these  experiments, those reported in \cite{journals/datamine/CoppersmithHH99}
and the  theoretical properties of the available algorithms, 
Table  \ref{tab:guidelines}  suggests  guidelines on how to solve the 
problem of finding the binary partition of minimum impurity  in practical situation.
Of course small, medium and large depend on the available
hardware and the time one  is up to wait to train/test classification models. 


\begin{table}[htb]
\centering
\caption{Guidelines on how to solve the problem of finding the partition with minimum impurity 
in Practice.}
\label{tab:guidelines}
\begin{tabular}{c|c|c}
{\bf n} & {\bf k} & {\bf Suggested Method} \\ \hline 
small &  &  Exact \\
not small  & small & Twoing \\
not small  & not small & PC-Ext \\
\end{tabular}
\end{table}
%We are not aiming to understand which is the best availble method 




 

--------------------------------------------------------------------------


All experiments in this section are Monte Carlo simulations with 10,000 runs, each using a randomly-generated contingency table for the given number of values $n$ and classes $k$. The contingency tables were created by uniformly picking a number in $\{0, \ldots, 7\}$ for each entry, as done in Copersmith et al (TODO: ref). This guarantees a substantial probability of a row/column having some zero frequencies, which is common in practice. If all the entries corresponding to a value or a class are zero, we re-generate the contingency table, otherwise the number of actual values and classes would not match $n$ and $k$.

We ran the experiments for two different impurity measures: the Gini Gain and the Information Gain. We measured the performance of the same criteria for both of them. All of the non-exact criteria studied are based on the superclass trick, using the theorem proved by Breiman et al (TODO: ref). To separate the classes into a pair of superclasses we use two heuristics: \Alg (where the class with the largest frequency is separated from the others) and ListScheduling (balancing the superclasses using a polynomial algorithm with a $4/3$-approximation, as explained in TODO: ref). Since \Alg (ListScheduling) should be the best heuristic for the Gini (InformationGain) impurity, we compare against ListScheduling (\Alg) as a baseline. Given a partition of classes into superclasses, we use the theorem (TODO:ref) to find the optimal partition of values in linear time after sorting.

We compare those methods with the exact and the Twoing criteria, which run in exponential time in $n$ and $k$, respectively. For all the non-exact criteria, we measured the impurity with respect to all the $k$ classes--instead of the impurity with respect to the superclasses--since that's what we are optimizing. We also studied a criterion that picks a random partition of values into two non-empty groups, but it was significantly worse than all the other methods, thus we chose not to report its results. 

We are interested in choosing what criterion to use when the exact ones don't run in reasonable time. In some experiments we choose values for $n$ that are not very large because we want to be able to compare the other criteria with the exact ones. In others, we choose $n$ to be very large, and don't analize the exact criteria because they don't finish executing.


%\subsection{Small Number of Classes ($k=3$)}
The first set of experiments uses $k=3$ classes and $n=6,~12$ values. Since the Twoing criterion behaves almost perfectly, both in terms of execution time and impurity found, we don't compare it against the \Alg and ListScheduling criteria.

From tables \ref{imp-all-gini}, \ref{imp-all-infogain}, \ref{match-all-gini} and \ref{match-all-infogain}, it is striking how well Twoing performs. In almost all the simulations the impurity found is exactly the same as the exact methods. This suggests that, in cases when the exact methods don't run in reasonable time but Twoing does (that is, large $n$ and small $k$), we lose almost nothing by using it.


%\subsection{Medium Number of Classes ($k=9$)}

The second set of experiments uses $k=9$ classes and $n=6,~12$ values. The results are shown in tables \ref{imp-all-gini}, \ref{imp-all-infogain}, \ref{match-all-gini} and \ref{match-all-infogain}. The comparison between \Alg and ListScheduling is shown in tables \ref{heuristics-gini} and \ref{heuristics-infogain}.


Once again Twoing behaves remarkably well when it comes to getting the lowest impurity, but its execution time is already much larger than the criteria that use a single superclass split.


Comparing the heuristics between themselves we can see that, in terms of the Gini impurity, the \Alg heuristic is the best among them, while for the Entropy impurity, the best one is the ListScheduling method. This is consistent with the theory developed earlier in this paper.


%\subsection{Very Large Number of Values ($n\ge30$)}

The last set of experiments is focused on studying what happens when the number of values is very large ($n\ge30$) and the number of classes increases. 

First we study what happens when the number of classes is not very large ($k=3, ~9$). In this comparison we use Twoing as a baseline, since its behavior is so close to the exact methods in the other experiments and since it is still feasible to run it, while the exact methods aren't. Note that, since Twoing evaluates all the possible superclass partitions, the impurity it finds is never worse than that of \Alg and ListSchedule. The results are shown in tables \ref{imp-large1-gini}, \ref{imp-large1-infogain}, \ref{match-large1-gini}, \ref{match-large1-infogain}, \ref{heuristics-gini} and \ref{heuristics-infogain}.



Note that the results for the \Alg and the ListScheduling methods are the same when $k=3$ because they give the same superclass partitions. We can also notice that the number of times that Twoing finds a partition with impurity smaller than the other methods increases with $k$. This is explained by the exponential growth of the number of possible superclass partitions with $k$. What's surprising is that the \Alg method gives slightly better partitions than ListSchedule for the Information Gain impurity. Since both methods are within the approximation bounds set in this paper, this does not contradict our results.



In the next experiments, we use large number of values and classes. This makes both Twoing and the exact methods infeasible to be executed, so we only compare \Alg with ListSchedule. We use the minimum impurity between the two as a baseline.


\begin{table}[]
\centering
\caption{Percentage of Gini impurity above the one found by the best heuristic.}
\label{imp-large2-gini}
\begin{tabular}{l|l|l|l}
n   & k   & \Alg          & ListSchedule \\
\hline
30  & 50  & 2.82e-3\%        & 2.86e-3\% \\
50  & 30  & 3.91e-3\%        & 6.04e-3\% \\
50  & 50  & 1.72e-3\%        & 2.23e-3\% \\
50  & 100 & 0.56e-3\%        & 0.65e-3\% \\
100 & 50  & 0.72e-3\%        & 1.90e-3\% \\
100 & 100 & 0.26e-3\%        & 0.49e-3\% \\
100 & 200 & 0.09e-3\%        & 0.13e-3\% \\
200 & 100 & 0.08e-3\%        & 0.46e-3\%
\end{tabular}
\end{table}

\begin{table}[]
\centering
\caption{Percentage of Information Gain impurity above the one found by the best heuristic.}
\label{imp-large2-infogain}
\begin{tabular}{l|l|l|l}
n   & k   & \Alg & ListSchedule \\
\hline
30  & 50  & 0.02\% & 0.02\% \\
50  & 30  & 0.02\% & 0.02\% \\
50  & 50  & 0.01\% & 0.01\% \\
50  & 100 & 5.89e-3\% & 5.96e-3\% \\
100 & 50  & 4.97e-3\% & 0.01\% \\
100 & 100 & 3.02e-3\% & 5.05e-3\% \\
100 & 200 & 1.55e-3\% & 2.17e-3\% \\
200 & 100 & 0.75e-3\% & 5.66e-3\%
\end{tabular}
\end{table}

\begin{table}[]
\centering
\caption{Percentage of simulations where the Gini impurity found is worse than the one found by the best heuristic.}
\label{match-large2-gini}
\begin{tabular}{l|l|l|l}
n   & k   & \Alg & ListSchedule \\
\hline
30  & 50  & 49.71\%        & 50.22\%        \\
50  & 30  & 42.98\%        & 57.02\%        \\
50  & 50  & 46.19\%        & 53.81\%        \\
50  & 100 & 47.39\%        & 52.57\%        \\
100 & 50  & 34.74\%        & 65.26\%        \\
100 & 100 & 39.72\%        & 60.28\%        \\
100 & 200 & 43.27\%        & 56.73\%        \\
200 & 100 & 24.22\%        & 75.78\%
\end{tabular}
\end{table}



Comparing the results in tables \ref{imp-large2-gini}, \ref{imp-large2-infogain}, \ref{match-large2-gini}, \ref{match-large2-infogain}, \ref{heuristics-gini} and \ref{heuristics-infogain} we can see that, for both the Gini and Information Gain impurity measures, the \Alg is usually the best method. This is specially interesting for the Information Gain impurity, since the approximation guarantees for the method that uses a balanced superclass partition is better. Furthermore, the results for the \Alg criterion improve as the number of values and classes increase. This suggests that, in practice, one should prefer the \Alg method when $n$ and $k$ are very large, even though the approximation guarantee for the ListSchedule method is tighter.

%\subsection{Experiments Conclusions}
The above experiments strongly suggest that, when we can't run the exact criteria, the best choice we can make is to choose the Twoing criterion, if its execution time is feasible. If that's also not possible, for the Gini impurity, the best choice is always to use the \Alg criterion. For the Information Gain, we should choose between ListSchedule, if $n$ and $k$ are not significantly large, or \Alg, for larger number of values and classes.

In all the simulations the criteria satistied the approximation guarantees given in this paper. Since they all performed much better than their approximation bounds, the criterion with the best guarantee is not necessarily the best in terms of the expected impurity of the chosen partition.


\newpage

\chapter{Experiments on Real Datasets}
\label{chap:experiments-datasets}


In this chapter we describe our experimental study on real datasets. First, we describe the chosen datasets. Next, we discuss the max-cut algorithms 
employed and, then, we present our results. We will compare the performance between Twoing, Hypercube Cover, PC-ext and some criteria generated from our framework. Both Hypercube Cover and PC-ext were chosen based on the experiments results of the previous chapter. We will use the Gini impurity with them throughout this chapter.

All  experiments described in the following sections were executed on a machine with the following settings: Intel(R) Core(TM) i7-4790 CPU @ 3.60GHz with 32 GB of RAM. The code was developed using Python 3.6.1 with the libraries numpy, scipy, scikit-learn and cvxpy.
The project can be accessed in {\tt github.com/felipeamp/dissertation-code}.

TODO: tornar repositorio publico e adicionar os resultados agregados.

\section{Datasets}
We employed 11 datasets in total. Eight of them are from the UCI repository:
Mushroom, KDD98, Adult, Nursery, Covertype, Cars, Contraceptive and Poker (\cite{Lichman:2013}).
Two others are available in Kaggle: San Francisco Crime and Shelter Animal Outcome
\cite{SFC,AnimalShelter}. The last dataset was created by translating texts from the Reuters database \cite{Lichman:2013} into phonemes, using the CMU pronouncing dictionary (see \cite{CMU-PD}).

We chose  these datasets  because they
have at least 1000 samples and they  either contain  multi-valued attributes 
or attributes that can be naturally aggregated to produce multi-valued attributes. 
From the KDD98 dataset we derived the datasets
KDD98-k, for $k = 2$ and $9$. These datasets contain
only the positive samples (people that donate money) 
of KDD98 and the target attribute, Target$\_$D, is split into $k$ classes, where the $i$-th
class correspond to the $i$-th quantile in terms of amount of money donated. For the Reuters Phonemes dataset,
we extracted 10000 samples containing the 15 most common phonemes as class and try to predict when they are about to happen given the 3 preceding phonemes.
This dataset is motivated by Spoken Language Recognition problems, where phonotactic models are used as an important part of the classification system, as seen in \cite{conf/interspeech/Navratil06}. 
For the San Francisco Crime dataset, we try to predict the crime category given the month, day of the week, police department district and latitute/longitude. Lastly, for the Shelter Animal Outcomes dataset, we converted the age into a numeric field containing the
number of days old and separated the breed into two categorical fields, repeating the breed in both in case there was only one originally. We also removed the AnimalID, Name and the DateTime. For this dataset we try to predict the outcome type and subtype (concatenated into a single categorical field). For both 
San Francisco Crime and  Shelter Animal Outcomes
 datasets we created a version of them ({\tt S.F. Crime-15} and {\tt Shelter-15}), containing only 15 classes, instead of the 39 and 22 original ones, respectively. This was done by grouping the rarest classes into a single one. 

\begin{table}
\centering
\begin{tabular}{c|c|c|c|c|c}
Dataset             & Samples  &  k        & $m_{nom}$ &  $m^{ext}_{nom}$ &   $m_{num}$   \\  \hline
{\tt Mushroom}      & 5644     & 2         & 22        & N/A              & 0             \\ 
{\tt Adult}         & 30162    & 2         & 8         & N/A              & 6             \\
{\tt KDD98}         & 4843     & {\tt Reg} & 65        & N/A              & 314           \\ 
{\tt Nursery}       & 12960    & 5         & 8         & 11               & 0             \\ 
{\tt CoverType}     & 581012   & 7         & 44        & 46               & 10            \\ 
{\tt Car}           & 1728     & 4         & 6         & 8                & 0             \\ 
{\tt Contracep}     & 1473     & 3         & 7         & 9                & 2             \\ 
{\tt Poker}         & 25010    & 10        & 10        & 0                & 0             \\
%{\tt Shelter-15}    & 26711    & 15        & 5         & N/A              & 1             \\
{\tt Shelter}       & 26711    & 22        & 5         & N/A              & 1             \\      
%{\tt S.F. Crime-15} & 878049   & 15        & 3         & N/A              & 2             \\      
{\tt S.F. Crime}    & 878049   & 39        & 3         & N/A              & 2             \\  
{\tt Phonemes}      & 10000    & 15        & 3         & N/A              & 0 
\normalsize
\end{tabular}
\caption{Information about  the employed datasets after data cleaning and attributes aggregation.
Column $k$ is the number of classes and {\tt Reg} stands
for Regression; columns $m_{nom}$ and $m^{ext}_{nom}$ are the
number of  nominal attributes in the original and the
extended datasets (when it exists), respectively; column $m_{num}$ is the number of  numeric attributes.}
\label{exp:datasets}

\end{table}



We also created extended versions
of some of the above datasets 
by adding nominal attributes  obtained by aggregating some of the original ones, as 
we detail below.  
Our goals are  examining the impact of multi-valued
attributes in the classification performance and 
also understanding how the  different 
splitting criteria  handle them.


\begin{table}[]
\centering
\begin{tabular}{c|c|c} 
{\tt parents}     & {\tt has\_nurs}    & {\tt Aggregated Attribute}   \\ \hline
usual       & proper       & usual-proper             \\
usual       & less\_proper & usual-less\_proper       \\
usual       & improper     & usual-improper           \\
usual       & critical     & usual-critical           \\
usual       & very\_crit   & usual-very\_crit         \\
pretentious & proper       & pretentious-proper       \\
pretentious & less\_proper & pretentious-less\_proper \\
pretentious & improper     & pretentious-improper     \\
pretentious & critical     & pretentious-critical     \\
pretentious & very\_crit   & pretentious-very\_crit   \\
great\_pret & proper       & great\_pret-proper       \\
great\_pret & less\_proper & great\_pret-less\_proper \\
great\_pret & improper     & great\_pret-improper     \\
great\_pret & critical     & great\_pret-critical     \\
great\_pret & very\_crit   & great\_pret-very\_crit  
\end{tabular}
\caption{Aggregation of attributes parents and has\_nurse from dataset {\tt Nursery}.}
\label{tab:Aggreation}

\end{table}

Table \ref{tab:Aggreation} illustrates this construction.

\begin{itemize}

\item {\tt Nursery-Ext}. This dataset is obtained by adding three
new attributes to dataset Nursery. 
The first attribute has  $15$ distinct values
and it is constructed through the  aggregation of   2 attributes 
from group  EMPLOY, one with 5 values and the other with 3 values. 
The second attribute has 72 distinct values
corresponding to the aggregation of attributes from 
the attributes in group {\tt STRUCT\_FINAN}.
The third attribute, with 9 distinct values, is the combination of
the  attributes in group {\tt SOC\_HEALTH}.


\item {\tt Covertype-Ext}. 
We combined 40 binary attributes
related with the soil type  into a new attribute with 40 distinct values.
The same approach was employed to combine the 4 binary attributes related
with the wilderness area into a new attribute
with 4 distinct values.
This is an interesting case because, apparently, the 40 (soil type)
binary attributes  as well as the 4 (wilderness area) binary attributes
 were derived from a binarization of two attributes, one with 40 distinct value and the other with 4 distinct values.
 

\item {\tt Cars-Ext}. To obtain this dataset, the 2 attributes
related with the concept {\tt PRICE},  {\tt buying} and {\tt maint},
were combined into an attribute
with 16 distinct values.
Moreover, the 3 attributes
related with concept {\tt CONFORT} were  combined into an 
an attribute with 36 distinct values.

\item {\tt Contraceptive-Ext}. The 2 attributes
related with the couple's education were combined into an
attribute with 16 distinct values.
Moreover, the 3 attributes related with the couple's occupations  and  standard of living
were  aggregated into a new attribute with 32 distinct values.

\end{itemize}

Samples with missing values were removed from the datasets.
Table \ref{exp:datasets} provides some statistics.




\section{Computing the Maximum Cut}
 

The GW algorithm  requires  the solution of a semidefinite program (SDP),
which may be computationally expensive despite  its 
polynomial time worst case  behavior.
As an example, for an attribute with 100 distinct values, the solution of
the corresponding SDP takes in average about 2 second in our machine.
One the one hand, this is a tiny amount of time 
compared with that required to perform an exhaustive search on the $2^{100}$
possible binary partitions. On the other hand, 
 faster alternatives are desirable, even
at the cost of losing part of the theoretical approximation guarantee. 

To avoid solving a SDP, 
we also evaluated a procedure
that first executes the {\tt GreedyCut} algorithm presented
in Section \ref{sec:maxcutbackground} and then runs a local search as described
in the Algorithm  \ref{alg:localsearch} in the same section. 
The use of this approach combined with the two
ways of setting the edges' weights lead to
 Greedy LocalSearch SquaredGini (GLSG) and 
Greedy LocalSearch $\chi^2$ (GL$\chi^2$) 
criteria, respectively.
For attributes with 100 distinct values this approach is 60-70
times faster than the one based on the GW algorithm.

Even though solving the SDP takes polynomial time, the GW criteria evaluated in this chapter were too slow compared with the others. Therefore their results won't be shown in the following Sections except for the Section for maximum depth 5 (\ref{subsec:depth-5}), which were the first experiments to be run.


\section{Experimental Results}

We performed a number of experiments to evaluate how the
proposed methods behave with real datasets.
All experiments consist of building decision trees
with a predefined maximum depth.  In addition,
to prevent the selection of non-informative nominal attributes, 
we used a $\chi^2$-test for each attribute at every node of
the tree: if the $\chi^2$-test  on the contingency table of attribute $A$
has $p$-value larger than $10\%$ at a node $\nu$, then
$A$ is not used in $\nu$. Furthermore, attributes with less than 15 samples associated with its
second most frequent value are also not considered for splitting. This helps  avoid data overfitting.


\subsection{Maximum Depth 1}

Table \ref{tab:nominal-1} presents  the results of an experiment to
compare the accuracy of  Decision Trees built by  our methods with those built by Twoing, Hypercube Cover and PC-ext.
In this experiment, the decision tree can have a single level and we considered just the nominal attributes of the datasets. 
The motivation for this depth is that some random
forests methods such as boosting work with very shallow trees.
Each accuracy is the average of 20 stratified 3-fold cross-validations,
each generated with a different seed.
The entry  associated with  $({\cal D},I)$ has two pieces of information: the average accuracy
of criterion $I$ on dataset ${\cal D}$ and the number of criteria
with accuracy   statistically lower than that of $I$ on dataset ${\cal D}$. 
The statistical test used for criteria comparison is a  one-tailed paired $t$-student test with a $95\% $ confidence level. 

In general, there was a clear advantage towards PC-ext, followed by Hypercube Cover and GL Squared Gini. Twoing and GL$\chi^2$had, in general, the worse performance. As expected, the criteria created the same tree for many datasets, and this happened more frequently when the number of classes was smaller.

\begin{table}
\scriptsize
\centering
\begin{tabular}{c|cc|cc|cc|cc|cc|cc} 
Dataset             & \multicolumn{2}{c|}{Twoing} &  \multicolumn{2}{c|}{GLSG}  & \multicolumn{2}{c|}{GL$\chi^2$} & \multicolumn{2}{c|}{PC-ext}& \multicolumn{2}{c|}{HcC} & \multicolumn{2}{c}{LCA} \\
\hline
{\tt Adult}         &  75.11       & (0)          &  75.11       & (0)          &  75.11       & (0)              & 75.11      & (0)           & 75.11      & (0)        & 75.11      & (0)        \\
{\tt Mushroom}      &  98.44       & (0)          &  98.44       & (0)          &  98.44       & (0)              & 98.44      & (0)           & 98.44      & (0)        & 98.44      & (0)        \\
{\tt KDD98}-2       &  {\bf 83.61} & {\bf (1)}    &  83.3        & (0)          &  {\bf 83.61} & {\bf (1)}        &{\bf 83.61} & {\bf (1)}     & {\bf 83.61}& {\bf (1)}  & {\bf 83.61}& {\bf (1)}  \\
%{\tt KDD98}-3       &  62.61       &              &  63.33       &              &  {\bf 63.56} &                  &            &               &            &            &             &           \\
%{\tt KDD98}-5       &  46.46       &              &  46.39       &              &  {\bf 47.53} &                  &            &               &            &            &             &           \\
{\tt KDD98}-9       &  31.66       & (0)          &  32.32       & {\bf (5)}    &  31.61       & (0)              & 31.74      & (2)           & 31.74      & (2)        & 31.74       & (2)       \\
{\tt Nursery}       &  66.25       & (0)          &  66.25       & (0)          &  66.25       & (0)              & 66.25      & (0)           & 66.25      & (0)        & 66.25       & (0)       \\
{\tt Nursery-ext}   &  66.25       & (0)          &  66.25       & (0)          &  66.25       & (0)              & 66.25      & (0)           & 66.25      & (0)        & 66.25       & (0)       \\
{\tt Cars}          &  70.02       & (0)          &  70.02       & (0)          &  70.02       & (0)              & 70.02      & (0)           & 70.02      & (0)        & 70.02       & (0)       \\
{\tt Cars-ext}      &  70.02       & (0)          &  70.02       & (0)          &  70.02       & (0)              & 70.02      & (0)           & 70.02      & (0)        & 70.02       & (0)       \\
{\tt Contracep}     &  43.39       & {\bf (2)}    &  42.31       & (0)          &  42.23       & (0)              & 43.63      & {\bf (2)}     & 43.63      & {\bf (2)}  & 43.63       & {\bf (2)} \\
{\tt Contracep-ext} &  42.58       & (1)          &  {\bf 44.29} & {\bf (5)}    &  42.23       & (0)              & 43.73      & (4)           & 43.38      & (2)        & 43.48       & (3)       \\
{\tt CoverType}     &  {\bf 51.93} & {\bf (1)}    &  48.76       & (0)          & {\bf 51.93}  & {\bf (1)}        &{\bf 51.93} & {\bf (1)}     & {\bf 51.93}& {\bf (1)}  & {\bf 51.93} & {\bf (1)} \\
{\tt CoverType-ext} &  51.54       & (0)          &  55.55       & (4)          &  51.93       & (2)              &{\bf 58.64} & {\bf (5)}     & 51.54      & (0)        & 55.53       & (3)       \\
{\tt Poker}         &  49.93       & (0)          &  {\bf 49.95} & (0)          & {\bf 49.95}  & (0)              & 49.93      & (0)           & 49.93      & (0)        & 49.93       & (0)       \\
{\tt Shelter-15}    &  {\bf 48.44} & {\bf (2)}    &  35.39       & (0)          &  40.1        & (1)              &{\bf 48.44} & {\bf (2)}     & {\bf 48.44}& {\bf (2)}  & {\bf 48.44} & {\bf (2)} \\
{\tt S.F. Crime-15} &  {\bf 21.61} & {\bf (2)}    &  21.24       & (1)          & 19.98        & (0)              & 21.58      & {\bf (2)}     & 21.58      & {\bf (2)}  & 21.58       & {\bf (2)} \\
{\tt Phonemes}      &  22.21       & (0)          &  22.22       & (0)          & 22.57        & (2)              &{\bf 23.92} & {\bf (4)}     & 23.91      & {\bf (4)}  & 23.88       & (3)       \\
\hline
Average (Sum)       &  55.81       & (9)          &  55.09       & (15)         & 55.14        & (7)              &  56.45     & (23)          & 55.99      & (16)       & 56.24       & (19)
\end{tabular}
\normalsize
\caption{Average accuracy and statistical tests  for  decision trees with depth at most 1 using only nominal attributes. The best accuracy for each dataset is bold-faced.}
\label{tab:nominal-1}
\end{table}

Table \ref{tab:ctree-1} presents the 
comparison between our GL methods, Twoing, Hypercube Cover and PC-ext in another scenario,
where we use $c_{quad}$, one of the bias-free criterion proposed in \cite{Hothorn:2006:URP}, to select the attribute at each node of the tree. 
Then, both Twoing, Hypercube Cover, PC-ext and our methods are  used only for splitting the chosen attribute, which allows for a  more direct comparison of their splitting ability. Since the maximum depth is set to 1, all the criteria are splitting the same attribute. This makes the analysis of their splitting ability more direct than with larger depths. This experiment is very similar to the one from the previous chapter, except now we are using real datasets.

As can be seen, they all obtained similar results. In terms of beating other criteria, Twoing obtained the best results, followed by Hypercube Cover. Ignoring the extended datasets, since no extended attribute was chosen by the conditional inference tree and thus its results are a repetition of the datasets without extended attributes, we note that PC-ext and the GL criteria obtained similar results. This suggests that Twoing and Hypercube Cover are better at choosing the best values partition, once the attribute is fixed. Nonetheless, since GLSG had significantly better results in the Mushroom dataset, it had the best average accuracy.


\begin{table*}
\scriptsize
\centering
\begin{tabular}{c|cc|cc|cc|cc|cc|cc} 
Dataset  &   \multicolumn{2}{c|}{CI-Twoing} &   \multicolumn{2}{c|}{CI-GLSG} & \multicolumn{2}{c|}{CI-GL$\chi^2$}& \multicolumn{2}{c|}{CI-PC-ext}& \multicolumn{2}{c|}{CI-HcC}& \multicolumn{2}{c}{CI-LCA} \\  \hline   
% Dataset          &        CI-Twoing       &        CI-GLSG          &      CI-GL$\chi^2$      &       CI-PC-ext         &       CI-HcC            &       CI-LCA
{\tt Adult}        &{\bf 75.13} & {\bf (2)} & 75.11       & (0)       & 75.11       & (0)       & {\bf 75.13} & {\bf (2)} & {\bf 75.13} & {\bf (2)} & {\bf 75.13} & {\bf (2)} \\
{\tt Mushroom}     &63.88       & (1)       &{\bf 68.03 } & {\bf (5)} & 66.5        & (4)       & 63.72       & (0)       & 63.88       & (1)       & 63.88       & (1)       \\
{\tt KDD98}-2      &65.29       & (0)       & 65.29       & (0)       & 65.29       & (0)       & 65.29       & (0)       & 65.29       & (0)       & 65.29       & (0)       \\
%{\tt KDD98}-3      &65.28       &           & {\bf 65.8  }&           & 65.07       &           &             &           &             &           &             &           \\
%{\tt KDD98}-5      &49.23       &           & {\bf 49.58} &           & 49.27       &           &             &           &             &           &             &           \\
{\tt KDD98}-9      &25.48       & (0)       & 25.48       & (0)       & 25.48       & (0)       & 25.48       & (0)       & 25.48       & (0)       & 25.48       & (0)       \\
{\tt Nursery}      &{\bf 41.9}  & {\bf (2)} & 41.05       & (0)       & {\bf 41.9}  & {\bf (2)} & {\bf 41.9}  & {\bf (2)} & {\bf 41.9}  & {\bf (2)} & 41.45       & (1)       \\
{\tt Nursery-Ext}  &{\bf 41.9}  & {\bf (2)} & 41.05       & (0)       & {\bf 41.9}  & {\bf (2)} & {\bf 41.9}  & {\bf (2)} & {\bf 41.9}  & {\bf (2)} & 41.45       & (1)       \\
{\tt Cars}         &70.02       & (0)       & 70.02       & (0)       & 70.02       & (0)       & 70.02       & (0)       & 70.02       & (0)       & 70.02       & (0)       \\
{\tt Cars-Ext}     &70.02       & (0)       & 70.02       & (0)       & 70.02       & (0)       & 70.02       & (0)       & 70.02       & (0)       & 70.02       & (0)       \\
{\tt Contracep}    &43.39       & {\bf (2)} & 42.66       & (1)       & 42.23       & (0)       & {\bf 43.63} & {\bf (2)} & {\bf 43.63} & {\bf (2)} & {\bf 43.63} & {\bf (2)} \\
{\tt Contracep-Ext}&43.39       & {\bf (2)} & 42.66       & (1)       & 42.23       & (0)       & {\bf 43.63} & {\bf (2)} & {\bf 43.63} & {\bf (2)} & {\bf 43.63} & {\bf (2)} \\
{\tt CoverType}    &48.76       & (0)       & 48.76       & (0)       & 48.76       & (0)       & 48.76       & (0)       & 48.76       & (0)       & 48.76       & (0)       \\
{\tt CoverType-Ext}&48.76       & (0)       & 48.76       & (0)       & 48.76       & (0)       & 48.76       & (0)       & 48.76       & (0)       & 48.76       & (0)       \\
{\tt Poker}        &49.93       & (0)       & 49.95       & (0)       & 49.95       & (0)       & 49.93       & (0)       & 49.93       & (0)       & 49.93       & (0)       \\  
{\tt Shelter-15}   &35.93       & (0)       & 35.93       & (0)       & 35.93       & (0)       & 35.93       & (0)       & 35.93       & (0)       & 35.93       & (0)       \\   
{\tt S.F. Crime-15}&19.92       & (0)       & 19.92       & (0)       & 19.92       & (0)       & 19.92       & (0)       & 19.92       & (0)       & 19.92       & (0)       \\ 
{\tt Phonemes}     &{\bf 15.7}  & {\bf (5)} & 15.34       & (0)       & 15.42       & (0)       & 15.46       & (2)       & 15.46       & (2)       & 15.36       & (0)       \\ 
\hline
Average (Sum)      & 47.46      & (16)      & 47.5        & (7)       & 47.46       & (8)       & 47.47       & (12)       & 47.48       & (13)      & 47.41      & (9)
\end{tabular}
\caption{Average accuracy and statistical tests  for  Conditional Inference trees with depth at most 1 using only nominal attributes. The best accuracy for each dataset is bold-faced.}
\label{tab:ctree-1}
\end{table*}


Table \ref{exp:numeric-1} shows experiments  similar to those presented at Table \ref{tab:nominal-1}, but now
using also the numeric attributes. This time, in terms of beating the accuracy of other criteria, the GL methods obtained the best results, while Twoing had the worst. This suggests that criteria from our framework are better in terms of comparing numerical and nominal attributes among themselves. Considering only the average accuracy, PC-ext and Hypercube Cover had the best results, mainly due to the accuracies found for both CoverType datasets.

\begin{table}
\scriptsize
\centering
\begin{tabular}{c|cc|cc|cc|cc|cc|cc} 
Dataset            &\multicolumn{2}{c|}{Twoing} & \multicolumn{2}{c|}{GLSG} & \multicolumn{2}{c|}{GL$\chi^2$} & \multicolumn{2}{c|}{PC-ext}& \multicolumn{2}{c|}{HcC}& \multicolumn{2}{c}{LCA}\\
\hline   
{\tt Adult}        & 75.11          &  (0)      & 75.11      & (0)          & {\bf 79.55} &  {\bf (5)}        & 75.11       &  (0)         & 75.11       &  (0)     & 75.11       & (0)       \\
{\tt KDD98}-2      & 83.62          &  (0)      & {\bf 83.93}& {\bf (5)}    & 83.67       &  (0)              & 83.62       &  (0)         & 83.62       &  (0)     & 83.62       & (0)       \\ 
%{\tt KDD98}-3      &                &           & 67.9       &              & 69.1        &                   &             &              &             &          &             &           \\ 
%{\tt KDD98}-5      &                &           & 54.5       &              & {\bf 55.3 } &                   &             &              &             &          &             &           \\ 
{\tt KDD98}-9      & 32.13          &  (3)      &{\bf 32.52} & {\bf (4)}    & 32.46       &  {\bf (4)}        & 31.44       &  (0)         & 31.44       &  (0)     & 31.44       & (0)       \\ 
{\tt Contracep}    & 42.7           &  (1)      & 42.21      & (0)          & {\bf 45.19} &  {\bf (5)}        & 42.7        &  (1)         & 42.7        &  (1)     & 42.7        & (1)       \\  
{\tt Contracep-Ext}& 42.7           &  (0)      & 44.29      & (4)          & {\bf 45.19} &  {\bf (5)}        & 42.68       &  (0)         & 42.59       &  (0)     & 42.59       & (0)       \\ 
{\tt CoverType}    & 63.31          &  (2)      & 62         & (1)          & 52.38       &  (0)              & {\bf 63.36} &  {\bf (3)}   & {\bf 63.36} & {\bf (3)}& {\bf 63.36} & {\bf (3)} \\  
{\tt CoverType-Ext}& 63.31          &  (2)      & 62         & (1)          & 52.38       &  (0)              & {\bf 63.36} &  {\bf (3)}   & {\bf 63.36} & {\bf (3)}& {\bf 63.36} & {\bf (3)} \\  
{\tt Shelter-15}   & {\bf 48.44}    & {\bf (2)} & 35.4       & (0)          & 43.29       &  (1)              & {\bf 48.44} &  {\bf (2)}   & {\bf 48.44} & {\bf (2)}& {\bf 48.44} & {\bf (2)} \\   
{\tt S.F. Crime-15}& {\bf 21.61}    & {\bf (2)} & 21.24      & (1)          & 20.61       &  (0)              & 21.58       &  {\bf (2)}   & 21.58       & {\bf (2)}& 21.58       & {\bf (2)} \\ 
\hline
Average (Sum)      & 52.55          &  (12)     & 50.97      & (16)         & 50.52       &  (20)             & 52.48       &  (11)        & 52.47       &  (11)    & 52.47       & (11)

\end{tabular}
\caption{Average accuracy and statistical test results for  Decision Trees using both nominal and numeric attributes.}
\label{exp:numeric-1}
\normalsize
\end{table}



\subsection{Maximum Depth 5}
\label{subsec:depth-5}

Table \ref{tab:nominal-5} presents  the results of an experiment to
compare the accuracy of  Decision Trees built by  our methods with those built by Twoing, Hypercube Cover and PC-ext.
In this experiment, the maximum depth was set to 5 and we considered just the nominal attributes of the datasets. 
The motivation for this depth is to produce trees that
are relatively easy to interpret and, in addition, some random
forests methods such as boosting work with shallow trees.
Each accuracy is the average of 20 stratified 3-fold cross-validations,
each generated with a different seed.
\small
The entry  associated with  $({\cal D},I)$ has two pieces of information: the average accuracy
of criterion $I$ on dataset ${\cal D}$ and the number of criteria
with accuracy   statistically lower than that of $I$ on dataset ${\cal D}$. 
The statistical test used for criteria comparison is a  one-tailed paired $t$-student test with a $95\% $ confidence level.

In general, there was a clear advantage towards PC-ext and a clear disadvantage towards the GW-based criteria. Twoing, Hypercube Cover and the GL criteria had somewhat similar results. The advantage of the GL-based criteria over the GW-based ones is likely related with the fact that the weights of the cuts computed by the GL approach in this experiment are, in general, larger than those obtained by the GW algorithm.


\begin{sidewaystable*}[ph!]
\centering
\begin{tabular}{c|cc|cc|cc|cc|cc|cc|cc|cc} 
Dataset & \multicolumn{2}{c|}{Twoing} &  \multicolumn{2}{c|}{GWSG}  
&   \multicolumn{2}{c|}{GW$\chi^2$}                   &\multicolumn{2}{c|}{GLSG}       &\multicolumn{2}{c|}{GL$\chi^2$} & \multicolumn{2}{c|}{PC-ext} & \multicolumn{2}{c|}{HcC}& \multicolumn{2}{c}{LCA}\\
\hline 
% Dataset           &        Twoing     &     GWSG          &      GW$\chi^2$   &       GLSG        &    GL$\chi^2$     &     PC-ext                  &         HcC         &         LCA
{\tt Adult}         & 82.21    & (2)    & 81.91    & (1)    & 82.24    & (2)    & 81.83    & (0)    & 82.24    & (2)    &{\bf82.31}&{\bf(7)}          & 82.21    & (2)      & 82.21    & (2)      \\
{\tt Mushroom}      & {\bf 100}&{\bf(2)}& 99.99    & (0)    & 99.98    & (0)    & 100      &{\bf(2)}& 99.99    & (0)    &{\bf 100} &{\bf(2)}          &{\bf100}  & {\bf(2)} & {\bf100} & {\bf(2)} \\
{\tt KDD98}-2       & 80.47    & (0)    & 81       & (4)    & 80.74    & (1)    & 81.16    & (5)    & 80.51    & (0)    &{\bf81.25}&{\bf(6)}          & 80.47    & (0)      & 80.47    & (0)      \\
%{\tt KDD98}-3       & 63.77    &        & 63.8     &        & 64.37    &        & 63.75    &        &{\bf64.54}&        &          &                  &          &          &          &          \\
%{\tt KDD98}-5       & 48.02    &        & 46.79    &        &{\bf48.59}&        & 46.77    &        & 48.58    &        &          &                  &          &          &          &          \\
{\tt KDD98}-9       & 40.35    & (3)    & 38.13    & (0)    & 40.93    &{\bf(6)}& 38.15    & (0)    &{\bf 41 } &{\bf(6)}& 40.27    & (2)              & 40.14    & (2)      & 39.96    & (2)      \\
{\tt Nursery}       & 88.25    &{\bf(3)}& 88.03    & (0)    & 88.2     & (0)    &{\bf88.33}&{\bf(3)}& 88.2     & (0)    & 88.25    &{\bf(3)}          & 88.25    & {\bf(3)} & 88.25    & {\bf(3)} \\
{\tt Nursery-Ext}   &{\bf93.82}&{\bf(4)}& 90.95    & (0)    & 93.02    & (2)    & 90.75    & (0)    & 93.13    & (2)    & 93.81    &{\bf(4)}          & 93.81    & {\bf(4)} & 93.81    & {\bf(4)} \\
{\tt Cars}          & 86.53    & (2)    & 85.39    & (1)    & 86.37    & (2)    &{\bf87.93}&{\bf(7)}& 86.42    & (2)    & 86.5     & (2)              & 86.5     & (2)      & 84.55    & (0)      \\
{\tt Cars-Ext}      & 90.3     & (0)    & 90.82    & (4)    & 91.57    & (6)    & 90.84    & (4)    &{\bf 91.9}&{\bf(7)}& 90.32    & (0)              & 90.32    & (0)      & 90.32    & (0)      \\
{\tt Contracep}     & 43.77    & (0)    & 43.96    & (0)    &{\bf44.04}&{\bf(2)}& 43.89    & (0)    & 44       & (0)    & 43.59    & (0)              & 43.62    & (0)      & 43.82    & {\bf(2)} \\
{\tt Contracep-Ext} & 43.17    & (0)    & 44.32    &{\bf(6)}& 43.44    & (0)    &{\bf44.35}&{\bf(6)}& 43.7     & (0)    & 43.77    & (2)              & 43.36    & (0)      & 43.46    & (0)      \\
{\tt CoverType}     & 52.97    & (2)    & 55.05    & (3)    & 51.07    & (0)    & 55.05    & (3)    & 51.07    & (0)    &{\bf58.12}&{\bf(5)}          &{\bf58.12}& {\bf(5)} &{\bf58.12}& {\bf(5)} \\
{\tt CoverType-Ext} & 64.48    & (4)    & 64.12    & (2)    & 57.73    & (0)    & 64.23    & (3)    & 59.95    & (1)    & 64.71    & (6)              & 64.54    & (5)      &{\bf64.81}& {\bf(7)} \\
{\tt Poker }        & 51.9     & (5)    & 50.28    & (1)    & 51.77    & (3)    & 49.94    & (0)    &{\bf51.92}&{\bf(6)}& 51.7     & (3)              & 51.69    & (3)      & 51.57    & (2)      \\
{\tt Shelter-15}    & 48.01    & (3)    & ---      & (0)    & ---      & (0)    & 45.31    & (2)    & 48.13    & (4)    & 48.07    & (3)              & 48.05    & (3)      &{\bf48.26}& {\bf(7)} \\  
{\tt S.F. Crime-15} &{\bf22.1} &{\bf(3)}& 21.32    & (1)    & 22.09    & (2)    & 21.23    & (0)    & 22.09    & (2)    & 22.09    &{\bf(3)}          & 22.09    & (2)      & 22.09    & (2)      \\
{\tt Phonemes}      &{\bf30.92}&{\bf(7)}& ---      & (0)    & ---      & (0)    & 30.29    & (5)    & 29.47    & (3)    & 30.59    & (6)              & 29.92    & (4)      & 28.97    & (2)      \\
\hline
Average (Sum)       &  63.7    & (40)   & 66.8*    & (23)   & 66.66*   & (26)   & 63.33    & (40)   & 63.36    & (35)   & 64.08    & (54)             & 63.94    & (37)     & 63.79    & (40)

\end{tabular}
\caption{Average accuracy and statistical tests  for  decision trees with depth at most 5 using only nominal attributes. The best accuracy for each dataset is bold-faced. Experiments that did not finish in reasonable time are considered statistically worse than the others. These criteria have a * mark besides their average accuracies, since they are calculated only on the experiments that finished.}
\label{tab:nominal-5}
\normalsize
\end{sidewaystable*}


The results of  Table \ref{tab:nominal-5} also provide evidence of  the potential of considering aggregated attributes. The accuracy obtained for the extended versions of datasets
{\tt Nursery}, {\tt Cars} and {\tt CoverType} are considerably higher than those obtained for the original versions. For {\tt Contracep}, the effect is not clear.

Another key aspect to discuss is the computational cost of the proposed criteria. Table \ref{tab:time-5} shows the running time of each criterion in the experiment of Table \ref{tab:nominal-5}. Twoing is the fastest method when the number of classes is small and the GL-based methods  and PC-ext become competitive and eventually the fastest ones when the number of classes gets larger. As the number of classes increases, the GL-based methods and PC-ext become much faster than Twoing, with the turning point being around $k=7$. For datasets with 15 classes our GL criteria are 15-300 times faster. The same can be seen for PC-ext, which is 15-600 times faster. We also ran experiments using all the classes available in both the {\tt S.F. Crime} and {\tt Shelter} datasets (39 and 22, respectively). Twoing and the GW criteria can not be executed in a reasonable time with that many classes, while GLSG and GL$\chi^2$ ran in approximately 100 seconds on the {\tt S.F. Crime} dataset and  300 seconds on the {\tt Shelter} dataset (PC-ext ran in 75 and 32 seconds, respectively). This behavior for the Twoing criterion is not surprising, since its running time has an exponential dependence of the number of classes $k$. Nonetheless, since the execution time for our GL criteria in this experiment grew in an approximately linear fashion with $k$, it suggests that they can also be used with datasets that have a much larger number of classes. It is also interesting to note that the aggregated attributes usually appeared at or near the root of the decision trees. Lastly, the running time for the GW-based criteria were usually one or two orders of magnitude larger than the others. The only clear exception was in the CoverType dataset, where the number of samples is very large while the attributes’ number of values is much smaller. Hypercube Cover behaves very similarly to the Twoing criterion, but it is a bit slower because the impurity calculation with $k$ classes is slower than with $2$ classes.

Since the GW-criteria performe much worse in terms of execution time and yield comparable or worse accuracy than the other criteria, they were not used in any other experiments.

\begin{table}[]
\scriptsize
\centering
\begin{tabular}{c|c|c|c|c|c|c|c|c|c}
Dataset             & k  & Twoing     & GWSG       & GW$\chi^2$  & GLSG      & GL$\chi^2$ & PC-ext     & HcC        & LCA        \\
\hline
{\tt Adult}         & 2  & {\bf 2.7}  & 88.2       & 41.3        & 3         & 4          & 2.8        & 3.8        &            \\
{\tt Mushroom}      & 2  & {\bf 0.6}  & 8.6        & 6.9         & 0.9       & 1          & 0.7        & 0.9        &            \\
{\tt KDD98}-2       & 2  & {\bf 4}    & 3579.3     & 2162        & 43.5      & 44         & 5.2        & 5.8        &            \\
{\tt Contracep}     & 3  & {\bf 0.1}  & 1          & 0.6         & 0.1       & 0.1        & 0.1        & 0.1        &            \\
{\tt Contracep-Ext} & 3  & {\bf 0.1}  & 12.7       & 3.3         & 0.2       & 0.2        & 0.2        & 0.2        &            \\
%{\tt KDD98}-3       & 3  & {\bf 5.2}  &            &             & 60        & 56         &            &            &            \\
{\tt Cars}          & 4  & {\bf 0.1}  & 2.4        & 2.5         & 0.1       & 0.1        & 0.1        & 0.2        &            \\
{\tt Cars-Ext}      & 4  & {\bf 0.2}  & 11.3       & 7.7         & 0.3       & 0.3        & 0.2        & 0.4        &            \\
{\tt Nursery}       & 5  & {\bf 0.8}  & 5          & 4.7         & 1         & 0.9        & 0.9        & 1.2        &            \\
{\tt Nursery-Ext}   & 5  & {\bf 1.1}  & 147.9      & 75.8        & 3.3       & 2.6        & 1.4        & 1.7        &            \\
%{\tt KDD98}-5       & 5  & {\bf 11}   &            &             & 81        & 63         &            &            &            \\
{\tt CoverType}     & 7  & 348.5      & {\bf 179.2}&  265.4      & 245.2     & 307.8      & 271.4      & 338.5      &            \\
{\tt CoverType-Ext} & 7  & 213        & 212.5      & 340.8       &{\bf 182.5}& 295.7      & 196.5      & 258.9      &            \\
{\tt KDD98}-9       & 9  & 132        & 5898.8     & 3410.1      & 97.2      & 73.9       & {\bf 14.7 }& 297.5      &            \\ 
{\tt Poker}         & 10 & 7.4        & 11.2       & 6.4         & 2.2       & {\bf 2.1}  & 2.5        & 13.3       &            \\
{\tt Shelter-15}    & 15 & 3599       & ---        & ---         & 149.9     & 166.3      & {\bf 14.1 }& 7968       &            \\   
{\tt S.F. Crime-15} & 15 & 638.2      & 605.2      & 823.7       & 40.7      & {\bf 40.2} & 41.7       & 1223.9     &            \\ 
{\tt Phonemes}      & 15 & 1343.3     & ---        & ---         & 4.4       & 5.5        & {\bf 2.2 } & 2187.4     &
\end{tabular}
\caption{Average time in seconds of a 3-fold cross validation for building decision trees with depth at most 5. The fastest method for each dataset is bold faced.}
\label{tab:time-5}
\end{table}



Table \ref{tab:ctree-5} presents the 
comparison between our GL methods, Twoing, Hypercube Cover and PC-ext in another scenario,
where we use $c_{quad}$, one of the bias-free criterion proposed in \cite{Hothorn:2006:URP}, to select the attribute at each node of the tree. 
Then, both Twoing, Hypercube Cover, PC-ext and our methods are  used only for splitting the chosen attribute, which allows for a  more direct comparison of their splitting ability. Once again we observed a balance between most of the criteria, with a significant advantage towards GL-$\chi^2$ and, perhaps surprisingly, a very poor performance by PC-ext. the bias free approach had significantly worse results for the datasets with extended attributes. This suggests that its advantage lies in comparing different attributes, and not necessarily finding the best split. This experiment also showed that it is not possible to run $c_{quad}$ in reasonable time for the {\tt Shelter-15} dataset. This happens because it calculates a pseudoinverse of a matrix whose dimension grows with the number of values and classes, which is infeasible for large $n$ and $k$.


\begin{table}
\scriptsize
\centering
\begin{tabular}{c|cc|cc|cc|cc|cc|cc} 
Dataset  &   \multicolumn{2}{c|}{CI-Twoing} &   \multicolumn{2}{c|}{CI-GLSG} & \multicolumn{2}{c|}{CI-GL$\chi^2$}& \multicolumn{2}{c|}{CI-PC-ext}& \multicolumn{2}{c|}{CI-HcC}& \multicolumn{2}{c}{CI-LCA}\\
\hline   
% Dataset          &        CI-Twoing       &        CI-GLSG          &      CI-GL$\chi^2$      &       CI-PC-ext         &       CI-HcC              &          CI-LCA
{\tt Adult}        &{\bf 81.96} &{\bf  (3)} & 81.61       & (1)       & 81.77       & (2)       & 79.98       & (0)       & {\bf 81.96} & {\bf  (3)}  & {\bf 81.96} & {\bf  (3)}  \\
{\tt Mushroom}     &86.97       & (1)       &{\bf  94.79 }& {\bf (5)} & 90.15       & (4)       & 86.77       & (0)       & 86.97       & (1)         & 86.97       & (1)         \\
{\tt KDD98}-2      &81.29       & (0)       & {\bf 82.34 }& {\bf (5)} & 81.68       & (4)       & 81.35       & (0)       & 81.29       & (0)         & 81.29       & (0)         \\
%{\tt KDD98}-3      &65.28       &           & {\bf 65.8  }&           & 65.07       &           &             &           &             &             &             &             \\
%{\tt KDD98}-5      &49.23       &           & {\bf 49.58} &           & 49.27       &           &             &           &             &             &             &             \\
{\tt KDD98}-9      &41.84       & (0)       & 42          & (0)       & {\bf 42.26} & {\bf (4)} & 41.73       & (0)       & 41.95       &  (1)        & 41.83       & (0)         \\
{\tt Nursery}      &{\bf 88.48} & {\bf (2)} & 88.3        & (0)       & 88.47       & {\bf (2)} &{\bf 88.48 } & {\bf (2)} & {\bf 88.48} & {\bf (2)}   & 88.09       & (0)         \\
{\tt Nursery-Ext}  &{\bf 88.48} & {\bf (2)} & 88.26       & (0)       & 88.47       & {\bf (2)} &{\bf 88.48 } & {\bf (2)} & {\bf 88.48} & {\bf (2)}   & 88.11       & (0)         \\
{\tt Cars}         &86.51       & {\bf (2)} & 85.02       & (0)       & {\bf 86.57} & {\bf (2)} & 86.48       & {\bf (2)} & 86.48       & {\bf (2)}   & 86.27       & (1)         \\
{\tt Cars-Ext}     &88.27       & {\bf (2)} & 85.51       & (0)       & {\bf 88.32} & {\bf (2)} & 88.26       & {\bf (2)} & 88.26       & {\bf (2)}   & 88.16       & (1)         \\
{\tt Contracep}    &43.83       & (1)       & {\bf 44.12} & {\bf (4)} & 43.87       & (0)       & 43.63       & (0)       & 43.69       & (1)         & 43.58       & (0)         \\
{\tt Contracep-Ext}&43.39       & (0)       & {\bf 43.81} & {\bf (4)} & 43.76       & {\bf (4)} & 43.21       & (0)       & 43.32       & (1)         & 43.21       & (0)         \\
{\tt CoverType}    &54.1        & (0)       & 54.1        & (0)       & 54.1        & (0)       & 54.1        & (0)       & 54.1        & (0)         & 54.1        & (0)         \\
{\tt CoverType-Ext}&54.1        & (0)       & 54.1        & (0)       & 54.1        & (0)       & 54.1        & (0)       & 54.1        & (0)         & 54.1        & (0)         \\
{\tt Poker}        &{\bf 51.05} & {\bf (4)} & 50.56       & (0)       & 50.91       & (1)       & 50.8        & (1)       & 50.79       & (1)         & 50.73       & (1)         \\  
{\tt Shelter-15}   & 48.03      &           & 48.03       &           & {\bf 48.22} &           & 48.22       &           & 48.2        &             &             &             \\   
{\tt S.F. Crime-15}&{\bf 21.59} & {\bf (5)} & 20.53       & (0)       & 21.49       & (2)       & 21.52       & (2)       & 21.52       & (2)         & 21.34       & (1)         \\ 
{\tt Phonemes}     & 23.12      & (4)       & 22.52       & (3)       & {\bf 23.8 } & {\bf (5)} & 21.3        & (1)       & 22.11       & (2)         & 20.4        & (0)         \\ 
\hline
Average (Sum)      & 61.44      & (26)+     & 61.6        & (22)+     & 61.75       & (34)+     & 61.15       & (12)+     & 61.36       & (20)        &             & (8)+
       \end{tabular}
    \caption{Average accuracy and statistical tests  for  Conditional Inference trees 
with depth at most 5 using only nominal attributes. The best accuracy for each dataset is bold-faced.}
\label{tab:ctree-5}
\end{table}


Table \ref{exp:numeric-5} shows experiments  similar to those presented at Table \ref{tab:nominal-5}, but now using also the numeric attributes. We observed a significant gain in terms of accuracy for all datasets except for KDD98-2. Also note that GLSG was much inferior to the other criteria. Other than that, the comparison results were very similar to the ones found in Table \ref{tab:nominal-5}, with a slight better performance by Hypercube Cover and GL-$\chi^2$ and slightly worse by Twoing.

\begin{table}
\scriptsize
\centering
\begin{tabular}{c|cc|cc|cc|cc|cc|cc} 
Dataset            &\multicolumn{2}{c|}{Twoing} & \multicolumn{2}{c|}{GLSG} & \multicolumn{2}{c|}{GL$\chi^2$} & \multicolumn{2}{c|}{PC-ext}& \multicolumn{2}{c|}{HcC} & \multicolumn{2}{c}{LCA} \\
\hline   
{\tt Adult}        & {\bf 84.4 }    & {\bf (2)} & 82.36      & (0)          &  84.15      &  (1)              & 84.4        & {\bf (2)}    & {\bf 84.4 } & {\bf (2)} & {\bf 84.4 } & {\bf (2)} \\
{\tt KDD98}-2      & 81.89          & (0)       & 81.92      & (0)          & {\bf 81.93} &  (0)              & 81.9        & (0)          & 81.89       & (0)       & 81.89       & (0)       \\ 
%{\tt KDD98}-3      &                &           & 67.9       &              & 69.1        &                   &             &              &             &           &             &           \\ 
%{\tt KDD98}-5      &                &           & 54.5       &              & {\bf 55.3 } &                   &             &              &             &           &             &           \\ 
{\tt KDD98}-9      & 46.78          & (1)       & 45.32      & (0)          & {\bf 48.73} & {\bf (5)}         &  47.38      & (4)          & 47.04       & (1)       & 46.99       & (1)       \\ 
{\tt Contracep}    & 54.39          & (1)       & 52.85      & (0)          & 53.99       & (1)               & {\bf 55.11} & {\bf (4)}    & 55.04       & (3)       & 54.95       & (3)       \\  
{\tt Contracep-Ext}& 52.41          & (0)       & 52.27      & (0)          & 53.26       & (2)               & {\bf 53.31} & {\bf (4)}    & 52.75       & (1)       & 52.77       & (0)       \\
{\tt CoverType}    & 70.07          & (2)       & 68.52      & (0)          & 69.23       & (1)               & {\bf 70.24} & {\bf (3)}    & {\bf 70.24} & {\bf (3)} & {\bf 70.24} & {\bf (3)} \\  
{\tt CoverType-Ext}& 71.6           & (3)       & 70.3       & (1)          & 69.23       & (0)               & 71.66       & (4)          & 71.25       & (2)       & {\bf 71.7 } & {\bf (5)} \\ 
{\tt Shelter-15}   & 54.68          & (2)       & 51.91      & (0)          & 54.57       & (1)               & {\bf 54.83} & {\bf (5)}    & 54.64       & (2)       & 54.53       & (1)       \\   
{\tt S.F. Crime-15}& 23.52          & (4)       & 23.18      & (0)          & {\bf 23.58} & {\bf (5)}         & 23.49       & (1)          & 23.49       & (1)       & 23.49       & (1)       \\ 
\hline
Average (Sum)      & 59.97          & (15)      & 58.74      & (1)          & 59.85       & (16)              & 60.26       & (27)         & 60.08       & (15)      & 60.11       & (16)

\end{tabular}
\caption{Average accuracy and statistical test results for  Decision Trees using both nominal and numeric attributes.}
\label{exp:numeric-5}
\normalsize
\end{table}


\subsection{Maximum Depth 16}

In this subsection we explore the same experiments studied in the previous one, except now the maximum depth allowed is 16. This larger depth is common in most random forest methods, thus the importance of this analysis.

Table \ref{tab:nominal-16} presents the results of an experiment to compare the accuracy of  Decision Trees built by our GL-methods with those built by Twoing, Hypercube Cover and PC-ext, using just the nominal attributes of the datasets. Once again PC-ext had the best results, but this time by a smaller margin. This suggests the possibility that, for a larger depth, it might not have the best results. Moreover, there was a balance between Twoing, Hypercube Cover and GL$\chi^2$, with GLSG being significantly worse. This suggests that GLSG loses competitiveness as the tree depth increases.

Table \ref{tab:ctree-16} presents the comparison between the different methods in the scenario where we use $c_{quad}$ to select the attribute at each node of the tree, and the criteria are used only for splitting the chosen attribute. We observed a very strong advantage for GL$\chi^2$, followed by Twoing. Both Hypercube Cover and GLSG are competitive among themselves, while PC-ext had the worse results.

\begin{table}
\scriptsize
\centering
\begin{tabular}{c|cc|cc|cc|cc|cc|cc} 
Dataset             & \multicolumn{2}{c|}{Twoing} &  \multicolumn{2}{c|}{GLSG}  & \multicolumn{2}{c|}{GL$\chi^2$} & \multicolumn{2}{c|}{PC-ext}& \multicolumn{2}{c|}{HcC}& \multicolumn{2}{c}{LCA}\\
\hline
{\tt Adult}         &  {\bf 82.52} & {\bf (2)}    &  82.38       & (0)          &  82.43       & (0)              & 82.51      & {\bf (2)}    & {\bf 82.52}& {\bf (2)}  & {\bf 82.52}& {\bf (2)}  \\
{\tt Mushroom}      &  {\bf 100}   & {\bf (1)}    &  100         & (0)          &  99.99       & (0)              & {\bf 100 } & {\bf (1)}    & {\bf 100}  & {\bf (1)}  & {\bf 100}  & {\bf (1)}  \\
{\tt KDD98}-2       &  79.41       & (0)          &  {\bf 80.65} & {\bf (5)}    &  79.73       & (0)              & 79.91      & (3)          & 79.41      & (0)        & 79.41      & (0)        \\
%{\tt KDD98}-3       &  62.61       &              &  63.33       &              &  {\bf 63.56} &                  &            &              &            &            &            &            \\
%{\tt KDD98}-5       &  46.46       &              &  46.39       &              &  {\bf 47.53} &                  &            &              &            &            &            &            \\
{\tt KDD98}-9       &  38.54       & (3)          &  37.97       & (0)          &  {\bf 39.68} & {\bf (5)}        & 38.55      & (3)          & 37.95      & (0)        & 38.12      & (0)        \\
{\tt Nursery}       &  93.51       & (2)          &  92.39       & (0)          &  {\bf 93.74} & {\bf (5)}        & 93.5       & (1)          & 93.5       & (1)        & 93.5       & (1)        \\
{\tt Nursery-ext}   &  95.83       & (1)          &  94.79       & (0)          &  {\bf 96.02} & {\bf (5)}        & 95.83      & (1)          & 95.83      & (1)        & 95.83      & (1)        \\
{\tt Cars}          &  92.82       & {\bf (1)}    &  90.51       & (0)          &  {\bf 92.88} & {\bf (1)}        & 92.69      & {\bf (1)}    & 92.69      & {\bf (1)}  & 90.14      & {\bf (1)}  \\
{\tt Cars-ext}      &  96.49       & {\bf (2)}    &  90.89       & (0)          &  94.44       & (1)              & 96.46      & {\bf (2)}    & {\bf 96.5} & {\bf (2)}  & {\bf 96.5} & {\bf (2)}  \\
{\tt Contracep}     &  43.5        & (0)          &  43.83       & (0)          &  {\bf 43.85} & {\bf (1)}        & 43.53      & (0)          & 43.53      & (0)        & 43.66      & (0)        \\
{\tt Contracep-ext} &  43.15       & (0)          &  {\bf 44.32} & {\bf (5)}    &  43.72       & (0)              & 43.75      & (3)          & 43.37      & (0)        & 43.35      & (0)        \\
{\tt CoverType}     &  64.09       & (1)          &  64.59       & (2)          &  63.61       & (0)              &{\bf 64.66} & {\bf (3)}    & {\bf 64.66}& {\bf (3)}  & {\bf 64.66}& {\bf (3)}  \\
{\tt CoverType-ext} &  65.08       & (1)          &  65.08       & (1)          &  65.05       & (0)              & 65.08      & (1)          & 65.08      & (1)        & {\bf 65.08}& {\bf (4)}  \\
{\tt Poker}         &  {\bf 52.24} & {\bf (3)}    &  49.88       & (0)          &  51.83       & (1)              & 52.09      & {\bf (3)}    & 52.09      & {\bf (3)}  & 51.97      & (1)        \\
{\tt Shelter-15}    &  47.64       & (2)          &  46.52       & (0)          & {\bf 48.09}  & {\bf (5)}        & 47.71      & (3)          & 47.26      & (1)        & 47.58      & (2)        \\
{\tt S.F. Crime-15} &  22.08       & {\bf (2)}    &  22.06       & (0)          & {\bf 22.08}  & (1)              & 22.08      & {\bf (2)}    & 22.08      & (1)        & 22.08      & (1)        \\
{\tt Phonemes}      &  37.13       & (2)          &  35.9        & (0)          & 35.75        & (0)              &{\bf 37.95} & {\bf (3)}    & 37.89      & {\bf (3)}  & 37.8       & {\bf (3)}  \\
\hline
Average (Sum)       &  65.88       & (23)         &  65.11       & (13)         & 65.81        & (25)             &  66.02     & (32)         & 65.9       & (20)       & 65.76      & (22)
\end{tabular}
\normalsize
\caption{Average accuracy and statistical tests  for  decision trees with depth at most 16 using only nominal attributes. The best accuracy for each dataset is bold-faced, even when multiple criteria have the same accuracy in the table because of rounding.}
\label{tab:nominal-16}
\end{table}


\begin{table}
\scriptsize
\centering
\begin{tabular}{c|cc|cc|cc|cc|cc|cc} 
Dataset             & \multicolumn{2}{c|}{CI-Twoing} &   \multicolumn{2}{c|}{CI-GLSG} & \multicolumn{2}{c|}{CI-GL$\chi^2$} & \multicolumn{2}{c|}{CI-PC-ext}& \multicolumn{2}{c|}{CI-HcC}& \multicolumn{2}{c}{CI-LCA}\\
\hline   
{\tt Adult}         & 82.33      &  {\bf (2)}        &   82.18      & (0)             & {\bf 82.35} &  {\bf (2)}           & 82.28       & (1)             & 82.33       & {\bf (2)}   & 82.33       & {\bf (2)}   \\
{\tt Mushroom}      & 99.55      &  (1)              &   99.6       & (1)             & {\bf 99.64} &  {\bf (4)}           & 99.4        & (0)             & 99.55       & (1)         & 99.55       & (1)         \\
{\tt KDD98}-2       & 79.92      &  (1)              &  {\bf 81.53} & {\bf (5)}       &  80.65      &  (4)                 & 79.72       & (0)             & 79.92       & (1)         & 79.92       & (1)         \\
%{\tt KDD98}-3       &63.72       &                   &  {\bf 64.66} &                 &  63.56      &                      &             &                 &             &             &             &             \\
%{\tt KDD98}-5       &46.88       &                   &  {\bf 48}    &                 &  47.75      &                      &             &                 &             &            &             &             \\
{\tt KDD98}-9       & 39.44      &  (3)              &  {\bf 40.65} & {\bf (5)}       &  40.01      &  (4)                 & 38.85       & (0)             & 39.07       & (0)         & 39.16       & (0)         \\
{\tt Nursery}       & 93.55      &  {\bf (2)}        &   92.22      & (0)             & {\bf 93.56} &  (1)                 & 93.55       & {\bf (2)}       & 93.55       & {\bf (2)}   & 93.54       & (1)         \\
{\tt Nursery-ext}   & 93.64      &  {\bf (2)}        &   92.32      & (0)             & {\bf 93.65} &  (1)                 & 93.64       & {\bf (2)}       & 93.64       & {\bf (2)}   & 93.63       & (1)         \\
{\tt Cars}          & 92.74      &  (2)              &   87.19      & (0)             & {\bf 92.92} &  {\bf (5)}           & 92.67       & (2)             & 92.67       & (2)         & 92.35       & (1)         \\
{\tt Cars-ext}      & 93.12      &  (2)              &   87.38      & (0)             & {\bf 93.34} &  {\bf (5)}           & 93.08       & (2)             & 93.08       & (2)         & 92.85       & (1)         \\
{\tt Contracep}     & 43.82      &  {\bf (3)}        &   {\bf 44.05}& {\bf (3)}       &  43.81      &  (0)                 & 43.59       & (0)             & 43.6        & (1)         & 43.57       & (0)         \\
{\tt Contracep-ext} & 43.21      &  (1)              &   {\bf 43.78}& {\bf (4)}       &  43.72      &  {\bf (4)}           & 43.07       & (0)             & 43.12       & (1)         & 42.92       & (0)         \\
{\tt CoverType}     & 61.88      &  (0)              &   61.88      & (0)             &  61.88      &  (0)                 & 61.88       & (0)             & 61.88       & (0)         & 61.88       & (0)         \\
{\tt CoverType-ext} & 61.88      &  (0)              &   61.88      & (0)             &  61.88      &  (0)                 & 61.88       & (0)             & 61.88       & (0)         & 61.88       & (0)         \\
{\tt Poker}         & {\bf 51.4} &  {\bf (5)}        &   50.67      & (0)             &  50.84      &  (1)                 & 51.04       & (1)             & 51.03       & (1)         & 50.98       & (1)         \\
{\tt Shelter-15}    & 47.66      &                   &   47.32      &                 & {\bf 48.14} &                      & 47.82       &                 &             &             &             &             \\
{\tt S.F. Crime-15} & 22.07      &  (0)              &   22.07      & (0)             & {\bf 22.07} &  {\bf (1)}           & 22.07       & (0)             & 22.07       & (0)         & 22.07       & {\bf (1)}   \\
{\tt Phonemes}      & 33.91      &  (3)              &   31.26      & (0)             &  33.32      &  (2)                 & 34.08       & (3)             & {\bf 34.77} & {\bf (5)}   & 32.62       & (1)         \\
\hline
Average (Sum)       & 65.01      &  (27)+            &   64.12      & (17)+           &  65.11      &  (34)+               & 64.91       & (13)+           &             & (20)+       &             & (11)+
       \end{tabular}
        \caption{Average accuracy and statistical tests  for  conditional inference trees 
with depth at most 16 using only nominal attributes. The best accuracy for each dataset is bold-faced, even when multiple criteria have the same accuracy in the table because of rounding.}
\label{tab:ctree-16}
\normalsize
\end{table}


Table \ref{tab:time-16} shows the running time of each criteria when used for both selecting and splitting purposes (the experiment of Table \ref{tab:nominal-16}).
When the number of classes is small all the criteria have very similar execution time. After running this experiments a few times, we have realized there is a variation of about 2-3 seconds in the execution time. This explains why Hypercube Cover is sometimes faster than Twoing.

As the number of classes increases, both PC-ext and the GL-based methods become much faster than Twoing and Hypercube Cover, with the turning point being once again around $k=7$. For datasets with 15 classes our criteria are 30-400 times faster than Twoing, while PC-ext is 50-600 times faster. We also ran experiments using all the classes available in both the {\tt S.F. Crime} and {\tt Shelter} datasets (39 and 22, respectively). Twoing cannot be executed in a reasonable time with that many classes, while GLSG and GL$\chi^2$ ran in approximately 100 seconds on the {\tt S.F. Crime} dataset and  300 seconds on the {\tt Shelter} dataset (PC-ext ran in 110 and 33 seconds, respectively). Since the execution time for our criteria in this experiment grew onde again in an approximately linear fashion with $k$, it suggests that they can also be used with datasets that have a much larger number of classes. It is also interesting to note that the aggregated attributes usually appeared at or near the root of the decision trees.

\begin{table}[]
\scriptsize
\centering
\begin{tabular}{c|c|c|c|c|c|c|c}
Dataset             & k  & Twoing        & GLSG      & GL$\chi^2$  & PC-ext    & HcC    & LCA    \\
\hline
{\tt Adult}         & 2  & 4.1           & 4.6       & 5.7         &{\bf 4}    & 5.8    &        \\
{\tt Mushroom}      & 2  & 1.7           & 1.5       & 2.3         &{\bf 0.6}  & 1      &        \\
{\tt KDD98}-2       & 2  & 9.1           & 50.9      & 53.4        &{\bf 8.3}  & 10.4   &        \\
{\tt Contracep}     & 3  & 0.2           & 0.2       & 0.2         &{\bf 0.1}  & 0.1    &        \\
{\tt Contracep-Ext} & 3  & 0.2           & 0.3       & 0.4         &{\bf 0.2}  & 0.3    &        \\
%{\tt KDD98}-3       & 3  & {\bf 12.5}    & 73.3      & 82.4        &           &        &        \\
{\tt Cars}          & 4  & 0.3           & 0.3       & 0.2         &{\bf 0.2}  & 0.3    &        \\
{\tt Cars-Ext}      & 4  & 0.3           & 0.5       & 0.4         &{\bf 0.2}  & 0.3    &        \\
{\tt Nursery}       & 5  & 4.1           & 3.9       & 3.7         &{\bf 1.3}  & 1.9    &        \\
{\tt Nursery-Ext}   & 5  & 4.6           & 9.5       & 8.6         &{\bf 1.4}  & 2.3    &        \\
%{\tt KDD98}-5       & 5  & {\bf 22.2}    & 100.8     & 82.6        &           &        &        \\
{\tt CoverType}     & 7  & 612.7         &{\bf 294.7}& 700         & 472.4     & 668.9  &        \\
{\tt CoverType-Ext} & 7  & 251.8         & 209.9     & 462.6       &{\bf 200.4}& 296.2  &        \\
{\tt KDD98}-9       & 9  & 434.5         & 256.2     & 243.6       &{\bf 23.8} & 442.9  &        \\
{\tt Poker}         & 10 & 25.2          & 10.8      & 9.2         &{\bf 3.8}  & 18     &        \\
{\tt Shelter-15}    & 15 & 5397          & 181.9     & 191.2       &{\bf 18.6} & 7311.6 &        \\
{\tt S.F. Crime-15} & 15 & 2731.2        & 95.9      & 81.8        &{\bf 50.5} & 3188.9 &        \\
{\tt Phonemes}      & 15 & 3894.6        & 9.4       & 11          &{\bf 6.3}  & 5804.8
\end{tabular}
\caption{Average time in seconds of a 3-fold cross validation
for building decision trees with depth at most 16.
The fastest method for each dataset is bold-faced.}
\label{tab:time-16}
\end{table}


Table \ref{exp:numeric-16} shows experiments  similar to those presented at Table \ref{tab:nominal-16}, except now it also uses the numeric attributes. We observed a significant gain in terms of accuracy for almost all datasets. This time Hypercube Cover has the best results, followed by PC-ext. The other 3 criteria have similar, but inferior, results.

\begin{table}
\scriptsize
\centering
\begin{tabular}{c|cc|cc|cc|cc|cc|cc} 
Dataset              &        \multicolumn{2}{c|}{Twoing} &   \multicolumn{2}{c|}{GLSG} &   \multicolumn{2}{c|}{GL$\chi^2$} & \multicolumn{2}{c|}{PC-ext}  & \multicolumn{2}{c|}{HcC}  & \multicolumn{2}{c}{LCA}  \\
\hline   
{\tt Adult}          &  83.25         &  {\bf (1)}        &  77.34      &  (0)          &  83.21       &  {\bf (1)}         & {\bf 83.28} & {\bf (1)}      & 83.25        & {\bf (1)} & 83.25        & {\bf (1)} \\
{\tt KDD98}-2        &  77.14         &  (2)              &  76.36      &  (1)          &  76.04       &  (0)               & {\bf 77.87} & {\bf (5)}      & 77.14        & (2)       & 77.14        & (2)       \\
%{\tt KDD98}-3        &                &                   &  62.21      &               &  {\bf 63.94} &                    &             &                &              &           &              &           \\
%{\tt KDD98}-5        &                &                   &  46.35      &               &  {\bf 49.71} &                    &             &                &              &           &              &           \\
{\tt KDD98}-9        &  38.73         &  (1)              &  37.49      &  (0)          &  {\bf 43.45} &  {\bf (5)}         &  39.88      & (4)            & 38.96        & (1)       & 38.8         & (1)       \\
{\tt Contracep}      &  {\bf 48.95}   &  {\bf (1)}        &  48.01      &  (0)          &  48.66       &  {\bf (1)}         &  48.93      & {\bf (1)}      & 48.86        & {\bf (1)} & 48.93        & {\bf (1)} \\
{\tt Contracep-Ext}  &  48.82         &  (1)              &  48.15      &  (0)          &  48.6        &  (0)               &  48.52      & (0)            & {\bf 49.31 } & {\bf (5)} & 48.97        & (2)       \\
{\tt CoverType}      &  86.43         &  (4)              &  {\bf 90.32}&  {\bf (5)}    &  81.38       &  (0)               &  86.23      & (1)            & 86.23        & (1)       & 86.23        & (1)       \\
{\tt CoverType-Ext}  &  88.96         &  (3)              &  {\bf 92.03}&  {\bf (5)}    &  82.46       &  (0)               &  88.32      & (1)            & 89.39        & (4)       & 88.72        & (2)       \\
{\tt Shelter-15}     &  53.97         &  (4)              &  52         &  (0)          &  {\bf 54.4}  &  {\bf (5)}         &  53.82      & (3)            & 53.59        & (1)       & 53.6         & (1)       \\   
{\tt S.F. Crime-15}  &                &                   &  26.71      &               &  27.13       &                    &  27.13      &                & 27.13        &           & {\bf 27.16}  &           \\
\hline
Average (Sum)        &                &  (17)+            & 60.93       &  (11)+        &   60.59      &  (12)+             &  61.55      & (16)+          &              & (16)+     & 61.42        & (11)+

\end{tabular}
\caption{Average accuracy and statistical test results for  Decision Trees using both nominal and numeric attributes with depth at most 16.}
\label{exp:numeric-16}
\normalsize
\end{table}


\newpage

\chapter{Conclusions}
\label{chap:conclusions}

In this dissertation we proposed a framework for designing splitting criteria for handling multi-valued nominal attributes. Criteria derived from our framework can be implemented to run in polynomial time in $n$ and $k$, with theoretical guarantee of producing a split that is close to the optimal one. We also made an experimental study on criteria based on different heuristics, some of them with approximation guarantees (Hypercube Cover and Largest Class Alone), some of them without one (PC-ext). We compared their ability of finding splits with the lowest impurity and, later, compared them with Twoing and criteria from our framework to analyse the accuracy of trees obtained by using them.

Experiments over 11 datasets suggest that the GL$\chi^2$ criterion, obtained from our framework, is competitive with the well-established Twoing criterion in terms of both accuracy and speed for datasets with a small number of classes ($k \leq 7$). It is also much faster than Twoing when the number of classes is greater than 10, while keeping a comparable accuracy. Overall, Hypercube Cover also had results similar to Twoing. Therefore, our methods are an interesting alternative to deal with datasets with a large number of classes that contain nominal attributes with a large number of different values, since those cannot be properly handled by Twoing due to its exponential running time dependence on the number of classes.

Even though the PC-ext criterion does not have a theoretical guarantee, the experiments also show that it has some advantage in terms of accuracy over the other methods, except when used in the conditional inference tree framework. This suggests that PC-ext is very good in terms of comparing different attributes among themselves, but not as good in finding the best split for a given attribute. In these bias-free experiments, the GL$\chi^2$ criterion had the best results. In terms of speed, it also has an advantage over every other criterion, except for LCA.

Although we discovered, in Chapter \ref{chap:experiments-splits}, that the Largest Class Alone heuristic obtains splits with worse impurity than Hypercube Cover and PC-ext, we saw in Chapter \ref{chap:experiments-datasets} that trees created using it have competitive accuracy. This suggests one should use it when the dataset is big enough for the training time to be a concern.

In practice, one should probably use PC-ext to train decision trees when its training time is acceptable, otherwise one should switch to Largest Class Alone. Furthermore, when using the Conditional Inference Tree framework, the best splitting criterion to use is GL-$\chi^2$. Its running time is polynomial---in general dominated by the framework itself---and obtains trees with the best accuracies. This indicates that, given an attribute, it is the best at choosing splits.

Lastly, our experiments also reinforce the potential of aggregating attributes as a tool for improving the accuracy of decision trees. An interesting topic for  future research is evaluating the behavior of our criteria in boosted tree methods. Another direction for future work is developing new methods for automatic aggregating attributes, or improving the available ones.


\begin{raggedright}
	\bibliography{thesis}
\end{raggedright}

\end{document}