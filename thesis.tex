\documentclass[msc,british,bibkey]{ThesisPUC_uk}

%---------- Math ----------%
\usepackage{amsmath}
\usepackage{amssymb}
%\usepackage{amsthm}
\usepackage{mathtools}
\usepackage{bbm}
%---------- Floting ----------%
\usepackage{float}
%---------- References ----------%
\usepackage{natbib}
%---------- Algorithm ----------%
\usepackage{algorithm}
\usepackage{algorithmic}
%---------- Tables ----------%
\usepackage{rotating}
\usepackage{tabularx}
\usepackage{multicol}
\usepackage{multirow}
\usepackage{booktabs}
\usepackage{subcaption}
\usepackage{diagbox}
%---------- Others ----------%
\usepackage{hyperref}
\usepackage{bold-extra}
\usepackage{graphicx}

%\colorlet{dark}{red!85!blue!60!black}

%---------- Cover ----------%
\author{\mbox{Felipe} \mbox{de} \mbox{Albuquerque} \mbox{Mello} \mbox{Pereira}}
\authorR{\mbox{Pereira}, \mbox{Felipe} \mbox{de} \mbox{Albuquerque} \mbox{Mello}}
\advisor{\mbox{Eduardo} \mbox{Sany} \mbox{Laber}}
\advisorR{\mbox{Laber}, \mbox{Eduardo} \mbox{Sany}}
\title{Binary Splitting Criteria for Large Categorical Attributes in Decision Trees}
\day{28$^{th}$} \month{February} \myyear{2018}

\city{Rio de Janeiro}
\CDD{004}
\department{Inform\'atica}
\departmentbr{Inform\'atica}
\program{Inform\'atica}
\programbr{Inform\'atica}
\school{Centro T\'{e}cnico Cient\'{\i}fico}
\university{Pontif\'{\i}cia Universidade Cat\'{o}lica do Rio de Janeiro}
\uni{PUC-Rio}

%---------- Jury ----------%

\jury{
  \jurymember{H\'elio C\^ortes Vieira Lopes}{Departamento de Inform\'atica --- PUC-Rio}
  \jurymember{Marco Serpa Molinaro}{Departamento de Inform\'atica --- PUC-Rio}
  \jurymember{M\'arcio da Silveira Carvalho}{Vice Dean of Graduate Studies\\ Centro T\'ecnico Cient\'ifico da PUC-Rio}
}

%---------- Front letters ----------%
\resume
{
Bachelor's in Electrical Engineering and Pure Mathematics at the Pontif\'icia Universidade Cat\'olica do Rio de Janeiro (2010 and 2011). Masters' in Mathematics at the Pontif\'icia Universidade Cat\'olica do Rio de Janeiro (2013).
}

\acknowledgment
{
TODO: acknowledgment.

Thanks to CNPq for the conceded scholarship during my Masters.
}

\keywords
{
  \key{Decision Trees; Max-cut Problem; Approximated Algorithms}
}

\abstract{
In this dissertation we proposed a framework for designing splitting criteria for handling multi-valued nominal attributes for decision trees. Criteria derived from our framework can be implemented to run in polynomial time in the number of classes and values, with theoretical guarantee of producing a split that is close to the optimal one. This is the only known criteria that have all these characteristics simultaneously. We also run multiple experiments to evaluate its running times and accuracy in real datasets.
}

\titulobr{Crit\'erios de Splits Bin\'arios para Atributos Categóricos Grandes em \'Arvores de Decis\~ao}
\departamentobr{Inform\'atica}

\chavesbr
{
  \chave{\'Arvores de Decis\~ao; Problema de Corte M\'aximo; Algoritmos Aproximativos}
}

\resumobr
{
Nesta disserta\c{c}\~ao \'e apresentado um framework para desenvolver crit\'erios de split para lidar com atributos nominais multi-valorados em \'arvores de decis\~ao. Crit\'erios gerados por este framework podem ser implementados para rodar em tempo polinomial no n\'umero de classes e valores, com garantia te\'orica de produzir um split pr\'oximo do \'otimo. Este \'e o \'unico crit\'erio conhecido que possui ambas caracter\'isticas simultaneamente. Tamb\'em s\~ao realizados m\'ultiplos experimentos para avaliar seu tempo de execu\c{c}\~ao e acur\'acia em datasets reais.
}

\tablesmode{figtab}


%%%%%%%%%%%%%%%%%%%%%%%%%%%%%%%%%%%%%%%%%%%%%%%%%%%%%%

\begin{document}

\newcommand{\remove}[1]{}
\newcommand{\dist}[3]{d(#1, #2)}
\newcommand{\distC}[2]{d(#1, #2)}
\newcommand{\OPT}[1]{\textrm{OPT}(#1)}
\newcommand{\OPTf}[2]{\textrm{OPT}(#1, #2)}
\newcommand{\Cf}[2]{\textrm{cost}(#1, #2)}
\newcommand{\C}[1]{\textrm{cost}(#1)}
\newcommand{\ans}[1]{\textbf{#1}}
\newcommand{\bl}{\textrm{blocked}}
\newcommand{\un}{\textrm{unassigned}}
\newcommand{\comments}[1]{}
\newcommand{\commento}[1]{\marginpar{\tiny \flushleft{#1}}}
\newtheorem{lemma}{Lemma}

\newpage

\chapter{Introduction}
\label{chap:introduction}

Decision Trees and Random Forests are among the most popular 
methods for classification tasks. Decision Trees, specially small ones, are easy to interpret,
while Random Forests usually yield more accurate classifications. One of the key issues in these methods
is how to select an attribute to associate with a node of the tree/forest. An important
related issue is how to split the samples once the attribute is selected.

There is a number of papers  discussing aspects related with 
attribute selection, such as:
how to design criteria to evaluate the quality of different types of attributes;
whether binary or multi-way splits shall be used and
how to remove bias from  splitting criteria.
For recent surveys on this topic we refer to \cite{books/sp/datamining2005/RokachM05},
\cite{Loh2014} and \cite{series/sbcs/BarrosCF15}.

Many criteria, with different properties,  have been proposed to evaluate 
the quality of different types of attributes, including
continuous and categorical ones.  Among the most popular criteria,
we have the Gini Gain and the Information Gain (\cite{Breiman84}, \cite{quinlan2014c4}).

Despite the large body of work we believe  there are still questions to be answered.
One of them is to how to  properly handle nominal  attributes that may assume a large number of values.
Before explaining the reason behind our  statement we would
like to remark that this kind of attribute
appears naturally in some applications  (e.g.: states of a country or letters from some alphabet).
In addition, they may arise as the result of aggregating
attributes that have few distinct values
with the goal of capturing possible correlation between them, as pointed out by \cite{Chou:91}.
As an example, consider 5 binary attributes (e.g. medical tests) and a
target binary variable  that has large probability of being positive if at least $3$ out
of the $5$ binary tests are positives. By aggregating
the $5$ binary variables we obtain a new attribute with $2^5=32$
values that  captures  this relation. 
If we used the 5 attributes separately we would 
need 5 levels in the tree to be able to capture the relation between
them and the target class, thus 
incurring a large fragmentation of the set of samples.

To properly face  multi-valued nominal attributes we have to deal with the computational time required to compute good splits.
Our contribution, explained in the next section, is related with this issue. 
 
A brute force search to compute the best binary split 
requires $\Omega(2^n)$ time, where $n$ is the number of distinct values the attribute may assume. The computational efficiency can be improved if a $n$-ary split is used rather than a binary one. However,  this may lead to a severe fragmentation of the sample space, which is not desirable: the number of samples available for each of the children of the split node 
may be small and, as a consequence, the underlying classification tasks may become significantly more difficult.
When the target variable is binary, a family of impurity measures that include both the Gini Gain and the Information Gain can be computed efficiently, as shown
in the influential monograph by Breiman et al \cite{Breiman84}.
However, when the number of classes $k$ is larger than 2,
most, if not all, of the available exact solutions take exponential time
in $(n,k)$.
The Twoing method, also from \cite{Breiman84}, 
is an  interesting case since its running time is $O(2^{\min\{n,k\}})$ rather than $O( 2^ n)$ while being equivalent to
Gini Gain when $k=2$.

When both $n$ and $k$ are large, in the sense that an exhaustive search does not run in a reasonable time, one can rely on heuristics to compute the best binary split.
As an example, the GUIDE algorithm  \cite{Loh2009}, the  last
of a series of algorithms/developments designed by Loh and its contributors, 
deals with a nominal variable $X$
as follows: if $k=2$ or $n \le 11$ the Gini Index is computed;
if $k \le 11$ and $n > 20$ a new variable $X'$ with at most $k$ distinct values is created according to a certain rule and an exhaustive search is performed over it;  finally, if $k > 11$ or $n \le 20$,  $X$ is binarized and a Linear Discriminant Analysis (LDA) is employed.
These rules reflect the difficulty in dealing with multi-valued nominal attributes. Other interesting heuristics are the PC and PC-ext criteria, which calculate the principal component of the class probability vectors and uses the order given by the vector projections in this direction to look for splits.
In general, the main drawback of using heuristics is the lack of a theoretical guarantee about their behavior. 


\section{Our Contribution}
\label{sec:contribution}

Given this scenario,  in chapter \ref{chap:framework} we propose
a framework for designing criteria, with nice theoretical properties, for evaluating the quality of 
 multi-valued nominal attributes. In general, finding the best binary partition according to an impurity measure has been proved to be NP-complete in \cite{icml2018}. Nonetheless, criteria generated according to our framework
run in polynomial time in both the number of values and classes and
have a theoretical guarantee that they are close to optimal.
The key idea consists of formulating the problem
of finding the best binary partition for a given attribute $A$ as the  problem of finding a 
cut with maximum weight  in a complete graph whose nodes are associated with the values that $A$ may assume and the edges' weights capture the benefit of putting
values in different partitions. The  motivation behind the use of the max-cut problem is 
the existence of efficient algorithms with 
approximation guarantee, in particular 
the one proposed  by \cite{GoeWil95}, with $0.878$ approximation,
and  local search  algorithms with 0.5-approximation as shown in \cite{journals/corr/AngelBPW16}.


We discuss two criteria that are derived from this framework:
the first one  can be seen as a natural variation of the
Gini Gain, while the second criterion uses the $\chi^2$-test  to set the edges' weights. For that, each
edge $e_{ij}$ between nodes  $v_i$ and $v_j$
is thought as a binary attribute $A(i,j)$ with values $v_i$ and $v_j$.
After discussing these criteria, we show how to extend them to handle
numeric attributes.

We also  present a number of experiments that suggest that one of our criteria is competitive with the Twoing method, which is -- as far as we know -- the only well-established criterion with binary splits that can be optimally computed for large $n$ when $k > 2$. However, in contrast with our methods, Twoing cannot handle datasets that also have a large number of classes. Some criteria based on heuristics, such as the PC-ext and Hypercube Cover, are also part of the comparison. In addition, the experiments also  provide evidence of the potential of aggregating  attributes for improving the accuracy of decision trees.

TODO: falar no paragrafo acima um pouco mais da comparacao com outros metodos.

\section{Related Work}
\label{chap:relatedwork}

Many splitting criteria have been proposed to 
deal with continuous and nominal attributes.
Arguably, the Gini Gain---used by CART---and entropy-based measures---such as 
the Information Gain, adopted by C4.5---are among
the most popular (\cite{books/sp/datamining2005/RokachM05,
Loh2014,series/sbcs/BarrosCF15}).

There has been  some investigation on 
methods to compute the best split efficiently 
(\cite{Breiman84,Chou:91,BPKN:92,journals/datamine/CoppersmithHH99}).
For the 2-class problem,  \cite{Breiman84} proved a theorem which states that an optimal
binary partition, for a certain class of splitting criteria,
can be determined in linear time on $n$, the number of distinct values of the attributes, after ordering.
The Gini  Gain belongs to this class.
The  other  three papers generalize
this theorem  in different directions 
and show necessary conditions that are satisfied by optimal partitions for a certain class of splitting criteria. 
These conditions, though useful to restrict the set of partitions
that need to be considered, do not yield  a method that
is efficient (polynomial time) for large values of $n$ and $k$. These papers also  present  heuristics, without approximation guarantee, to obtain good splits.
Another related result is a theorem from \cite{journals/datamine/CoppersmithHH99} that guarantees that the optimum split can be found by separating the class probability vectors by a hyperplane. This motivated the creation of the PC criterion, as will be shown in the next chapter. Two other existing heuristics are the SLIQ (\cite{mehta1996sliq} and the FlipFlop criterion (\cite{nadas1991iterative}). While they execute in polynomial time, they have no approximation guarantee in terms of the impurity of the optimal partition. Moreover, they usually finds worse partitions than the PC criterion, as shown in the experiments section of \cite{journals/datamine/CoppersmithHH99}. Lastly, \cite{icml2018} presents two heuristics that have approximation guarantees: the Hypercube Cover and the Largest Class Alone. The first has exponential running time on the number of classes and is similar to the Twoing criterion, but has an approximation guarantee of 2 for every impurity measure as defined in the paper. The second one, Largest Class Alone, runs in quasilinear time on the number of values and linear time on the number of classes. This criterion has an approximation guarantee of 2 for the Gini impurity and 3 for the Entropy impurity. 

Other proposals to  speed up the attribute selection phase
 include  \cite{MolaSiciliano1997,Shih2001}. 
The first presents a simple  heuristic
to reduce the number of binary splits considered to
choose the best nominal variable among the $m$ available ones.
 The second   extends the method for another class
of impurity measures.

In order to properly
handle nominal attributes with a large number of values,
apart from efficiently computing good splits, it is
important to prevent bias in the attribute selection.
Indeed, it is widely  known that many splitting criteria have bias toward
attributes with a large number of values. There are some  proposals available
to cope with this issue 
(\cite{conf/icml/DobraG01,Shih2004,Hothorn:2006:URP}). 
This topic, though relevant, is not the focus of this dissertation.

\section{Organization}
\label{sec:organization}
In Chapter \ref{chap:background} we explain how decision trees are used for classification problems and how they are constructed. We also present the main impurity measures and splitting criteria used in the literature, together with their execution-time complexity.

Chapter \ref{chap:framework} contains the framework for generating splitting criteria that run in polynomial time. Its relation with the Max-Cut problem and its approximation algorithms are explained and some criteria obtained from this framework are presented.

Later, in Chapters \ref{chap:experiments-splits} and \ref{chap:experiments-datasets}, we compare the proposed criteria and see how they perform in practice. In Chapter \ref{chap:experiments-splits} we explore how the many heuristics used to find splits with optimal impurity compare among themselves. This suggests a couple of criteria that perform better and can be used when the number of values and classes are large. In Chapter \ref{chap:experiments-datasets} we analyze these methods on real datasets that contain attributes with large number of values and classes. This analysis is done using different maximum depth allowed for the decision tree, which is allows us to see the the advantages and shortcomings of each criterion. Lastly, in Chapter \ref{chap:conclusions} we present our study conclusions.


\newpage

\chapter{Background}
\label{chap:background}

\section{Notation}
\label{sec:notation}
We adopt the following notation throughout the dissertation.
Let $S$ be a set of $N$ samples and 
 $C=\{c_1,\ldots,c_k\}$ be the domain of the class label. 
In addition, for an attribute  $A$, we use $A(s)$ to denote the value taken by attribute
$A$ on sample $s$; we use 
  $V=\{ v_1,\ldots,v_n \}$ to denote the set of values
taken by $A$;
$A_{ij}$ to refer to the  number of samples
from class $c_j$ for which  $A$ takes value $v_i$; 
 $N_i$ for the number of samples with value $v_i$ for attribute $A$
and $S_j$ for the number of samples from class $c_j$.
Furthermore, we let $p_j = S_j /N$ and $p_{ij}= Pr[C=c_j | A = v_i]$.
We observe that the estimator of maximum likelihood for $p_{ij} $ is
$A_{ij} / N_i$.  

\section{Impurity Measures}
Many of the splitting criteria follow the same algorithm:

TODO: algorithm
create_tree(S: set of samples, L: list of attributes information, I: impurity measure)
if S does not meet the stopping criterion:
    For each attribute:
        get the values\' split that yields the smallest impurity (measured by I)
    Split S using the attribute whose best values\' split has the smallest impurity
    Call create_tree recursively on each child node

Therefore a good place to start is by presenting the two most common impurity measures found in the literature.

TODO: falar de contingency tables aqui.

\subsection{Gini}
The Gini Index for a set of samples $S$ is given by 
\begin{equation}
 Gini(S) =  1- \sum_{i=1}^k (p_i)^2 .
\label{eq:gini}
\end{equation}

The Gini Index can be used to generate binary splits
and, as a consequence, binary decision trees.

The Gini Gain, $\Delta_G$, induced by a binary partition $(L,R)$ 
of the set of values $V$ is
given by 
\begin{equation}
 \Delta_G (L,R) = Gini(S) -
p_L Gini(S_L) - p_R Gini(S_R),
\label{eq:Ginigain}
\end{equation}
where $S_L= \{ s \in S | A(s) \in L \}$, $S_R= \{ s \in S | A(s) \in R \}$,
 $p_L=|S_L| /N $
and $p_R=|S_R| /N$. Therefore, the largest the Gini Gain is, the better the partition.

\subsection{Entropy}
The Entropy for a set of samples $S$ is given by 
\begin{equation}
 Entropy(S) =  - \sum_{i=1}^k p_i \log(p_i)
\label{eq:entropy}
\end{equation}

The Entropy can be used to generate binary splits
and, as a consequence, binary decision trees.

The Information Gain $IG$ induced by a binary partition $(L,R)$ 
of the set of values $V$ is given by 
\begin{equation}
 IG(L,R) = Entropy(S) -
p_L Entropy(S_L) - p_R Entropy(S_R),
\label{eq:InformationGain}
\end{equation}
where $S_L= \{ s \in S | A(s) \in L \}$, $S_R= \{ s \in S | A(s) \in R \}$,
 $p_L=|S_L| /N $ and $p_R=|S_R| /N$. Therefore, the largest the Information Gain is, the better the partition.


\section{Splitting Criteria}
In this section we recall some well-known splitting criteria.


\subsection{Gini Gain}
This criterion generates all $2^n$ binary values\' split and the partition with maximum Gini Gain shall be selected.
As shown in \cite{Breiman:84}, for the 2-class problem this optimal partition  
can be computed  in $O(n \log n + N )$ time: sort the values by the frequency of class $0$ on them. The best binary split of $S$ will be given by one of the values splits that follow this order (TODO: explicar melhor).

For problems with more than 2 classes, however, there is no efficient procedure with theoretical approximation guarantee to compute the Gini Gain in subexponential time in $n$.


\subsection{Twoing}
The Twoing criterion
for a  binary partition $(L,R)$ 
of the set of values $V$ is given by
$$ 0.25 \cdot p_L \cdot p_R  \cdot \left ( \sum_{i=1}^k | p_L^i - p_R^i | \right )^2, $$
where

$ p_L^i= \frac{|\{s \in S_L: s \mbox{ belongs to class } c_i \}|}{ |S_L|} $
 and 
$ p_R^i= \frac{|\{s \in S_R: s \mbox{ belongs to class } c_i\} |}{ |S_R|} $.

When the Twoing criterion is used to generate binary decision trees, the binary partition with maximum twoing shall be selected at each node. 

As shown in \cite{Breiman:84}, such partition can be calculated in $O(N + \min \{ n \log n 2^k, 2^n \} )$
time by considering all possibilities of partitioning the classes into two superclasses
and applying the Gini Gain criterion on each of them.
We shall remark that, for the $2$-class problem, the Twoing criterion and the
Gini Gain compute the same binary partitions.

\subsection{Information Gain}
This criterion works exactly the same as the Gini Gain, but replacing the Gini impurity by the Entropy. First it generates all $2^n$ binary values\' split and the partition with maximum Information Gain shall be selected.
For the 2-class problem, the same result valid for the Gini Gain works here, and the optimal partition  
can be computed  in $O(n \log n + N )$ time. Once again, when the number of classes is larger than 2 there is no efficient procedure with theoretical approximation guarantee to compute the Information Gain in subexponential time in $n$.

A related criterion is the Gain Ratio, where the Information Gain of an attribute is normalized by the potential information of that attribute. This is used as a way of decreasing the bias of the k-ary Information Gain criterion towards attributes with larger number of values. Since we are only interested in binary splits in this dissertation, we will not go into its details.


\subsection{$\chi^2$ criterion}
The $\chi^2$ is a popular criterion that was  used in \cite{Mingers.87}. It is also the first one shown here not based on impurity measures, and it only works for k-ary (instead of binary) splits. Is is mentioned here because of its relation to the framework presented in chapter \ref{chap:framework}.

For an attribute $A$ the $\chi^2$ criterion is given by 
\begin{equation}
\label{eq:chitest}
\sum_{i=1}^n \sum_{j=1}^k \frac{(A_{ij}-E[A_{ij}] )^2}{E[A_{ij}]},
\end{equation}
where $E[A_{ij}]=N_i p_j$.

\subsection{Conditional Inference Trees}
Conditional Inference Trees are actually a framework of creating criteria that are bias-free when it comes to the number of values in an attribute. It was published by \cite{Hawthorn} and still is the only known method of obtaining criteria that do not have any bias towards attributes with larger number of values.

It first chooses the best attribute to split at the current node and then evaluates all possible binary splits using any given impurity measure, choosing the best one found.

TODO: ver se notacao bate.

To choose the attribute in which to split, first one has to calculate the conditional expectation $\mu_j$ and covariance $\Sigma_j$ of the permutation test of all attributes $A_j$. Then, in order to compare the attributes, we need to calculate the p-value of a univariate test statistics $c_{quad}$ calculated on $\mu_j$ and $\Sigma_j$. The only exact form of doing this is by using the quadratic form (TODO: add equation and ref), which it follows an asymptotic $\chi^2$ distribution with degrees of freedom given by the rank of $\Sigma_j$. Since this involves the calculation of a pseudo-inverse, this is usually time consuming (the time complexity of the pseudo-inverse is cubic on the dimension of $\Sigma_j$, which is $n_j * k$).

TODO: colocar formulas

This method, although very complicated and somewhat slow, is used by the community very much. We mentioned it here because it will be used in our experiments in chapter \ref{chap:experiments-datasets}, since measuring the performance of any impurity measure in it is important in order to evaluate its usefulness. (TODO: melhorar texto desse paragrafo)




\section{Heuristics for Splitting Decision Tree Nodes}
As seen in the previous section, calculating the optimal split takes exponential time in the number of values or classes. Therefore many heuristics were created to create decision trees in this situation. The most used ones are listed below. All of them work with any impurity measure (e.g.: Gini or Entropy), but some of them work best with one of them. When this is the case, it will be mentioned.

\subsection{SLIQ and SLIQ-ext}
SLIQ was presented in (TODO: add citation) and it's a very simple greedy heuristic. Given an attribute, one starts with all the values going to the left split, and none on the right split. We then choose a value to go from the left to the right split. This value is the one that, when changing from the left to the right sides, decreases the impurity (increases the impurity gain) the most. This is repeated until there is no way of moving a value from the left to the right and decreasing the impurity.

SLIQ-ext is a simple extension, where we keep changing values from the left to the right until the left side is empty (that is, we move from the left to the right even if that increases the impurity). Once again the value to move is chosen in a greedy fashion. SLIQ-ext returns the values' split seen that had the lowest impurity.

\subsection{PC and PC-ext}
These heuristics are based on the Principal Component of the contingency table the and were presented in (TODO: add citation). One first calculates the class probability distribution of every value, which is done by normalizing the contingency table rows to have 1 in the sum norm. We then group values into ``supervalues'' where each value in the same supervalue has the same class probability distribution. Now we get the contingency table of these supervalues and calculate the first principal component of this matrix. One then calculates the inner product of each class probability vector of the supervalues  with the principal component and sort the supervalues by it. We then calculate the supervalues split's impurity gain given by splits of the form

TODO: write formula left, right.

Once we find the supervalues split with the largest impurity gain, we translate the supervalues into original values to obtain a valid partition.

PC-ext is a simple extension of this algorithm, where instead of only testing the supervalues splits given by (TODO: ex ref), we also test the split given by it and exchanging the last supervalue on the left with the first supervalue on the right (where first and last are given by the order after calculating the inner product).

\subsection{Largest Alone}
First one calculates the most frequent class and group the other classes in a single superclass. We then apply the Gini Gain criterion on this two-class problem. Since calculating the class frequencies can be done using the contingency table, this heuristic takes $O(N + n*k + n * \log(n))$ time in total.

Another advantage of the Largest Alone heuristic is that is has an approximation guarantee of 2 (TODO: conferir) to the best gini impurity of the original problem. It is also proved that there is no other way of grouping classes into superclasses that has a smaller approximation guarantee for the Gini impurity. It can also be used with the Information Gain, instead of Gini Gain, but its approximation guarantee increases to 3 (TODO: conferir). These bounds are proved in (TODO: ver como colocar referencia).

\subsection{List Scheduling}
First one calculates the frequency of every class. Then, we use a List Scheduling algorithm to group the classes into 2 superclasses as balanced as possible (in terms of number of samples). Lastly we apply the Information Gain criterion on this two-class problem. Again, since calculating the class frequencies can be done using the contingency table, this heuristic and the List Scheduling algorithm is linear in the number of classes, this heuristic also takes $O(N + n*k + n * \log(n))$ time in total.

Similarly to the Largest Alone heuristic, the List Scheduling heuristic has an approximation guarantee of 2 (TODO: conferir) to the best entropy impurity of the original problem. This is proved in (TODO: ver como colocar referencia). It is also proved that, for the entropy impurity, the best form of grouping classes into superclasses is by balancing them the best way possible.

\newpage

\chapter{Framework for Generating Splitting Criteria for Multi-valued Attributes}
\label{chap:framework}

First we recall some definitions and results for the Max-Cut problem. These definitions will be used in the following section, when we define our framework.

\section{The Maximum Weighted Cut Problem}
\label{sec:maxcutbackground}

We  recall some definitions from graph theory. A cut  $X$ in  a weighted graph $G=(V,E)$ is
a subset of vertexes of $V$. The weight of a cut $X$, denoted here by $w(X)$, is the sum of the weights
of the edges that have one endpoint in $X$ and the other one in $V-X$.

The problem of computing the  cut $X^*$ with maximum weight in a graph with non-negative weights is NP-Hard.
However, there are good  approximation algorithms
available. A remarkable one is the randomized algorithm
 proposed in  \cite{GoeWil95} that relies on a formulation of
the max-cut problem via semidefinite programming (SDP). This algorithm,  denoted here by GW, 
returns a cut $X$ that satisfies  $E[w(X)] \ge 0.878 w(X^*)$. It involves solving an SDP on the graph weights matrix, calculating the Cholesky decomposition of it and then generating a random partition of the values based on the inner product of the decompositions column vectors with a randomly generated vector on the sphere of dimension $n$. As solving an SDP takes $O(n^4)$ arithmetic operations (see \cite{navascues2009power}) and calculating the Cholesky decomposition takes $O(n^3)$ operations, the time complexity is high but polynomial.

Another possibility to solve the Max-Cut problem is by using the {\tt GreedyCut} algorithm, presented in Algorithm \ref{alg:greedy}. It
obtains a cut $X$ such that $w(X)  \ge 0.5 w(X^*)$ \cite{SahGon:76}.
The algorithm starts with two empty sets $X$ and $X'$. Then, it
scans the nodes 
and assigns each of them to the set that provides
the maximum improvement on  the weight of the current cut (ties are broken arbitrarily). Is it easy to see that the time complexity of this greedy algorithm is $O(n^2)$.


\begin{algorithm}[tb]
   \caption{ GreedyCut($V$: set of nodes)}
   \label{alg:greedy}
\begin{algorithmic}
\STATE{$X \leftarrow \emptyset$;$X' \leftarrow \emptyset$}
\FOR{$j=1,..,n$ }
\STATE{{\bf If} $$\sum_{v \in X} w(v_i,v) > \sum_{v \in X'-V} w(v_i,v) $$ add $v_i$ to $X'$ {\bf Else} add $v_i$ to $ X$  }
\ENDFOR
\STATE{{\bf Return} $X$ and $X'$ }

\end{algorithmic}
\end{algorithm}


The solutions obtained by both GW and {\tt GreedyCut} 
can be improved via a local search.
In its simplest version, it
moves a node from one group to
the other while some improvement on the cut weight is possible.
Although this algorithm is not polynomial  in the worst case,
it has polynomial behavior in the smoothed analysis framework
\cite{journals/corr/AngelBPW16}. In addition, it is always possible
to set a limit on the number of moves.
A more refined version  allows exchanging a pair of nodes
as long as the weight of the cut is improved.
In our experiments we use the version presented at 
Algorithm \ref{alg:localsearch}.


\begin{algorithm}[tb]
   \caption{ LocalSearch($X$, $X'$): set of nodes}
   \label{alg:localsearch}
\begin{algorithmic}
\STATE{$label:~loop$\_$start$}
\FOR{$i = 1, ..., n$ }
\IF{switching $v_i$'s side improves cut weight}
\STATE switch $v_i$ and update cut weight, $X$, $X'$
\STATE $goto~ loop$\_$start$
\ENDIF
\ENDFOR
\FOR{pair $(v_i, v_j) \in X \times X'$ }
\IF{switching $v_i$ and $v_j$ improves cut weight}
\STATE{switch $v_i$ and $v_j$, update cut weight, $X$, $X'$}
\STATE{$goto~ loop$\_$start$}
\ENDIF
\ENDFOR
\STATE{{\bf Return} $X$, $X'$}

\end{algorithmic}
\end{algorithm}


\section{A Framework for Generating Splitting Criteria}
\label{sec:maxcut}
In this section we explain 
our approach to building binary splitting criteria
for  multi-valued nominal attributes.

Let $A$ be a nominal attribute  that takes
values in the domain $V=\{v_1,\ldots,v_n\}$.
Our framework to produce a splitting criterion $I$ 
consists of three steps:

\begin{enumerate}
\item  Create a complete graph $G=(V,E)$ with $n$ vertexes.

\item  Assign a non-negative weight $w_{ij}$ to the edge 
that connects $v_i$ to $v_j$. This value shall reflect the benefit of putting  $v_i$ and $v_j$ in different partitions.
Different definitions of $w_{ij}$ yield to different criteria,
as we explain further.

\item  Ideally, the value of the criterion $I$ for attribute
$A$ is the weight of the cut with maximum weight  in $G$.
However, this is not a reasonable possibility for large $n$ since the problem of computing the  cut $X^*$ with maximum weight in a graph with non-negative weights is NP-Hard.
Thus, the value of criterion $I$ is given by the weight
of the cut obtained by some  algorithm, with approximation guarantee, for the maximum cut
problem in $G$.   
\end{enumerate}

What distinguishes the  criteria
generated  by our framework
is how the weights of the edges are set and the
method employed to compute the cut on graph $G$.
Here, we discuss two ways to set the weights:
the first one yields to criteria that
are related with the Gini Gain, while the second 
is built upon some given splitting  criterion that works well for binary attributes.


\subsection{The Squared Gini Criterion}
Here, we discuss how to set the weights so that we obtain 
a criterion that can be seen as a variation of the Gini Gain discussed in Section \ref{subsec:Gini}.

In fact, Lemma \ref{lem:GiniSq} below  show that  it is possible to define the
weights of the edges so that 
\begin{equation}
 \label{lem:squaredgini}
w(S_L)= Gini(S) - p^2_L \cdot Gini(S_L) - p^2_R \cdot Gini(S_R) 
\end{equation}
for every partition $(L,R)$ of  $V$. 

Note that the weight of the cut  $S_L$ in the above identity 
is similar to the expression for
the Gini Gain given by equation (\ref{eq:Ginigain}).
The difference is that $p_L$ and $p_R$ are replaced with
$p_L^2$ and $p_R^2$, respectively.
Because of the squares, this new criterion tends to favor more balanced partitions. 

For that the proof of Lemma \ref{lem:GiniSq}, recall that $A_{ix}$ is the number
of samples of  class $x$ 
that have value $v_i$, and  $C$ is used to denote the set of classes.


\begin{lemma}
For every $i,j$, with $i \ne j$ and $i,j \in \{1,\ldots,n\}$,  let
$$ w_{ij} = \frac{ 2 \sum_{ x,y \in C \atop x \ne y }  A_{ix} A_{jy} }{ N^2}.$$
Then, for every partition $(L,R)$ of  $V$ we have $w(S_L)=Gini(S) - p^2_L \cdot Gini(S_L) - p^2_R \cdot Gini(S_R)$. 
\label{lem:GiniSq}
\end{lemma}

\begin{proof}
Let $S_{x,L}$ and $S_{y,R}$  be the number of samples of classes $x$ and $y$ in groups $L$
and $R$, respectively. Moreover, let  $N_L$ and $N_R$ be 
the number of samples in $L$ and $R$, respectively.
It follows from equation (\ref{eq:gini}) that
$$N^2 Gini(S)=  N^2 -  \sum_{x=1}^k (S_{x,L} + S_{x,R})^2 $$
$$N_L^2 Gini(S_L)=  N_L^2 - \sum_{x=1}^k S_{x,L}^2 $$
and
$$N_R^2 Gini(S_R)= N_R^2 - \sum_{x=1}^k S_{x,L}^2. $$ 
Since $N=(N_L+N_R)$ it follows that 
$$N^2 Gini(S) - N_L^2 Gini(S_L)  - N_R^2 Gini(S_R) =$$
$$ 2 N_L N_R - 2 \sum_{x \in C} S_{x,L} S_{x,R} = $$
$$ 2  \sum_{x \in C } S_{x,L} \sum_{x \in C} S_{x,R}  - 2 \sum_{x \in C} S_{x,L} S_{x,R} =$$  
$$ 2  \sum_{ x \neq y \atop x,y \in C } S_{x,L} S_{y,R} =  2 \sum_{ x \neq y \atop x,y \in C } \left ( \sum_{ i \in L   }\sum_{ j \in R   }  A_{ix}A_{jy} \right ) =$$
$$ N^2 \sum_{i \in L  } \sum_{j \in R  } w_{ij} =  N^2 w(S_L) $$
Dividing the first term and the last term by $N^2$ in the  above expression and, using
 $N_L=p_L \cdot N$ and $N_R=p_R \cdot N$,
we  establish 
the lemma.
\end{proof}


It is worth mentioning that symmetric 
misclassification costs can be easily introduced in this case.
In fact, let $mix(x,y)$  be the cost 	
of  mixing  samples from classes $x$ and $y$.
We can define 
$$ w_{ij} =   \sum_{ x,y \in C \atop x \ne y } mix(x,y)  p_{ix} p_{jy} .$$
This measure favors the separation of the classes
that incur  a large cost in the case they are mixed.


\subsection{Setting weights according to other splitting criteria}

Our second way of defining the weights makes use of some 
given splitting criterion 
for  binary nominal attributes. 
Such  criterion is used to measure the quality of separating samples with value $v_i$ from those with value $v_j$,
for each $i$ and $j$, and, thus, defining the edges' weights.
Here, we investigate the criterion obtained by 
defining $w_{ij}$ as the value of the $\chi^2$-test
for the attribute $A$ when  evaluated over the restricted dataset 
that contains only the samples of $S$ with values $v_i$ and $v_j$.
In formulae,
$$w_{ij}=  \sum_{\ell=1}^k \frac{(A_{i \ell}-E[A_{i \ell}] )^2}{E[A_{ i \ell}]}
+ \sum_{\ell=1}^k \frac{(A_{j \ell}-E[A_{j \ell}] )^2}{E[A_{ j \ell}]}
$$
where $E[A_{i \ell }]=N_i p_{\ell}$
and $E[A_{j \ell }]=N_j p_{\ell} $.

In order to reduce the bias towards attributes with many values, we divide 
$w_{ij}$ by  $n-1$, for every pair $i,j$. We make this adjustment  because each value contributes
to the weight of $n-1$ edges.

We shall remark that, although not explored in this work,
other criteria such as Information Gain or Gini Index
could be used, instead of $\chi^2$ test, to
set the weights.


\subsection{Handling Numeric Attributes}
We observe  that criteria from our framework can handle
 a numeric attribute $A$ with $t$ distinct values
$v_1,\ldots,v_t$
by considering it as collection of 
$t-1$ binary attributes, where the
$j$-th attribute, $A^j$,  splits the samples into the
groups $\{s | A(s) \le v_j \}$ and $\{s | A(s) > v_j \}$. 
The split  obtained by criterion $I$ on a numeric attribute
$A$ matches the split of  the best attribute 
among $A^1,\ldots,A^ {t-1}$, according to $I$.

\newpage

\chapter{Experiments on Splits with Reduced Impurity}
\label{chap:experiments-splits}

In this chapter we compare the ability of different heuristics in finding the values' split with lowest impurity. We are interested in choosing what heuristic/criterion to use when the exact ones don't run in reasonable time. In the next chapter the performance of the best heuristics will be compared on real datasets against Twoing and criteria generated from our framework.

Our experiments are very similar to those proposed in \cite{journals/datamine/CoppersmithHH99} except for a few details. All experiments are Monte Carlo simulations with 10,000 runs, each using a randomly-generated contingency table for the given number of values $n$ and classes $k$. Each table  was created by uniformly picking a number in $\{0, \ldots, 7\}$ for every entry. This guarantees a substantial probability of a row/column having some zero frequencies, which is common in practice. Differing from  \cite{journals/datamine/CoppersmithHH99}, if all the entries corresponding to a value or a class are zero, we re-generate the contingency table, otherwise the number of actual values and classes would not match $n$ and $k$. We evaluated Hypercube Cover, PC-ext,
%SLIQ-ext, 
Largest Class Alone and List Scheduling for both the Gini and Entropy impurities. Note that we don't evaluate Twoing because the split selected by Hypercube Cover is always purer by construction. Moreover, it is also more natural, since we are measuring the performance of these criteria on the k-class impurity.

Tables \ref{tab:Splits-Gini} and \ref{tab:Splits-Entropy} show, for different values of $n$ and $k$, the percentage of times that
each criterion found the best Gini and Entropy impurities, respectively. Note that the percentages do not necessarily sum exactly $100\%$ since
there were ties. In these tables we only show results for $k \leq 9$ because for larger values
of $k$ Hypercube Cover becomes non-practical due to its running time. In addition, we do not present results for small values of $n$ because in this
case the optimal  partition can be found quickly by testing all possible partitions, so that there is no motivation for heuristics.

\begin{table}
\centering
\begin{tabular}{c|c|c|c|c|c} 
        n            &    k        &   HcC   &   PC-ext   &   LCA   &   LS   \\
\hline
\multirow{4}{*}{12}  &    3        &   97.3  &   91.2     &   42.8  &  42.8  \\
                     &    5        &   99.2  &   88.0     &   19.1  &  17.8  \\
                     &    7        &   99.9  &   86.6     &   11.5  &  10.8  \\
                     &    9        &   100   &   85.0     &    8.5  &   8.4  \\
\hline
\multirow{4}{*}{25}  &    3        &   73.9  &   72.7     &   24.3  &  24.3  \\
                     &    5        &   65.8  &   62.4     &    5.9  &   4.0  \\
                     &    7        &   73.3  &   58.8     &    2.0  &   1.7  \\
                     &    9        &   85.3  &   53.2     &    0.8  &   0.9  \\
\hline
\multirow{4}{*}{50}  &    3        &   51.4  &   50.6     &   16.0  &  16.0  \\
                     &    5        &   33.1  &   41.1     &    3.3  &   1.3  \\
                     &    7        &   31.0  &   40.7     &    1.0  &   0.4  \\
                     &    9        &   33.9  &   37.8     &    0.4  &   0.1
\end{tabular}
\caption{Percentage of times each criterion finds the smallest Gini impurity, compared among themselves.}
\label{tab:Splits-Gini}
\end{table}

% \begin{table}
% \tiny 
% \centering
% \begin{tabularx}{\textwidth}{c|xxxx|xxxx|xxxx} 
% \# values                       & \multicolumn{4}{c|}{12}        &  \multicolumn{4}{c|}{25}        & \multicolumn{4}{c}{50}  \\ 
% \diagbox{Criterion}{\# classes} &   3   &   5   &   7   &   9    &   3   &   5   &   7   &   9     &   3   &   5   &   7   &   9    \\
% \hline
% Hypercube Cover                 & 97.3  & 99.2  & 99.9  & 100    & 73.9  & 65.8  & 73.3  & 85.3    & 51.4  & 33.1  & 31.0  & 33.9   \\
% PC-ext                          & 91.2  & 88.0  & 86.6  & 85.0   & 72.7  & 62.4  & 58.8  & 53.2    & 50.6  & 41.1  & 40.7  & 37.8   \\
% %SLIQ-ext                        & 89.9  & 81.9  & 78.3  & 75.5   & 78.8  & 64.6  & 57.9  & 52.5    & 68.1  & 53.1  & 47.1  & 42.9   \\
% LargestClassAlone               & 42.8  & 19.1  & 11.5  &  8.5   & 24.3  &  5.9  &  2.0  &  0.8    & 16.0  &  3.3  &  1.0  &  0.4   \\
% ListScheduling                  & 42.8  & 17.8  & 10.8  &  8.4   & 24.3  &  4.0  &  1.7  &  0.9    & 16.0  &  1.3  &  0.4  &  0.1 
% \end{tabularx}
% \normalsize
% \caption{Percentage of times each criterion finds the smallest Gini impurity, compared among themselves.}
% \label{tab:Splits-Gini}
% \end{table}


\begin{table}
\centering
\begin{tabular}{c|c|c|c|c|c} 
        n            &    k        &   HcC   &   PC-ext   &   LCA   &   LS   \\
\hline
\multirow{4}{*}{12}  &    3        &   98.3  &   80.2     &   33.5  &  33.5  \\
                     &    5        &   99.4  &   74.2     &   13.6  &  15.3  \\
                     &    7        &   100   &   73.2     &    8.3  &  10.1  \\
                     &    9        &   100   &   72.4     &    6.8  &   8.0  \\
\hline
\multirow{4}{*}{25}  &    3        &   83.3  &   54.4     &   18.5  &  18.5  \\
                     &    5        &   76.9  &   42.7     &    5.3  &   2.6  \\
                     &    7        &   81.0  &   39.2     &    2.0  &   1.2  \\
                     &    9        &   87.7  &   37.7     &    1.3  &   0.8  \\
\hline
\multirow{4}{*}{50}  &    3        &   70.0  &   29.5     &   13.4  &  13.4  \\
                     &    5        &   57.4  &   22.0     &    3.7  &   0.6  \\
                     &    7        &   53.5  &   21.7     &    1.6  &   0.1  \\
                     &    9        &   52.5  &   22.1     &    0.8  &   0.1
\end{tabular}
\caption{Percentage of times each criterion finds the smallest Entropy impurity, compared among themselves.}
\label{tab:Splits-Entropy}
\end{table}


% \begin{table}
% \tiny 
% \centering
% \begin{tabularx}{\textwidth}{c|xxxx|xxxx|xxxx} 
% \# values                       & \multicolumn{4}{c|}{12}        &  \multicolumn{4}{c|}{25}        & \multicolumn{4}{c}{50}  \\ 
% \diagbox{Criterion}{\# classes} &   3   &   5   &   7   &   9    &   3   &   5   &   7   &   9     &   3   &   5   &   7   &   9    \\
% \hline
% Hypercube Cover                 & 98.3  & 99.4  & 100   & 100    & 83.3  & 76.9  & 81.0  & 87.7    & 70.0  & 57.4  & 53.5  & 52.5   \\
% PC-ext                          & 80.2  & 74.2  & 73.2  & 72.4   & 54.4  & 42.7  & 39.2  & 37.7    & 29.5  & 22.0  & 21.7  & 22.1   \\
% %SLIQ-ext                        & 87.5  & 78.2  & 75.2  & 72.8   & 71.8  & 57.1  & 52.1  & 47.1    & 55.1  & 42.5  & 38.8  & 36.2   \\
% LargestClassAlone               & 33.5  & 13.6  &  8.3  &  6.8   & 18.5  &  5.3  &  2.0  &  1.3    & 13.4  &  3.7  &  1.6  &  0.8   \\
% ListScheduling                  & 33.5  & 15.3  & 10.1  &  8.0   & 18.5  &  2.6  &  1.2  &  0.8    & 13.4  &  0.6  &  0.1  &  0.1 
% \end{tabularx}
% \normalsize
% \caption{Percentage of times each criterion finds the smallest Entropy impurity, compared among themselves.}
% \label{tab:Splits-Entropy}
% \end{table}

In general, we observe an advantage for Hypercube Cover with both impurities, being more clear for the Entropy, followed by PC-ext. This suggests that the Largest Class Alone and List Scheduling heuristics are not competitive with them in terms of split impurity found.

Another possible comparison between them is to see what happens when Hypercube Cover and PC-ext find different partitions. To measure this difference, let us define the relative excess (in percentage) of a partition P w.r.t. a partition Q as $100 \times (I(P)/I(Q) - 1)$. The results of this measurement are shown in Tables \ref{tab:Relative-Excess-Gini} and \ref{tab:Relative-Excess-Entropy}. For the Gini impurity (Table \ref{tab:Relative-Excess-Gini}), we can observe that Hypercube Cover finds partitions closer to the optimal than PC-ext. This behavior is even stronger for the Entropy impurity (Table \ref{tab:Relative-Excess-Entropy}), where both the average and maximum relative excess of Hypercube Cover over PC-ext is much smaller than PC-ext's over Hypercube Cover. These numbers suggest that the risk of finding a ``bad'' partition is smaller when Hypercube Cover is used, specially for the Entropy impurity.

\begin{table}
\centering
\begin{tabular}{c|c|c|c|c|c} 
\multirow{2}{*}{n}   & \multirow{2}{*}{k}  &   \multicolumn{2}{c|}{HcC excess over PC-ext} &  \multicolumn{2}{c}{PC-ext excess over HcC}   \\
                     &                     &   Average           &    Max                  &   Average           &    Max                  \\
\hline
\multirow{4}{*}{12}  &    3                &   0.15              &   0.97                  &   0.37              &  2.36                   \\
                     &    5                &   0.06              &   0.22                  &   0.14              &  0.98                   \\
                     &    7                &   0.02              &   0.05                  &   0.08              &  0.49                   \\
                     &    9                &   ---               &   ---                   &   0.05              &  0.36                   \\
\hline
\multirow{4}{*}{25}  &    3                &   0.14              &   0.83                  &   0.24              &  1.72                   \\
                     &    5                &   0.05              &   0.29                  &   0.1               &  0.84                   \\
                     &    7                &   0.32              &   1.84                  &   0.32              &  1.62                   \\
                     &    9                &   0.19              &   1.06                  &   0.2               &  1.16                   \\
\hline
\multirow{4}{*}{50}  &    3                &   1.35              &   7.27                  &   1.37              &  8.66                   \\
                     &    5                &   0.39              &   2.02                  &   0.38              &  2.1                    \\
                     &    7                &   0.19              &   0.95                  &   0.18              &  1.11                   \\
                     &    9                &   0.11              &   0.62                  &   0.11              &  0.53
\end{tabular}
\caption{Relative excess impurity, in percentage, for experiments where Hypercube Cover and PC-ext found different partitions using the Gini impurity.}
\label{tab:Relative-Excess-Gini}
\end{table}


\begin{table}
\centering
\begin{tabular}{c|c|c|c|c|c} 
\multirow{2}{*}{n}   & \multirow{2}{*}{k}  &   \multicolumn{2}{c|}{HcC excess over PC-ext} &  \multicolumn{2}{c}{PC-ext excess over HcC}   \\
                     &                     &   Average           &    Max                  &   Average           &    Max                  \\
\hline
\multirow{4}{*}{12}  &    3                &   0.17              &   1.07                  &   0.77              &  6.59                   \\
                     &    5                &   0.09              &   0.39                  &   0.37              &  2.64                   \\
                     &    7                &   0.04              &   0.04                  &   0.24              &  2.11                   \\
                     &    9                &   ---               &   ---                   &   0.18              &  1.42                   \\
\hline
\multirow{4}{*}{25}  &    3                &   0.14              &   0.81                  &   0.5               &  4.26                   \\
                     &    5                &   0.08              &   0.49                  &   0.26              &  2.02                   \\
                     &    7                &   0.53              &   3.23                  &   0.58              &  3.12                   \\
                     &    9                &   0.38              &   2.11                  &   0.42              &  2.31                   \\
\hline
\multirow{4}{*}{50}  &    3                &   1.33              &   7.87                  &   1.46              &  7.31                   \\
                     &    5                &   0.5               &   2.69                  &   0.58              &  3.07                   \\
                     &    7                &   0.29              &   1.5                   &   0.35              &  1.85                   \\
                     &    9                &   0.21              &   1.2                   &   0.25              &  1.25
\end{tabular}
\caption{Relative excess impurity, in percentage, for experiments where Hypercube Cover and PC-ext found different partitions using the Entropy impurity.}
\label{tab:Relative-Excess-Entropy}
\end{table}


Lastly, we note that, due to Largest Class Alone and List Scheduling running times, they might be used when both $n$ and $k$ are very large and speed is an  issue. When $n=200$ and $k=100$, using a single core, they are almost 50 times faster than PC-ext, with the latter using 8 cores. In addition, they could be  used together with PC-ext, incurring a negligible overhead, to guarantee that the ratio between the impurity of the partition found and the optimal one is bounded.

 
Taking into account these  experiments, those reported in \cite{journals/datamine/CoppersmithHH99} and the  theoretical properties of the available algorithms, Table \ref{tab:guidelines}  suggests  guidelines on which criterion to use to solve the problem of finding the binary partition of minimum impurity in practical situation. Of course small, medium and large depend on the available hardware and the time one accepts to wait for training/testing classification models. In the next chapter we will analyze if this behavior is consistent with what we see in practice.


\begin{table}[htb]
\centering
\begin{tabular}{c|c|c}
{\bf n}    & {\bf k}   & {\bf Suggested Method} \\ \hline 
small      & any       &  Exact \\
not small  & small     &  Hypercube Cover \\
not small  & not small &  PC-ext \\
\end{tabular}
\caption{Guidelines on how to solve the problem of finding the partition with minimum impurity in practice.}
\label{tab:guidelines}
\end{table}


\newpage

\chapter{Experiments on Real Datasets}
\label{chap:experimentsdatasets}


In this chapter we describe our
experimental study on real datasets.
First, we describe the chosen datasets.
Next, we discuss the max-cut algorithms 
employed and, then, we
present our results.

All  experiments described in the following sections were executed on a machine with the following settings: Intel(R) Core(TM) i7-4790 CPU @ 3.60GHz with 32 GB of RAM. The code was developed using Python 3.6.1 with the libraries numpy, scipy, scikit-learn and cvxpy.
The project can be accessed in {\tt github.com/felipeamp/max\_cut\_paper}. It includes  the code, the datasets and the results of our experiments.


\section{Datasets}
We employed 11 datasets in total. Eight of them are from the UCI repository:
Mushroom, KDD98, Adult, Nursery, Covertype, Cars, Contraceptive and Poker  \cite{Lichman:2013}.
Two others are available in Kaggle: San Francisco Crime and Shelter Animal Outcome
\cite{SFC,AnimalShelter}. The last dataset was created by translating texts from the Reuters database \cite{Lichman:2013} into phonemes, using the CMU pronouncing dictionary \cite{CMU-PD}.

We chose  these datasets  because they
have at least 1000 samples and they  either contain  multi-valued attributes 
or attributes that can be naturally aggregated to produce multi-valued attributes. 
From the KDD98 dataset we derived the datasets
KDD98-k, for $k = 2$ and $9$. These datasets contain
only the positive samples (people that donate money) 
of KDD98 and the target attribute, Target$\_$D, is split into $k$ classes, where the $i$-th
class correspond to the $i$-th quantile in terms of amount of money donated. For the Reuters Phonemes dataset,
we extracted 10000 samples containing the 15 most common phonemes as class and try to predict when they are about to happen given the 3 preceding phonemes.
This dataset is motivated by Spoken Language Recognition problems, where phonotactic models are used as an important part of the classification system
\cite{conf/interspeech/Navratil06}. 
For the San Francisco Crime dataset, we give the month, day of the week, police department district and latitute/longitude and try to predict the crime category. Lastly, for the Shelter Animal Outcomes dataset, we converted the age into a numeric field containing the
number of days old and separated the breed into two categorical fields, repeating the breed in both in case there was only one originally. We also removed the AnimalID, Name and the DateTime. For this dataset we try to predict the outcome type and subtype (concatenated into a single categorical field). For both 
San Francisco Crime and  Shelter Animal Outcomes
 datasets we created a version of them ({\tt S.F. Crime-15} and {\tt Shelter-15}), containing only 15 classes, instead of the 39 and 22 original ones, respectively. This was done by grouping the rarest classes into a single one. 

\begin{table}
\centering
\caption{Information about  the employed datasets after data cleaning and attributes aggregation.
Column $k$ is the number of classes and {\tt Reg} stands
for Regression; columns $m_{nom}$ and $m^{ext}_{nom}$ are the
number of  nominal attributes in the original and the
extended datasets (when it exists), respectively; column $m_{num}$ is the number of  numeric attributes.}
\label{exp:datasets}

\medskip

\begin{tabular}{c|c|c|c|c|c}
Dataset             & Samples  &  k        & $m_{nom}$ &  $m^{ext}_{nom}$ &   $m_{num}$   \\  \hline
{\tt Mushroom}      & 5644     & 2         & 22        & N/A              & 0             \\ 
{\tt Adult}         & 30162    & 2         & 8         & N/A              & 6             \\
{\tt KDD98}         & 4843     & {\tt Reg} & 65        & N/A              & 314           \\ 
{\tt Nursery}       & 12960    & 5         & 8         & 11               & 0             \\ 
{\tt CoverType}     & 581012   & 7         & 44        & 46               & 10            \\ 
{\tt Car}           & 1728     & 4         & 6         & 8                & 0             \\ 
{\tt Contracep}     & 1473     & 3         & 7         & 9                & 2             \\ 
{\tt Poker}         & 25010    & 10        & 10        & 0                & 0             \\
%{\tt Shelter-15}    & 26711    & 15        & 5         & N/A              & 1             \\
{\tt Shelter}       & 26711    & 22        & 5         & N/A              & 1             \\      
%{\tt S.F. Crime-15} & 878049   & 15        & 3         & N/A              & 2             \\      
{\tt S.F. Crime}    & 878049   & 39        & 3         & N/A              & 2             \\  
{\tt Phonemes}      & 10000    & 15        & 3         & N/A              & 0 
\normalsize
\end{tabular}

\end{table}



We also created extended versions
of some of the above datasets 
by adding nominal attributes  obtained by aggregating some of the original ones, as 
we detail below.  
Our goals are  examining the impact of multi-valued
attributes in the classification performance and 
also understanding how the  different 
splitting criteria  handle them.


\begin{table}[]
\centering
\begin{tabular}{c|c|c} 
{\tt parents}     & {\tt has\_nurs}    & {\tt Aggregated Attribute}   \\ \hline
usual       & proper       & usual-proper             \\
usual       & less\_proper & usual-less\_proper       \\
usual       & improper     & usual-improper           \\
usual       & critical     & usual-critical           \\
usual       & very\_crit   & usual-very\_crit         \\
pretentious & proper       & pretentious-proper       \\
pretentious & less\_proper & pretentious-less\_proper \\
pretentious & improper     & pretentious-improper     \\
pretentious & critical     & pretentious-critical     \\
pretentious & very\_crit   & pretentious-very\_crit   \\
great\_pret & proper       & great\_pret-proper       \\
great\_pret & less\_proper & great\_pret-less\_proper \\
great\_pret & improper     & great\_pret-improper     \\
great\_pret & critical     & great\_pret-critical     \\
great\_pret & very\_crit   & great\_pret-very\_crit  
\end{tabular}
\caption{Aggregation of attributes parents and has\_nurse from dataset {\tt Nursery}}
\label{tab:Aggreation}

\end{table}

Table \ref{tab:Aggreation} illustrates this construction.

\begin{itemize}

\item {\tt Nursery-Ext}. This dataset is obtained by adding three
new attributes to dataset Nursery. 
The first attribute has  $15$ distinct values
and it is constructed through the  aggregation of   2 attributes 
from group  EMPLOY, one with 5 values and the other with 3 values. 
The second attribute has 72 distinct values
corresponding to the aggregation of attributes from 
the attributes in group {\tt STRUCT\_FINAN}.
The third attribute, with 9 distinct values, is the combination of
the  attributes in group {\tt SOC\_HEALTH}.


\item {\tt Covertype-Ext}. 
We combined 40 binary attributes
related with the soil type  into a new attribute with 40 distinct values.
The same approach was employed to combine the 4 binary attributes related
with the wilderness area into a new attribute
with 4 distinct values.
This is an interesting case because, apparently, the 40 (soil type)
binary attributes  as well as the 4 (wilderness area) binary attributes
 were derived from a binarization of two attributes, one with 40 distinct value and the other with 4 distinct values.
 

\item {\tt Cars-Ext}. To obtain this dataset, the 2 attributes
related with the concept {\tt PRICE},  {\tt buying} and {\tt maint},
were combined into an attribute
with 16 distinct values.
Moreover, the 3 attributes
related with concept {\tt CONFORT} were  combined into an 
an attribute with 36 distinct values.

\item {\tt Contraceptive-Ext}. The 2 attributes
related with the couple's education were combined into an
attribute with 16 distinct values.
Moreover, the 3 attributes related with the couple's occupations  and  standard of living
were  aggregated into a new attribute with 32 distinct values.

\end{itemize}

Samples with missing values were removed from the datasets.
Table \ref{exp:datasets} provides some statistics.




\section{Computing the Maximum Cut}
 

The GW algorithm  requires  the solution of a semidefinite program (SDP),
which may be computationally expensive despite  its 
polynomial time worst case  behavior.
As an example, for an attribute with 100 distinct values, the solution of
the corresponding SDP takes in average about 2 second in our machine.
One the one hand, this is a tiny amount of time 
compared with that required to perform an exhaustive search on the $2^{100}$
possible binary partitions. On the other hand, 
 faster alternatives are desirable, even
at the cost of losing part of the theoretical approximation guarantee. 

To avoid solving a SDP, 
we also evaluated a procedure
that first executes the {\tt GreedyCut} algorithm presented
in Section \ref{sec:maxcutbackground} and then runs a local search as described
in the Algorithm  \ref{alg:localsearch} in the same section. 
The use of this approach combined with the two
ways of setting the edges' weights lead to
 Greedy LocalSearch SquaredGini (GLSG) and 
Greedy LocalSearch $\chi^2$ (GL$\chi^2$) 
criteria, respectively.
For attributes with 100 distinct values this approach is 60-70
times faster than the one based on the GW algorithm.


\section{Experimental Results}

We performed a number of experiments to evaluate how the
proposed methods behave with real datasets.
All experiments consist of building decision trees
with a predefined maximum depth.  In addition,
to prevent the selection of non-informative nominal attributes, 
we used a $\chi^2$-test for each attribute at every node of
the tree: if the $\chi^2$-test  on the contingency table of attribute $A$
has $p$-value larger than $10\%$ at a node $\nu$, then
$A$ is not used in $\nu$. Furthermore, attributes with less than 15 samples associated with its
second most frequent value are also not considered for splitting. This helps  avoid data overfitting.


\subsection{Maximum Depth 5}

TODO: adicionar resultados pro PC-ext e possivelmente pra uma heuristica.

Table \ref{tab:CrossVal}.(a) presents  the results of an experiment to
compare the accuracy of  Decision Trees built by  our methods with those built by Twoing.
In this experiment, the maximum depth was set to 5 and we considered just the nominal attributes of the datasets. 
The motivation for this depth is to produce trees that
are relatively easy to interpret and, in addition, some random
forests methods (such as boosting) work with shallow trees.
Each accuracy is the average of 20 stratified 3-fold cross-validations,
each generated with a different seed.
The entry  associated with  $({\cal D},I)$ has two pieces of information: the average accuracy
of criterion $I$ on dataset ${\cal D}$ and the number of criteria
with accuracy   statistically lower than that of $I$ on dataset ${\cal D}$. 
The statistical test used for criteria comparison is a  one-tailed paired $t$-student test with a $95\% $ confidence level. 
In general, there was a balance among the different methods,
except for GWSG, which had slightly inferior accuracies. Another interesting observation is that the GW-based criteria were
about equal or slightly inferior to their GL-based counterparts. This advantage is likely related with the fact that the weights of
the cuts computed by the GL approach in this experiment are, in general, larger than those obtained by the GW algorithm.


\begin{table*}[t]
\small
\centering
\caption{Average accuracy and statistical tests  for  decision trees 
with depth at most 5 using only nominal attributes. The best accuracy for each dataset is bold-faced.}
\begin{tabular}{c|cc|cc|cc|cc|cc|cc} 
Dataset & \multicolumn{2}{c|}{Twoing} &  \multicolumn{2}{c|}{GWSG}  
&   \multicolumn{2}{c|}{GW$\chi^2$}                   &\multicolumn{2}{c|}{GLSG}       &\multicolumn{2}{c|}{GL$\chi^2$} & \multicolumn{2}{c}{PC-ext}\\  \hline 
% Dataset           &        Twoing     &     GWSG          &      GW$\chi^2$   &       GLSG        &    GL$\chi^2$     &     PC-ext
{\tt Adult}         & 82.21    &{\bf(2)}& 81.91    & (1)    &{\bf82.24}&{\bf(2)}& 81.83    & (0)    & 82.24    & (2)    &{\bf82.31}&        \\
{\tt Mushroom}      & {\bf 100}&{\bf(2)}& 99.99    & (0)    & 99.98    & (0)    &{\bf  100}&{\bf(2)}& 99.99    & (0)    &          &        \\
{\tt KDD98}-2       & 80.47    & (0)    & 81       & (2)    & 80.74    & (1)    &{\bf81.16}&{\bf(3)}& 80.51    & (0)    &          &        \\
%{\tt KDD98}-3       & 63.77    & (0)    & 63.8     & (0)    & 64.37    &{\bf(3)}& 63.75    & (0)    &{\bf64.54}&{\bf(3)}&          &        \\
%{\tt KDD98}-5       & 48.02    & (2)    & 46.79    & (0)    &{\bf48.59}&{\bf(3)}& 46.77    & (0)    & 48.58    &{\bf(3)}&          &        \\
{\tt KDD98}-9       & 40.35    & (2)    & 38.13    & (0)    & 40.93    &{\bf(3)}& 38.15    & (0)    &{\bf 41 } &{\bf(3)}&          &        \\
{\tt Nursery}       & 88.25    &{\bf(3)}& 88.03    & (0)    & 88.2     & (0)    &{\bf88.33}&{\bf(3)}& 88.2     & (0)    &          &        \\
{\tt Nursery-Ext}   &{\bf93.82}&{\bf(4)}& 90.95    & (0)    & 93.02    & (2)    & 90.75    & (0)    & 93.13    & (2)    &          &        \\
{\tt Cars}          & 86.53    & (1)    & 85.39    & (0)    & 86.37    & (1)    &{\bf87.93}&{\bf(4)}& 86.42    & (1)    & 86.5     &        \\
{\tt Cars-Ext}      & 90.3     & (0)    & 90.82    & (1)    & 91.57    & (3)    & 90.84    & (1)    &{\bf 91.9}&{\bf(4)}& 90.32    &        \\
{\tt Contracep}     & 43.77    & (0)    & 43.96    & (0)    &{\bf44.04}& (0)    & 43.89    & (0)    & 44       & (0)    & 43.59    &        \\
{\tt Contracep-Ext} & 43.17    & (0)    & 44.32    &{\bf(3)}& 43.44    & (0)    &{\bf44.35}&{\bf(3)}& 43.7     & (0)    & 43.77    &        \\
{\tt CoverType}     & 52.97    & (2)    &{\bf55.05}&{\bf(3)}& 51.07    & (0)    &{\bf55.05}&{\bf(3)}& 51.07    & (0)    &          &        \\
{\tt CoverType-Ext} &{\bf64.48}&{\bf(4)}& 64.12    & (2)    & 57.73    & (0)    & 64.23    & (3)    & 59.95    & (1)    &          &        \\ 
{\tt Poker }        & 51.9     & (2)    & 50.28    & (1)    & 51.77    & (2)    & 49.94    & (0)    &{\bf51.91}&{\bf(3)}&          &        \\ 
{\tt Shelter-15}    & 48.01    &        &          &        &          &        & 45.31    &        & 48.13    &        &          &        \\   
{\tt S.F. Crime-15} &{\bf22.1} &        &          &        &          &        & 21.23    &        & 22.09    &        &          &        \\ 
{\tt Phonemes}      &{\bf30.92}&        &          &        &          &        & 30.29    &        & 29.47    &        &          &        \\
\hline
Average (Sum)       &     63.7 &        &          &        &          &        & 63.33    &        & 63.36    &        &          & 

\end{tabular}
\label{exp:thirdset}
\normalsize
\end{table*}


The results of  Table \ref{tab:CrossVal}.(a) also
provide evidence of  the potential
of considering aggregated attributes. 
The accuracy obtained for the extended versions of datasets
{\tt Nursery}, {\tt Cars} and {\tt CoverType} are considerably higher than those obtained for 
the original versions. For {\tt Contracep}, the effect is not clear.

Another key aspect to discuss is the computational cost of the
proposed criteria. Table \ref{tab:time-depth5} shows the running time of each criterion in the experiment of Table \ref{tab:CrossVal}.(a). Twoing is the fastest method when the number of classes is small and the GL-based methods become competitive and eventually the
fastest ones when the number of classes gets larger, as illustrated by the results on KDD98 dataset. As the number of classes increases, the GL-based methods become much faster than Twoing, with the turning point being around $k=7$. For datasets with 15 classes our criteria are 30-300 times faster. We also ran experiments using all the classes available in both the {\tt S.F. Crime} and {\tt Shelter} datasets (39 and 22, respectively). Twoing can not be executed in a reasonable time with that many classes, while GLSG and GL$\chi^2$ ran in approximately 100 seconds on the {\tt S.F. Crime} dataset and  300 seconds on the {\tt Shelter} dataset.  This behavior for the Twoing criterion is not surprising, since its running time has an exponential dependence of the number of classes $k$. Nonetheless, since the execution time for our criteria in this experiment grew in an approximately linear fashion with $k$, it suggests that they can also be used with datasets that have a much larger number of classes. It is also interesting to note that the aggregated attributes usually appeared at or near the root of the decision trees. Lastly, the running time for the GW-based criteria were usually one or two orders of magnitude larger than the others. The only clear exception was in the CoverType dataset, where the number of samples is very large while the attributes’ number of values is much smaller.

TODO: adicionar acima os resultados e analise dos criterios em depth 5 pra SF Crime, Shelter e Reuters.

Since the GW-criteria performe much worse in terms of execution time and yield comparable or worse accuracy than the other criteria, they were not used in any experiments that follow.

\begin{table*}[]
\small
\centering
\caption{Average time in seconds of a 3-fold cross validation
for building decision trees with depth at most 5.
The fastest method for each dataset is bold faced.}
\begin{tabular}{c|c|c|c|c|c|c|c}
Dataset             & k  & Twoing    & GWSG  & GW$\chi^2$ & GLSG      & GL$\chi^2$ & PC-ext     \\ \hline
{\tt Adult}         & 2  & {\bf  2.7} & 41   & 88         & 3         & 4          & 2.8        \\
{\tt Mushroom}      & 2  & {\bf 0.6} & 6.9  & 8.6         & 0.9       & 1          &            \\
{\tt KDD98}-2       & 2  & {\bf 4}   & 2162 & 3579        & 44        & 44         &            \\
{\tt Contracep}     & 3  & {\bf 0.1} & 0.6  & 1           & 0.1       & 0.1        & 0.1        \\
{\tt Contracep-Ext} & 3  & {\bf 0.1} & 3.3  & 13          & 0.2       & 0.2        & 0.2        \\
%{\tt KDD98}-3       & 3  & {\bf 5.2} & 2289 & 4342        & 60        & 56         &            \\
{\tt Cars}          & 4  & {\bf 0.1} & 2.5  & 2.4         & 0.1       & 0.1        & 0.1        \\
{\tt Cars-Ext}      & 4  & 0.2       & 7.7  & 11          & 0.3       & 0.3        & {\bf 0.2}  \\
{\tt Nursery}       & 5  & {\bf 0.8} & 4.7  & 5           & 1         & 0.9        &            \\
{\tt Nursery-Ext}   & 5  & {\bf 1.1} & 76   & 148         & 3.3       & 2.6        &            \\
%{\tt KDD98}-5       & 5  & {\bf 11}  & 2469 & 4956        & 81        & 63         &            \\
{\tt CoverType}     & 7  & 349       & 265  & {\bf 179}   & 245       & 308        &            \\
{\tt CoverType-Ext} & 7  & 213       & 341  & 213         & {\bf 182} & 296        &            \\
{\tt KDD98}-9       & 9  & 132       & 3410 & 5899        & 97        & {\bf 74}   &            \\ 
{\tt Poker}         & 10 & 7.4       & 6.5  & 11.2        & 2.2       & {\bf 2.1}  &            \\
{\tt Shelter-15}    & 15 & 3599      &      &             & {\bf149.9}& 166.3      &            \\   
{\tt S.F. Crime-15} & 15 & 638.2     &      &             & 40.7      & {\bf 40.2} &            \\ 
{\tt Phonemes}      & 15 & 1343      &      &             &      4.4  & 5.5        &       
\end{tabular}
\label{tab:time}
\end{table*}



Table \ref{tab:CrossVal}.(b) presents the 
comparison between our GL methods and Twoing in another scenario,
where we use $c_{quad}$, one of the bias-free   criterion proposed in \cite{Hothorn:2006:URP}, to select the attribute at each node of the tree. 
Then, both Twoing and our methods are  used only for splitting the chosen attribute,  which allows for a  more direct comparison of their splitting ability. Once again we observed a balance between the different criteria, with a small advantage towards our methods. Perhaps surprisingly, the bias free approach had significantly worse results for the datasets with extended attributes. This experiment also showed that it is not possible to run $c_{quad}$ in reasonable time for the {\tt Shelter-15} dataset. This happens because it calculates a pseudoinverse of a matrix whose dimension grows with the number of values and classes, which is infeasible for large $n$ and $k$.

TODO: adicionar acima os resultados e analise dos criterios com ctree em depth 5 pra SF Crime, Shelter e Reuters.


\begin{table*}
\small
\centering
    \caption{Average accuracy and statistical tests  for  Conditional Inference trees 
with depth at most 5 using only nominal attributes. The best accuracy for each dataset is bold-faced.}
\label{tab:CrossValCTree}
\begin{tabular}{c|cc|cc|cc|cc} 
Dataset  &   \multicolumn{2}{c|}{CI-Twoing} &   \multicolumn{2}{c|}{CI-GLSG} & \multicolumn{2}{c|}{CI-GL$\chi^2$}& \multicolumn{2}{c}{CI-PC-ext} \\  \hline   
% Dataset          &        CI-Twoing       &        CI-GLSG          &      CI-GL$\chi^2$      &       CI-PC-ext
{\tt Adult}        &{\bf 81.96} &{\bf  (2)} & 81.61       & (0)       & 81.77       & (1)       &  79.98      &           \\
{\tt Mushroom}     &86.97       & (0)       &{\bf  94.79 }& {\bf (2)} & 90.15       & (1)       &             &           \\
{\tt KDD98}-2      &81.29       & (0)       & {\bf 82.34 }& {\bf (2) }& 81.68       & (1)       &  81.35      &           \\
%{\tt KDD98}-3      &65.28       & (0)       & {\bf 65.8  }& {\bf (2) }& 65.07       & (0)       &             &           \\
%{\tt KDD98}-5      &49.23       & (0)       & {\bf 49.58} & {\bf (2)} & 49.27       & (0)       &             &           \\
{\tt KDD98}-9      &41.84       & (0)       & 42          & (0)       & {\bf 42.26} & {\bf (1)} &             &           \\
{\tt Nursery}      &{\bf 88.48} & {\bf (1)} & 88.3        & (0)       & 88.47       & {\bf (1)} &             &           \\
{\tt Nursery-Ext}  &{\bf 88.48} & {\bf (1)} & 88.26       & (0)       & 88.47       & {\bf (1)} &             &           \\
{\tt Cars}         &86.51       & {\bf (1)} & 85.02       & (0)       & {\bf 86.57} & {\bf (1)} & 86.48       &           \\
{\tt Cars-Ext}     &88.27       & {\bf (1) }& 88.27       & (0)       & {\bf 88.32} & {\bf (1)} & 88.26       &           \\
{\tt Contracep}    &43.83       & (0)       & {\bf 44.12} & {\bf (1)} & 43.87       & (0)       & 43.63       &           \\
{\tt Contracep-Ext}&43.39       & (0)       & {\bf 43.81} & {\bf (1)} & 43.76       & {\bf (1)} & 43.21       &           \\
{\tt CoverType}    &54.1        & (0)       & 54.1        & (0)       & 54.1        & (0)       & 54.1        &           \\
{\tt CoverType-Ext}&54.1        & (0)       & 54.1        & (0)       & 54.1        & (0)       & 54.1        &           \\
{\tt Poker}        &{\bf 51.05} & {\bf (1)} & 50.56       & (0)       & 50.91       & {\bf (1)} &             &           \\  
{\tt Shelter-15}   &            &           &             &           &             &           &             &           \\   
{\tt S.F. Crime-15}& 21.59      & {\bf (2)} & 20.53       & (0)       & 21.49       & (1)       &             &           \\ 
{\tt Phonemes}     & 23.12      & (1)       & 22.52       & (0)       & 23.8        & (2)       &             &           \\ 
\hline
Average (Sum)      &            &           &             &           &             &           &             & 
       \end{tabular}
\end{table*}


Table \ref{exp:secondsetnumeric} shows experiments  similar to those presented at Table \ref{tab:CrossVal}.(a), but now
using also the numeric attributes. We observed a significant gain in terms of accuracy for all datasets except for KDD98-2. 
Also note that GLSG was inferior to both Twoing and GL$\chi^2$.

TODO: adicionar acima os resultados e analise dos criterios em arvore com numerico em depth 5 pra SF Crime e Shelter.

\begin{table}
\small
\caption{Average accuracy and statistical test results for  Decision Trees using both nominal and numeric attributes.}
\centering
\begin{tabular}{c|cc|cc|cc|cc} 
Dataset            &\multicolumn{2}{c|}{Twoing} & \multicolumn{2}{c|}{GLSG} & \multicolumn{2}{c|}{GL$\chi^2$} & \multicolumn{2}{c}{PC-ext}\\  \hline   
{\tt Adult}        & 84.2           & (1)       & 82.4       & (0)          &  84.2       & (1)               &  {\bf 84.4} &             \\
{\tt KDD98}-2      & 80.9           & (0)       & {\bf 81.9 }& {\bf (1)}    & {\bf 81.9}  & {\bf (1)}         &             &             \\ 
%{\tt KDD98}-3      & {\bf 69.5 }    &{\bf  (2)} & 67.9       & (0)          & 69.1        & (1)               &             &             \\ 
%{\tt KDD98}-5      & 54.6           &  (0)      & 54.5       &  (0)         & {\bf 55.3 } & {\bf  (2) }       &             &             \\ 
{\tt KDD98}-9      & 45.7           &   (1)     & 45.3       &  (0)         & {\bf 48.7 } &{\bf  (2)  }       &             &             \\ 
{\tt Contracep}    & 54.1           &  (1)      & 52.9       &  (0)         & 54          & (1)               & {\bf 55.11} &             \\ 
{\tt Contracep-Ext}& 51.7           &  (0)      & 52.3       &  (0)         & 53.3        & (2)               & {\bf 53.31} &             \\ 
{\tt CoverType}    &  {\bf 70.3 }   &  {\bf (2)}& 68.5       &  (0)         & 69.2        & (1)               &             &             \\ 
{\tt CoverType-Ext}& {\bf 70.9}     & {\bf (2) }& 70.3       &  (1)         & 69.2        & (0)               &             &             \\ 
{\tt Shelter-15}   &  53.7          &           & 51.9       &              & {\bf 54.6 } &                   &             &             \\   
{\tt S.F. Crime-15}& 23.5           &           & 23.2       &              & {\bf 23.6 } &                   &             &             \\ 
\hline
Average (Sum)      & 59.44          &           & 58.74      &              & 59.86       &                   &             &

\end{tabular}
\label{exp:secondsetnumeric}
\normalsize
\end{table}


\subsection{Maximum Depth 16}


In this subsection we explore the same experiments, except now the maximum depth allowed is 16. 


Table \ref{tab:CrossVal-a} presents  the results of an experiment to
compare the accuracy of  Decision Trees built by our GL-methods with those built by Twoing, using just the nominal attributes of the datasets.
In general there was a balance between Twoing and GL$\chi^2$, with GLSG being slightly worse. This suggests that GLSG loses competitiveness as the tree depth increases.

Table \ref{tab:CrossVal-b} presents the 
comparison between  our  methods and Twoing in the scenario where we use $c_{quad}$ to select the attribute at each node of the tree, and the criteria are used only for splitting the chosen attribute. Again, we observed a balance between Twoing and GL$\chi^2$. This experiment also showed that it is not possible to run $c_{quad}$ in reasonable time for the {\tt Shelter-15} dataset. This happens because it calculates a pseudoinverse of a matrix whose dimension grows with the number of values and classes, which is infeasible for large $n$ and $k$.


\begin{table}
\small
\caption{Average accuracy and statistical tests  for  decision trees 
with depth at most 16 using only nominal attributes. The best accuracy for each dataset is bold-faced, even when multiple criteria have the same accuracy in the table because of rounding.}
\centering
\begin{tabular}{c|cc|cc|cc|cc} 
Dataset             & \multicolumn{2}{c|}{Twoing} &  \multicolumn{2}{c|}{GLSG}  & \multicolumn{2}{c|}{GL$\chi^2$} & \multicolumn{2}{c}{PC-ext}\\ \hline
{\tt Adult}         &  {\bf 82.52} & {\bf (2)}    &  82.38       &  (0)         &  82.43       & (0)              &            &              \\
{\tt Mushroom}      &  {\bf 100}   & {\bf (1)}    &  100         &  (0)         &  99.99       & (0)              &            &              \\
{\tt KDD98}-2       &  79.41       & (0)          &  {\bf 80.65} & {\bf (2)}    &  79.73       & (0)              &            &              \\
%{\tt KDD98}-3       &  62.61       & (0)          &  63.33       & {\bf (1)}    &  {\bf 63.56} & {\bf (1)}        &            &              \\
%{\tt KDD98}-5       &  46.46       & (0)          &  46.39       & (0)          &  {\bf 47.53} & {\bf (2)}        &            &              \\
{\tt KDD98}-9       &  38.54       & (1)          &  37.97       & (0)          &  {\bf 39.68} & {\bf (2)}        &            &              \\
{\tt Nursery}       &  93.51       & (1)          &  92.39       & (0)          &  {\bf 93.74} & {\bf (2)}        &            &              \\
{\tt Nursery-ext}   &  95.83       & (1)          &  94.79       & (0)          &  {\bf 96.02} & {\bf (2)}        &            &              \\
{\tt Cars}          &  92.82       & {\bf (1)}    &  90.51       & (0)          &  {\bf 92.88} & {\bf (1)}        &            &              \\
{\tt Cars-ext}      &  {\bf 96.49} & {\bf (2)}    &  90.89       & (0)          &  94.44       & (1)              &            &              \\
{\tt Contracep}     &  43.5        & (0)          &  43.83       & (0)          &  {\bf 43.85} & {\bf (1)}        &            &              \\
{\tt Contracep-ext} &  43.15       & (0)          &  {\bf 44.32} & {\bf (1)}    &  43.72       & (0)              &            &              \\
{\tt CoverType}     &  64.09       & (1)          &  {\bf 64.59} & {\bf (2)}    &  63.61       & (0)              &            &              \\
{\tt CoverType-ext} &  {\bf 65.08} & {\bf (1)}    &  {\bf 65.08} & {\bf (1)}    &  65.05       & (0)              &            &              \\
{\tt Poker}         &  {\bf 52.24} & {\bf (2)}    &  49.88       & (0)          &  51.83       & (1)              &            &              \\ 
{\tt Shelter-15}    &  47.64       & (1)          &  46.52       & (0)          & {\bf 48.09}  & {\bf (2)}        &            &              \\ 
{\tt S.F. Crime-15} &  22.08       & {\bf (1)}    &  22.06       & (0)          & {\bf 22.08}  & {\bf (1)}        &            &              \\   
{\tt Phonemes}      &  {\bf 37.13} & {\bf (2)}    &  35.9        & (0)          & 35.75        & (0)              &            &              \\
\hline
Average (Sum)       &  65.87       &  (17)        &  65.11       & (6)          & 65.81        & (13)             &            &
\end{tabular}
\normalsize
\label{tab:CrossVal-a}
\end{table}

    \begin{table}
    \small
      \centering
        \caption{Average accuracy and statistical tests  for  conditional inference trees 
with depth at most 16 using only nominal attributes. The best accuracy for each dataset is bold-faced, even when multiple criteria have the same accuracy in the table because of rounding.}

\begin{tabular}{c|cc|cc|cc|cc} 
Dataset             & \multicolumn{2}{c|}{CI-Twoing} &   \multicolumn{2}{c|}{CI-GLSG} & \multicolumn{2}{c|}{CI-GL$\chi^2$} & \multicolumn{2}{c}{CI-PC-ext}\\  \hline   
{\tt Adult}         & 82.33      &  {\bf (1)}        &   82.18      & (0)             & {\bf 82.35} &  {\bf (1)}           & 82.28       &                \\
{\tt Mushroom}      &  99.55     &  (0)              &   99.6       & (0)             & {\bf 99.64} &  {\bf (1)}           &             &                \\
{\tt KDD98}-2       &79.92       &  (0)              &  {\bf 81.53} & {\bf (2)}       &  80.65      &  (1)                 &             &                \\
%{\tt KDD98}-3       &63.72       &  (0)              &  {\bf 64.66} & {\bf (2)}       &  63.56      &  (0)                 &             &                \\
%{\tt KDD98}-5       &46.88       &  (0)              &  {\bf 48}    & {\bf (1)}       &  47.75      &  {\bf (1)}           &             &                \\
{\tt KDD98}-9       & 39.44      &  (0)              &  {\bf 40.65} & {\bf (2)}       &  40.01      &  (1)                 &             &                \\
{\tt Nursery}       & 93.55      &  {\bf (1)}        &   92.22      & (0)             & {\bf 93.56} &  {\bf (1)}           &             &                \\
{\tt Nursery-ext}   & 93.64      &  {\bf (1)}        &   92.32      & (0)             & {\bf 93.65} &  {\bf (1)}           &             &                \\
{\tt Cars}          & 92.74      &  (1)              &   87.19      & (0)             & {\bf 92.92} &  {\bf (2)}           & 92.67       &                \\
{\tt Cars-ext}      & 93.12      &  (1)              &   87.38      & (0)             & {\bf 93.34} &  {\bf (2)}           & 93.08       &                \\
{\tt Contracep}     &43.82       &  (0)              &   {\bf 44.05}& (0)             &  43.81      &  (0)                 & 43.59       &                \\
{\tt Contracep-ext} &43.21       &  (0)              &   {\bf 43.78}& {\bf (1)}       &  43.72      &  {\bf (1)}           & 43.07       &                \\
{\tt CoverType}     &61.88       &  (0)              &   61.88      & (0)             &  61.88      &  (0)                 & 61.88       &                \\
{\tt CoverType-ext} &61.88       &  (0)              &   61.88      & (0)             &  61.88      &  (0)                 &             &                \\
{\tt Poker}         &{\bf 51.4}  &  {\bf (2)}        &   50.67      & (0)             &  50.84      &  {\bf (1)}           &             &                \\
{\tt Shelter-15}    &---         &                   &   ---        &                 &   ---       &                      & ---         &                \\
{\tt S.F. Crime-15} &22.07       &  (0)              &  22.07       & (0)             & {\bf 22.07} &  {\bf (1)}           &             &                \\
{\tt Phonemes}      & {\bf 33.91}&  {\bf (2)}        & 31.26        & (0)             &  33.32      &  (1)                 &             &                \\
\hline
Average (Sum)       & 66.16      &  (9)              & 65.24        & (5)             &  66.24      & (14)                 &             &
       \end{tabular}
\normalsize
\label{tab:CrossVal-b}
\end{table}


Table \ref{tab:time} shows the running time of Twoing, GL$\chi^2$ and GLSG  when
they are used for both selecting and splitting purposes (the experiment of Table \ref{tab:CrossVal}.(a)).
When the number of classes is small all the criteria have very similar execution time, with Twoing being faster only on the {\tt KDD98}-2 dataset. As the number of classes increases, the GL-based methods become much faster than Twoing, with the turning point being around $k=7$. For datasets with 15 classes our criteria are 30-300 times faster. We also ran experiments using all the classes available in both the {\tt S.F. Crime} and {\tt Shelter} datasets (39 and 22, respectively). Twoing can not be executed in a reasonable time with that many classes, while GLSG and GL$\chi^2$ ran in approximately 100 seconds on the {\tt S.F. Crime} dataset and  300 seconds on the {\tt Shelter} dataset. Nonetheless, since the execution time for our criteria in this experiment grew in an approximately linear fashion with $k$, it suggests that they can also be used with datasets that have a much larger number of classes. It is also interesting to note that the aggregated attributes usually appeared at or near the root of the decision trees.


\begin{table}[]
\small
\caption{Average time in seconds of a 3-fold cross validation
for building decision trees with depth at most 16.
The fastest method for each dataset is bold-faced.}
\centering
\begin{tabular}{c|c|c|c|c|c}
Dataset             & k  & Twoing        & GLSG      & GL$\chi^2$  & PC-ext \\ \hline
{\tt Adult}         & 2  & 4.3           & {\bf 4.3} & 7.1         &        \\
{\tt Mushroom}      & 2  & 0.7           & {\bf 0.6} & 0.9         &        \\
{\tt KDD98}-2       & 2  & {\bf 10.8}    & 57.8      & 60.8        &        \\
{\tt Contracep}     & 3  & 0.2           & 0.2       & {\bf 0.1}   &        \\
{\tt Contracep-Ext} & 3  & {\bf 0.2}     & 0.3       & 0.3         &        \\
%{\tt KDD98}-3       & 3  & {\bf 12.5}    & 73.3      & 82.4        &        \\
{\tt Cars}          & 4  & 0.3           & 0.2       & {\bf 0.2}   &        \\
{\tt Cars-Ext}      & 4  & {\bf 0.3}     & 0.4       & 0.4         &        \\
{\tt Nursery}       & 5  & 1.6           &  {\bf 1.4}      &  1.4  &        \\
{\tt Nursery-Ext}   & 5  & {\bf 1.7}           &  3.9 & 3.6        &        \\
%{\tt KDD98}-5       & 5  & {\bf 22.2}    & 100.8     & 82.6        &        \\
{\tt CoverType}     & 7  & 846.8         &{\bf 373.4}& 969.6       &        \\
{\tt CoverType-Ext} & 7  & 338.6         &{\bf 280.5}& 505.7       &        \\
{\tt KDD98}-9       & 9  & 209.9         & 119.9     & {\bf 112.5} &        \\
{\tt Poker}         & 10 & 10.7          & 3.9       & {\bf 3.7}   &        \\ 
{\tt Shelter-15}    & 15 & 5183.3        &{\bf 155}  & 165.7       &        \\   
{\tt S.F. Crime-15} & 15 & 2667.9        & 94.2      &{\bf 79.6}   &        \\ 
{\tt Phonemes}      & 15 & 3738.6        &{\bf 8.7}  & 10.2        &
\end{tabular}
\label{tab:time}
\end{table}


Table \ref{exp:secondsetnumeric}
shows experiments  similar to those presented at Table \ref{tab:CrossVal}.(a), except now it also uses the numeric attributes.
We observed a significant gain in terms of accuracy for almost all datasets. 
The performance of GL$\chi^2$ was competitive with that of Twoing for all datasets but {\tt KDD98-9} and {\tt CoverType}, where it was
considerably better and worse, respectively.


\begin{table}
\small
\caption{Average accuracy and statistical test results for  Decision Trees using both nominal and numeric attributes with depth at most 16.}
\centering
\begin{tabular}{c|cc|cc|cc|cc} 
Dataset              &        \multicolumn{2}{c|}{Twoing} &   \multicolumn{2}{c|}{GLSG} &   \multicolumn{2}{c|}{GL$\chi^2$} & \multicolumn{2}{c}{PC-ext}  \\  \hline   
{\tt Adult}          &  83            &  (1)              &  77.34      &  (0)          &  83.21       &  (2)               & {\bf 83.28} &               \\
{\tt KDD98}-2        &  {\bf 76.67}   &  {\bf (2)}        &  76.36      &  (0)          &  76.04       &  (1)               &             &               \\
%{\tt KDD98}-3        &  63.24         &  (1)              &  62.21      &  (0)          &  {\bf 63.94} &  {\bf (2)}         &             &              \\
%{\tt KDD98}-5        &  47.5          &  (1)              &  46.35      &  (0)          &  {\bf 49.71} &  {\bf (2)}         &             &              \\
{\tt KDD98}-9        &  37.73         &  (1)              &  37.49      &  (0)          &  {\bf 43.44} &  {\bf (2)}         &             &               \\
{\tt Contracep}      &  48.78         &  (1)              &  48.01      &  (0)          &  48.66       &  (1)               & {\bf 48.93} &               \\
{\tt Contracep-Ext}  &  48.53         &  (0)              &  48.15      &  (0)          &  {\bf 48.6}  &  (0)               &  48.52      &               \\
{\tt CoverType}      &  85.14         &  (1)              &  {\bf 90.32}&  {\bf (2)}    &  81.38       &  (0)               &             &               \\
{\tt CoverType-Ext}  &  89.1          &  (1)              &  {\bf 92.03}&  {\bf (2)}    &  82.46       &  (0)               &             &               \\
{\tt Shelter-15}     &  53.79         &  (1)              &  52         &  (0)          &  {\bf 54.4}  &  {\bf (2)}         &             &               \\   
{\tt S.F. Crime-15}  &  26.66         &  (0)              &  26.71      &  (1)          &  {\bf 27.13} &  {\bf (2)}         &             &               \\
\hline
Average (Sum)        &   61.04        &  (8)              & 60.93       & (5)           &   60.59      & (10)               &             &

\end{tabular}
\label{exp:secondsetnumeric}
\normalsize
\end{table}


\newpage

\chapter{Conclusions}
\label{chap:conclusions}

In this paper we proposed a framework for
designing splitting criteria for handling 
multi-valued nominal attributes.
Criteria derived from our framework 
can be implemented to run in polynomial time in
$n$ and $k$, with 
theoretical guarantee of producing a split that is close to the optimal
one. This is the only known criteria that have all these characteristics simultaneously.

Experiments over 11 datasets
suggest that the GL$\chi^2$ criterion, obtained from our framework, is competitive with the well-established Twoing criterion in terms of both accuracy and speed for datasets with a small number of classes ($k \leq 7$). It is also much faster than Twoing when the number of classes is greater than 10, while keeping a comparable accuracy. Even though the PC-ext criterion does not have a theoretical guarantee, the experiments also show that it has some advantage in terms of accuracy and speed over the other methods, except when using it in the conditional inference tree framework. This suggests that PC-ext is very good in terms of comparing different attributes among themselves, but not in terms of finding the best split for a given attribute. On the other hand, our criteria performed well in all the experiments.
 
Therefore, our methods are an interesting alternative to deal with
datasets with a large number of classes that contain nominal attributes with a large
number of different values, since those cannot be properly handled by Twoing due to its exponential running time dependence on the number of classes. In practice, one should also consider using PC-ext and comparing the results obtained.

Furthermore, our experiments also reinforce
the potential of  aggregating attributes as a tool 
for improving the accuracy of decision trees.
An interesting topic for  future research is evaluating the behavior of our criteria in boosted tree methods.
Another direction for future work is developing new methods for automatic aggregating attributes, or improving the available ones.


\begin{raggedright}
	\bibliography{thesis}
\end{raggedright}

\end{document}