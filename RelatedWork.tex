\newpage

\chapter{Related Work}
\label{chap:relatedwork}

Many splitting criteria have been proposed to 
deal with continuous and nominal attributes.
Arguably, the Gini gain, used by CART,  and entropy based measures, such  as 
the Information Gain, adopted by C4.5, are among
the most popular \cite{books/sp/datamining2005/RokachM05,
Loh2014,series/sbcs/BarrosCF15}.

There has been  some investigation on 
methods to compute the best split efficiently 
\cite{Breiman84,Chou:91,BPKN:92,journals/datamine/CoppersmithHH99}.
For the 2-class problem,  \cite{Breiman84} proved a theorem which states that an optimal
binary partition, for a certain class of splitting criteria,
can be determined in linear time on $n$, the number of distinct values of the attributes, after ordering.
The Gini  Gain belongs to this class.
The  other  three papers generalize
this theorem  in different directions 
and show necessary conditions that are satisfied by optimal partitions for a certain class of splitting criteria. 
These conditions, though useful to restrict the set of partitions
that need to be considered, do not yield  a method that
is efficient (polynomial time) for large values of $n$ and $k$.  These papers also  present  heuristics, without approximation guarantee, to obtain good splits.

Other proposals to  speed up the attribute selection phase
 include  \cite{MolaSiciliano1997,Shih2001}. 
The first presents a simple  heuristic
to reduce the number of binary splits considered to
choose the best nominal variable among the $m$ available ones.
 The second   extends the method for another class
of impurity measures.

In order to properly
handle nominal attributes with a large number of values,
apart from efficiently computing good splits, it is
important to prevent bias in the attribute selection.
Indeed, it is widely  known that many splitting criteria have bias toward
attributes with a large number of values. There are some  proposals available
to cope with this issue 
\cite{conf/icml/DobraG01,Shih2004,Hothorn:2006:URP}. 
This topic, though relevant, is not the focus of our paper.